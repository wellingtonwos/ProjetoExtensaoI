%% abtex2-modelo-trabalho-academico.tex, v-1.9.2 laurocesar
%% Copyright 2012-2014 by abnTeX2 group at http://abntex2.googlecode.com/ 
%%
%% This work may be distributed and/or modified under the
%% conditions of the LaTeX Project Public License, either version 1.3
%% of this license or (at your option) any later version.
%% The latest version of this license is in
%%   http://www.latex-project.org/lppl.txt
%% and version 1.3 or later is part of all distributions of LaTeX
%% version 2005/12/01 or later.
%%
%% This work has the LPPL maintenance status `maintained'.
%% 
%% The Current Maintainer of this work is the abnTeX2 team, led
%% by Lauro César Araujo. Further information are available on 
%% http://abntex2.googlecode.com/
%%
%% This work consists of the files abntex2-modelo-trabalho-academico.tex,
%% abntex2-modelo-include-comandos and abntex2-modelo-references.bib
%%

% ------------------------------------------------------------------------
% ------------------------------------------------------------------------
% abnTeX2: Modelo de Trabalho Academico (tese de doutorado, dissertacao de
% mestrado e trabalhos monograficos em geral) em conformidade com 
% ABNT NBR 14724:2011: Informacao e documentacao - Trabalhos academicos -
% Apresentacao
% ------------------------------------------------------------------------
% ------------------------------------------------------------------------

\documentclass[
	% -- opções da classe memoir --
	12pt,				% tamanho da fonte
	openright,			% capítulos começam em pág ímpar (insere página vazia caso preciso)
	twoside,			% para impressão em verso e anverso. Oposto a oneside
	a4paper,			% tamanho do papel. 
	% -- opções da classe abntex2 --
	%chapter=TITLE,		% títulos de capítulos convertidos em letras maiúsculas
	%section=TITLE,		% títulos de seções convertidos em letras maiúsculas
	%subsection=TITLE,	% títulos de subseções convertidos em letras maiúsculas
	%subsubsection=TITLE,% títulos de subsubseções convertidos em letras maiúsculas
	% -- opções do pacote babel --
	english,			% idioma adicional para hifenização
	french,				% idioma adicional para hifenização
	spanish,			% idioma adicional para hifenização
	brazil				% o último idioma é o principal do documento
	]{abntex2}

% ---
% Pacotes básicos 
% ---
\usepackage{lmodern}			% Usa a fonte Latin Modern			
\usepackage[T1]{fontenc}		% Selecao de codigos de fonte.
\usepackage[utf8]{inputenc}		% Codificacao do documento (conversão automática dos acentos)
\usepackage{lastpage}			% Usado pela Ficha catalográfica
\usepackage{indentfirst}		% Indenta o primeiro parágrafo de cada seção.
\usepackage{color}				% Controle das cores
\usepackage{graphicx}			% Inclusão de gráficos
\usepackage{microtype} 			% para melhorias de justificação
% ---
		

% ---
% Pacotes de citações
% ---
\usepackage[brazilian,hyperpageref]{backref}	 % Paginas com as citações na bibl
\usepackage[alf]{abntex2cite}	% Citações padrão ABNT
\usepackage{ragged2e}

% --- 
% CONFIGURAÇÕES DE PACOTES
% --- 

% ---
% Configurações do pacote backref
% Usado sem a opção hyperpageref de backref
\renewcommand{\backrefpagesname}{Citado na(s) página(s):~}
% Texto padrão antes do número das páginas
\renewcommand{\backref}{}
% Define os textos da citação
\renewcommand*{\backrefalt}[4]{
	\ifcase #1 %
		Nenhuma citação no texto.%
	\or
		Citado na página #2.%
	\else
		Citado #1 vezes nas páginas #2.%
	\fi}%
% ---

% ---
% Informações de dados para CAPA e FOLHA DE ROSTO
% ---
\titulo{CarneUp: Sistema de Gestão de Açougue de Pequeno e Médio Porte}


\autor{BRENO DIAS OLIVEIRA SP3015645\\FELIPE DALBOSCO PALUDO SP3123766\\GUSTAVO GOUVEA ANDRADE SP3076725\\PEDRO AUGUSTO SILVA SANTOS SP3098559\\RAQUEL CORREIA DA SILVA 
SP3098567\\WELLINGTON OLIVEIRA DE SOUSA SP3161307}

\local{São Paulo - SP - Brasil}
\data{2025}
\orientador{Marcelo Tavares de Santana}

\instituicao{%
   IFSP- Instituto Federal de Educação, Ciência e Tecnologia
 Câmpus São Paulo
  \par
   Tecnologia em Análise e Desenvolvimento de Sistemas}

% O preambulo deve conter o tipo do trabalho, o objetivo, 
% o nome da instituição e a área de concentração 
\preambulo{ Trabalho apresentado ao Instituto Federal de Educação, Ciência e Tecnologia de São Paulo, como parte dos requisitos para a conclusão da disciplina Projeto Integrado I.}
% ---


% ---
% Configurações de aparência do PDF final

% alterando o aspecto da cor azul
\definecolor{blue}{RGB}{41,5,195}

% informações do PDF
\makeatletter
\hypersetup{
     	%pagebackref=true,
		pdftitle={\@title}, 
		pdfauthor={\@author},
    	pdfsubject={\imprimirpreambulo},
	    pdfcreator={LaTeX with abnTeX2},
		pdfkeywords={abnt}{latex}{abntex}{abntex2}{trabalho acadêmico}, 
		colorlinks=true,       		% false: boxed links; true: colored links
    	linkcolor=blue,          	% color of internal links
    	citecolor=blue,        		% color of links to bibliography
    	filecolor=magenta,      		% color of file links
		urlcolor=blue,
		bookmarksdepth=4
}
\makeatother
% --- 

% --- 
% Espaçamentos entre linhas e parágrafos 
% --- 

% O tamanho do parágrafo é dado por:
\setlength{\parindent}{1.3cm}

% Controle do espaçamento entre um parágrafo e outro:
\setlength{\parskip}{0.2cm}  % tente também \onelineskip

% ---
% compila o indice
% ---
\makeindex
% ---

% ----
% Início do documento
% ----
\begin{document}

% Retira espaço extra obsoleto entre as frases.
\frenchspacing 

% ----------------------------------------------------------
% ELEMENTOS PRÉ-TEXTUAIS
% ----------------------------------------------------------
% \pretextual

% ---
% Capa
% ---
\imprimircapa
% ---

% ---
% Folha de rosto
% (o * indica que haverá a ficha bibliográfica)
% ---
\imprimirfolhaderosto*
% ---

% ---
% Inserir a ficha bibliografica
% ---

% Isto é um exemplo de Ficha Catalográfica, ou ``Dados internacionais de
% catalogação-na-publicação''. Você pode utilizar este modelo como referência. 
% Porém, provavelmente a biblioteca da sua universidade lhe fornecerá um PDF
% com a ficha catalográfica definitiva após a defesa do trabalho. Quando estiver
% com o documento, salve-o como PDF no diretório do seu projeto e substitua todo
% o conteúdo de implementação deste arquivo pelo comando abaixo:
%
% \begin{fichacatalografica}
%     \includepdf{fig_ficha_catalografica.pdf}
% \end{fichacatalografica}





% ---
% RESUMOS
% ---

% resumo em português
\setlength{\absparsep}{18pt} % ajusta o espaçamento dos parágrafos do resumo
\begin{resumo}


 Este trabalho propõe o desenvolvimento do CarneUp, um sistema de gerenciamento de açougues focado em digitalizar e otimizar as operações de venda, compra e controle de estoque de carnes. A ausência de controle digital em muitas boutiques e açougues de pequeno e médio porte, onde as vendas e o gerenciamento de estoque ainda são feitos de forma analógica em cadernos, gera ineficiência e perda de informações cruciais. O CarneUp visa solucionar este problema, oferecendo uma plataforma robusta e intuitiva para o gerenciamento completo do negócio, desde o registro de vendas com captura automática de peso/valor via código de barras, até o controle de validade e a rastreabilidade por fornecedor. O sistema é projetado para açougues de pequeno e médio porte, fornecendo ferramentas essenciais para aumentar a precisão do estoque, melhorar o controle financeiro por peça e reduzir o desperdício, garantindo assim a eficiência e a competitividade do negócio. 

 \textbf{Palavras-chaves}: Gestão de açougue, Controle de estoque de carnes, Sistema de vendas, Software de açougue.
\end{resumo}

% resumo em inglês
\begin{resumo}[Abstract]
 \begin{otherlanguage*}{english}
   This work proposes the development of CarneUp, a butcher shop management system focused on digitizing and optimizing sales, purchasing, and meat stock control operations. The lack of digital control in many small and medium-sized butcher shops, where sales and inventory management are still carried out analogically in notebooks, leads to inefficiency and loss of crucial information. CarneUp aims to solve this problem by offering a robust and intuitive platform for complete business management, from sales registration with automatic weight/value capture via barcode, to validity control and traceability by supplier. The system is designed for small and medium-sized butcher shops, providing essential tools to increase inventory accuracy, improve financial control per piece, and reduce waste, thereby ensuring business efficiency and competitiveness.

   \vspace{\onelineskip}
 
   \noindent 
   \textbf{Key-words}: Butcher shop management, Meat inventory control, Sales system, Butcher shop software.
 \end{otherlanguage*}
\end{resumo}



% ---
% inserir lista de ilustrações
% ---
\pdfbookmark[0]{\listfigurename}{lof}
\listoffigures*
\cleardoublepage
% ---

% ---
% inserir lista de tabelas
% ---
\pdfbookmark[0]{\listtablename}{lot}
\listoftables*
\cleardoublepage
% ---

% ---
% inserir lista de abreviaturas e siglas
% ---
\begin{siglas}
  \item[AWS] Amazon Web Services
  \item[API] Application Programming Interface
  \item[CRUD] Create, Read, Update, Delete
  \item[FDD] Feature Driven Development
  \item[MER] Modelo Entidade-Relacionamento
  \item[MVP] Produto Mínimo Viável
  \item[Scrum] (Metodologia Ágil)
  \item [PDV] Ponto de Venda
  \item [POC]Prova de Conceito
  \item [RNF] Requisito Não Funcional
  \item [RF] Requisito Funcional
  \item [SaaS] Software as a Service
  \item [SSL/TLS] Secure Sockets Layer/Transport Layer Security
\end{siglas}
% ---



% ---
% inserir o sumario
% ---
\pdfbookmark[0]{\contentsname}{toc}
\tableofcontents*
\cleardoublepage
% ---



% ----------------------------------------------------------
% ELEMENTOS TEXTUAIS
% ----------------------------------------------------------
\textual





\chapter[Introdução]{Introdução}
% ---
% \section {Observação}
%  Eliminado o titulo 1.1.
% --- 

O principal objetivo do projeto CarneUp é desenvolver um sistema de gerenciamento digital para açougues de pequeno e médio porte. Este sistema deve permitir que os proprietários e funcionários digitalizem e otimizem a gestão completa de suas operações.

    Os objetivos específicos incluem:

\begin{itemize}
    \item Gerenciar Vendas: Permitir o registro rápido e eficiente de vendas de carnes.
    
    \item Controle de Estoque: Fornecer uma visão em tempo real do estoque de carnes, incluindo o controle das peças vendidas, suas validades e fornecedores.
    
    \item Digitalização de Pesagem: Integrar-se com códigos de barras gerados pela balança para capturar automaticamente o valor e o peso da peça no momento da venda.
    
    \item Otimização de Compras: Registrar o valor pago nas carnes e seus fornecedores para análises de custo e rastreabilidade.

\end{itemize}

% ---
\section{Problema e Solução Proposta}
% ---

Problema Identificado

O problema central abordado pelo CarneUp é a gestão analógica e ineficiente em açougues de pequeno e médio porte. Em muitas boutiques de carnes, o controle de vendas, o registro de compras, o controle de validade e o acompanhamento de estoque são realizados manualmente em cadernos e planilhas. Esta abordagem é propensa a erros, dificulta a rastreabilidade de peças por fornecedor ou validade, e impede uma análise precisa da margem de lucro por corte. A falta de um sistema digital resulta em desperdício de estoque vencido e em perda de dados valiosos para a tomada de decisão.\cite: {ref:acougues_pequeno_porte}


Solução Proposta (CarneUp)
O CarneUp é a solução proposta para digitalizar e centralizar a gestão do açougue. O sistema será uma aplicação web com foco em facilidade de uso para o público-alvo.

As funcionalidades chave do CarneUp include:

\begin{itemize}
\item Registro de Vendas Otimizado: Utilização de códigos de barras gerados pela balança para preenchimento automático do valor e peso da peça vendida.

\item Rastreabilidade Completa: Registro de dados como fornecedor, valor de compra e validade de cada lote ou peça de carne no estoque.

\item Controle de Estoque Inteligente: Alertas de validade próxima e visualização clara das peças em estoque.

\end{itemize}


% ---
\section{Justificativa}
% ---

A relevância do CarneUp reside na necessidade urgente de modernização do setor de açougues de pequeno e médio porte. Ao oferecer uma solução digital acessível, o CarneUp impacta diretamente:

\begin{itemize}
    \item Controle Financeiro e Lucratividade: O registro do valor pago por peça e o rastreio por fornecedor permitem uma análise de custo mais precisa, maximizando a margem de lucro e identificando cortes de maior rentabilidade.

    \item Redução de Desperdício (Extensão): O controle de validade digitalizado reduz significativamente as perdas de estoque devido ao vencimento, um problema comum na gestão manual de perecíveis. \cite{gestao_de_pereciveis}

    \item Eficiência Operacional: A automatização do registro de vendas via código de barras acelera o atendimento ao cliente e reduz a chance de erros humanos no caixa.\cite{erros_pdv}
\end{itemize}
A falta de um sistema digital especializado no controle detalhado de carnes, considerando fornecedor, valor de compra e validade em açougues menores, torna o CarneUp uma ferramenta de grande importância para a sobrevivência e crescimento desses negócios em um mercado cada vez mais competitivo.

% ---
\section{Análise da Concorrência}
% ---
O CarneUp se posiciona contra sistemas de gestão (ERP/PDV) mais genéricos, que não possuem a granularidade necessária para o controle de carnes com base em peso variável, validade e fornecedor, sendo adaptados para o segmento de açougues.

\subsection{Concorrente 1: ConnectPlug}

O ConnectPlug é um sistema de PDV (Ponto de Venda) e gestão comercial mais abrangente, atendendo diversos segmentos, incluindo açougues. Ele oferece funcionalidades como frente de caixa, gestão de estoque e emissão de notas fiscais. \cite{connectplug}

\begin{itemize}
    \item Foco: Solução completa de gestão comercial.

    \item Diferencial (para o CarneUp): Pode ser adaptado para açougues, mas pode não ter o foco e a profundidade de rastreabilidade (valor de compra, fornecedor, validade individualizada) que o CarneUp busca.
\end{itemize}

\subsection{Concorrente 2: Aliar Sistemas}

A Aliar Sistemas oferece soluções específicas para o varejo de carnes, com foco em açougues e frigoríficos. Seu sistema inclui gestão de produção, rastreabilidade e controle de estoque. \cite{concorrente_aliar}

\begin{itemize}
    \item Foco: Soluções específicas para o setor de carnes.

    \item Por focar em frigoríficos e grandes açougues, pode ter um custo elevado e complexidade excessiva para açougues de pequeno porte, que é o alvo do CarneUp.
\end{itemize}

\subsection{Concorrente 3: SOFTClass}

A SOFTClass oferece software de gestão para diversos segmentos do varejo, incluindo açougues. Possui módulos de frente de caixa, estoque e financeiro.\cite{softclass}

\begin{itemize}
    \item Foco: Solução de gestão empresarial para varejo.

    \item Semelhante ao ConnectPlug, pode ser mais um sistema de PDV adaptado, carecendo da simplicidade e do foco no ciclo de vida da carne (compra/validade/fornecedor) que é o diferencial do CarneUp.
\end{itemize}


\subsection{Comparativo}
O quadro abaixo resume as principais funcionalidades e destaca o diferencial do CarneUp.

% Requires: \usepackage{array}
\begin{table}[h]
    \centering
    \caption{Comparativo}
    \label{tab:placeholder_label}
    \begin{tabular}{|m{4cm}|c|c|c|c|}
        \hline
        \textbf{Funcionalidade} & \textbf{CarneUp} & \textbf{ConnectPlug} & \textbf{Aliar Sistemas} & \textbf{SOFTClass} \\ \hline
        Registro de Vendas & X & X & X & X \\ \hline
        Controle Básico de Estoque & X & X & X & X \\ \hline
        Captura de Valor/Peso por Código de Barras da Balança & X & X & X & X \\ \hline
        Rastreabilidade de Peça: Fornecedor e Valor de Compra & X & X & - & - \\ \hline
        Alertas de Validade por Peça/Lote & X & - & - & - \\ \hline
        Foco em Pequenos/Médios Açougues e Facilidade de Uso & X & - & - & X \\ \hline
    \end{tabular}
\end{table}



% Capitulo com exemplos de comandos inseridos de arquivo externo 
% ---
%\include{1 Introdução}
% ---

% ---
% \include{2 Revisao da literatura}
% ---

% ---
%\include{3 Gestão do Projeto }
% ---
\chapter{Gestão do Projeto}\label{cap_exemplos}
%---
% ACRESCENTADO PARAGRAFO INTRODUTORIO NO CAPITULO 2
A presente seção detalha o planejamento, a organização e o controle das atividades essenciais para a execução do projeto CarneUp. O sucesso na entrega do produto final, dadas as restrições de tempo e os marcos acadêmicos fixos, depende intrinsecamente de uma gestão eficiente. Assim, este capítulo estabelece a estrutura da equipe, define os papéis e responsabilidades dos membros, e descreve a metodologia ágil adotada, incluindo a estruturação do cronograma e os ciclos de desenvolvimento. Por fim, são apresentadas as ferramentas de versionamento e o acesso ao código-fonte.


% ---
\section{Organização da Equipe}
% ---

O projeto será desenvolvido por uma equipe de cinco membros, com papéis definidos para cobrir as áreas de desenvolvimento, administração de dados e gestão.

% ---
%\subsection{Responsabilidades / Papéis / Atividades}
% REMOVIDO O TITULO 2.1.1
% ---


% ---
  \begin{figure}[htb]
	\begin{center}
	    \includegraphics[scale=0.5]{Imagens/Responsabilidades_Papeis_Atividades.png}
        \caption{\label  RResponsabilidades / Papéis / Atividades}
	\end{center}
\end{figure}


% ---
\section{Metodologias de gestão e desenvolvimento}
% ---

%ALTERAÇÕES NA SECÇÃO 2 PEDIDAS PELO PROFESSOR, TEXTO ANTIGO ESTA COMENTADO
O projeto foi planejado para ter uma duração total de aproximadamente 2,5 meses, iniciando em 6 de Setembro de 2025 e com previsão de conclusão em 31 de Outubro de 2025. Essa duração é ditada pelos Marcos Fixos de Entrega.

O desenvolvimento do CarneUp adota a metodologia ágil Scrum, organizada em ciclos de trabalho curtos chamados Sprints. Dada a restrição temporal imposta pelos marcos de entrega acadêmica, o projeto foi dividido em 4 Sprints principais com duração variável (aproximadamente 10 a 14 dias úteis) para focar na entrega dos marcos fixos. O foco não é no número de Sprints, mas sim no cumprimento dos seguintes marcos de entrega que guiam o ritmo do projeto:

%O projeto utilizará a metodologia Scrum para o desenvolvimento e a gestão de projetos.

% ---
%---\subsection{Scrum / Sprints }
% ---
%O Scrum é uma metodologia ágil que permite à equipe se auto-organizar e fazer mudanças de forma rápida, em resposta a novos requisitos ou feedback do cliente. Ele se baseia em ciclos de trabalho chamados Sprints e em três papéis principais (Scrum Master, Product Owner e Time de Desenvolvimento) \cite{scrum20}, garantindo clareza e foco na entrega de valor contínuo. A escolha do Scrum se deve à sua capacidade de lidar com a evolução dos requisitos e garantir entregas incrementais do sistema CarneUp ao longo do tempo. 

%O desenvolvimento do CarneUp adota a metodologia ágil Scrum, organizada em ciclos de trabalho curtos chamados Sprints. Dada a restrição temporal imposta pelos marcos de entrega acadêmica, o projeto foi dividido em 4 Sprints principais com duração variável (aproximadamente 10 a 14 dias úteis) para focar na entrega dos marcos fixos, conforme a tabela detalhada na seção 3.2.1.1.
%O foco não é no número de Sprints, mas sim no cumprimento dos seguintes marcos de entrega que guiam o ritmo do projeto:


\begin{enumerate}
    \item Desenho da Aplicação (Entregue em 20/09/2025).
    \item Prova de Conceito (POC) (Apresentada em 10/10/2025).
    \item Produto Mínimo Viável (MVP).
    \item Entrega Final e Documentação (31/10/2025).
\end{enumerate}

% ---
\section{Repositório da aplicação}
% ---

% ---
%SUBSEÇÃO ELIMINADA
%\subsection{Definição do Repositório da Aplicação}
% ---

O projeto utilizará o GitHub como plataforma de versionamento de código e colaboração. O GitHub permite o controle detalhado das alterações de código (commits), facilitando o trabalho em equipe, a revisão de código (Pull Requests) e o rastreamento de problemas (Issues).


% ---
%SUBSEÇÃO ELIMINADA
%\subsection{Link do Repositório e Especificações para Acesso}
% ---
%Plataforma: GitHub

Link: https://github.com/wellingtonwos/ProjetoExtensaoI
Acesso: O repositório será privado durante o desenvolvimento e pode ser tornado público (ou manter-se privado com acesso via convite) após a Entrega Final. Para fins de avaliação, os colaboradores deverão ser convidados a ter acesso "Read" ou "Collaborator".

% ---
%SEÇÃO ELIMINADA, TRATADA JÁ NA 2.2
%\section{Duração / Cronograma}
% ---
%O projeto foi planejado para ter uma para ter uma duração total de aproximadamente 2,5 meses, iniciando em 6 de Setembro de 2025 e com previsão de conclusão em 31 de Outubro de 2025.
%Essa duração é ditada pelos Marcos Fixos de Entrega, sendo a fase de desenvolvimento intensa e totalmente alinhada ao período de:


% ---
  \begin{figure}[htb]
	\begin{center}
	    \includegraphics[scale=0.5]{Imagens/Duracao_Cronograma.png}
        \caption{\label  DDuração / Cronograma}
	\end{center}
\end{figure}
% ---
%\include{4 Desenvolvimento do projeto}
% ---
\chapter{Desenvolvimento Do Projeto}\label{cap_exemplos}


% ---
\section{Escopo do Projeto}
% ---

CarneUp abrange o desenvolvimento de uma aplicação web completa para a gestão de açougues de pequeno e médio porte, com foco nas operações de Vendas, Compras e Estoque.


% ---
\subsection{Regras de Negócio}
% ---
  \begin{figure}[htb]
	\begin{center}
	    \includegraphics[scale=0.5]{Imagens/Regras_de_Negocio.png}
        \caption{\label - Regras de Negócio}
	\end{center}
\end{figure}


% ---
\subsection{Requisitos Funcionais}
% ---

  \begin{figure}[htb]
	\begin{center}
	    \includegraphics[scale=0.5]{Imagens/Requisitos_Funcionais.png}
        \caption{\label - Requisitos Funcionais}
	\end{center}
\end{figure}




% ---
\subsection{Requisitos Não Funcionais}
% ---

.

  \begin{figure}[htb]
	\begin{center}
	    \includegraphics[scale=0.5]{Imagens/Requisitos_Nao_Funcionais.png}
        \caption{\label - Requisitos Não Funcionais}
	\end{center}
\end{figure}


% ---
\section{Histórias de Usuário}
% ---
Como a metodologia de desenvolvimento escolhida é o Scrum, as funcionalidades serão expressas através de Histórias de Usuário.

% ---
\subsection{Descrição das Histórias de Usuário}
% ---
\begin{itemize}
    \item US01: Registrar Venda com Código de Barras (RF01, RF02)
        \begin{itemize}
            \item Descrição: Como um operador de caixa, quero registrar uma venda lendo o código de barras da balança, para que o sistema preencha automaticamente o item, peso e valor, agilizando o atendimento.
            
            \item Critérios de Aceitação: O código de barras deve ser lido pelo leitor, o sistema deve exibir o item e valor unitário. O operador deve confirmar a venda e o estoque deve ser atualizado.
        \end{itemize}
\end{itemize}

\begin{itemize}
    \item US01: Registrar Venda com Código de Barras (RF01, RF02)
        \begin{itemize}
            \item Descrição: Como um operador de caixa, quero registrar uma venda lendo o código de barras da balança, para que o sistema preencha automaticamente o item, peso e valor, agilizando o atendimento.
            
            \item Critérios de Aceitação: O código de barras deve ser lido pelo leitor, o sistema deve exibir o item e valor unitário. O operador deve confirmar a venda e o estoque deve ser atualizado.
        \end{itemize}
\end{itemize}

\begin{itemize}
    \item US02: Inclusão de Estoque com Rastreabilidade (RF03, RF04)
        \begin{itemize}
            \item Descrição: Como um gerente de compras, quero registrar a entrada de novas peças de carne, incluindo o fornecedor, valor de custo e data de validade, para que eu possa ter controle de rastreabilidade e calcular a margem de lucro.
            
            \item Critérios de Aceitação: Deve haver um formulário com campos obrigatórios para Fornecedor, Data de Validade e Valor de Custo. O registro deve ser armazenado no banco de dados com todas as informações.
        \end{itemize}
\end{itemize}

\begin{itemize}
    \item US03: Alerta de Validade Próxima (RF06, RNF05)
        \begin{itemize}
            \item Descrição: Como um gerente, quero receber um alerta no dashboard e por e-mail quando um item estiver a 7 dias do vencimento, para que eu possa tomar medidas (promoção/descarte) antes da perda total do produto.
            
            \item Critérios de Aceitação: O alerta deve ser exibido no dashboard principal com destaque. Um e-mail deve ser enviado para o e-mail do gerente.
        \end{itemize}
\end{itemize}

\begin{itemize}
    \item US04: Visualização de Margem de Lucro (RF05)
        \begin{itemize}
            \item Descrição: Como um proprietário, quero visualizar a margem de lucro por peça de carne vendida, para identificar quais cortes são mais rentáveis.
            
            \item Critérios de Aceitação: Deve haver uma tela de relatório que liste as vendas e mostre a diferença (em R\$ e \% ) entre o valor de venda e o valor de custo.
        \end{itemize}
        
\end{itemize}

\begin{itemize}
    \item US05: Gerenciamento de Descarte (RF07)
        \begin{itemize}
            \item Descrição: Como um gerente de estoque, quero registrar o descarte de uma peça, indicando o motivo (vencimento, estrago), para que o estoque seja baixado e o motivo da perda seja registrado para análise.
            
            \item Critérios de Aceitação: A baixa de estoque deve ocorrer. O sistema deve exigir a seleção do motivo e a data do descarte.
        \end{itemize}
\end{itemize}

% ---
\section{Arquitetura}
% ---

O CarneUp utilizará uma arquitetura de microsserviços simples para Front-end e Back-end, implantada na nuvem da AWS. O Front-end será desenvolvido em React e se comunicará com o Back-end via API RESTful. O Back-end será construído utilizando Node.js e Python, aproveitando o Node.js para a lógica de negócio principal (serviço de vendas) e o Python para tarefas específicas como processamento de dados e relatórios (serviço de relatórios/estoque). O banco de dados PostgreSQL será o coração do sistema para garantir a integridade transacional dos dados de estoque e venda. A escolha da AWS garante escalabilidade e alta disponibilidade.

% ---
\subsection{Diagrama da Arquitetura}
% ---



% ---
\section{Tecnologias}
% ---

Foram escolhidas para garantir robustez, performance e facilidade de manutenção.

% ---
\subsection{Front-end}
% ---

\begin{itemize}
    \item React: Biblioteca JavaScript para construção de interfaces de usuário reativas e componentizadas. Permite a criação de um Front-end rápido e modular (RNF03).
    
    \item Bootstrap: Framework de Front-end para design responsivo e consistente. Acelera o desenvolvimento da interface.
\end{itemize}

% ---
\subsection{Back-end}
% ---

\begin{itemize}
    \item A linguagem principal do Back-end será Java, utilizando o framework Spring Boot. O Spring Boot é ideal para construir microsserviços e sistemas transacionais de alta segurança e performance, como o módulo de registro de vendas e controle de estoque.
\end{itemize}

% ---
\subsection{Banco de Dados}
% ---
\begin{itemize}
    \item PostgreSQL: Sistema de gerenciamento de banco de dados relacional robusto e confiável, fundamental para a integridade dos dados de vendas e estoque (RNF04).
\end{itemize}
% ---
\subsection{Infraestrutura}
% ---

\begin{itemize}
    \item AWS (Amazon Web Services): Plataforma de computação em nuvem que fornecerá os serviços de hospedagem e deployment.
\end{itemize}

% ---
\section{Logs}
% ---

Será utilizada a biblioteca de logging nativa do Spring Boot (como SLF4J/Logback) para registrar logs de erro, transação e alertas de validade (RF05).



% ---
\section{Segurança, Privacidade e Legislação}
% ---


% ---
\subsection{Critérios de Segurança e Privacidade}
% ---
\begin{itemize}
    \item Criptografia de Senhas: Utilização de bcrypt para armazenar as senhas dos usuários (administradores/caixas) no banco de dados (RNF04).
\end{itemize}
% ---
\subsubsection{Criptografia em Trânsito: SSL/HTTPS}
% ---
O protocolo HTTPS (HyperText Transfer Protocol Secure) é essencial para o projeto CarneUp, pois garante a segurança das comunicações entre o navegador do usuário (cliente) e o servidor de aplicação (Back-end).

O HTTPS é, fundamentalmente, o protocolo HTTP combinado com a camada de segurança SSL/TLS (Secure Sockets Layer/Transport Layer Security). Essa combinação cria um link criptografado que protege os dados transmitidos contra interceptação e adulteração, garantindo a confidencialidade e a integridade das informações.
\begin{alineas}
    \item Critérios de Segurança e Conformidade
            A implementação do SSL/TLS no CarneUp atende a critérios cruciais para um sistema de gestão:
    \begin{itemize}
        \item Proteção de Credenciais: Durante o login e autenticação, as senhas (mesmo que criptografadas no banco de dados) são protegidas durante o trânsito da máquina do cliente até o servidor, impedindo ataques de sniffing ou Man-in-the-Middle.
        \item Integridade dos Dados de Venda: Garante que os dados financeiros críticos — como o valor final de uma venda, o valor de custo de um produto ou a atualização de estoque — não possam ser modificados por terceiros maliciosos durante a transmissão.
        \item Confiança do Usuário: O ícone de cadeado na barra de endereço (\url{HTTPS}) reforça a confiabilidade e o compromisso da aplicação com a segurança dos dados.
    \end{itemize}

    \item Implantação na Infraestrutura AWS
        Na arquitetura do CarneUp, a proteção HTTPS será implementada utilizando os serviços nativos da Amazon Web Services (AWS), garantindo alta disponibilidade e gerenciamento facilitado:

        \begin{itemize}
            \item Certificado SSL/TLS: Será utilizado um certificado SSL/TLS (emitido via AWS Certificate Manager - ACM, ou adquirido) para autenticar a identidade do domínio do CarneUp.
            \item Terminação SSL: A terminação (ou offloading) SSL será configurada em um Load Balancer (como o Application Load Balancer - ALB) ou no serviço de distribuição de conteúdo (AWS CloudFront, se utilizado), que são pontos de entrada para o tráfego. Isso permite que a carga de criptografia/descriptografia seja removida dos servidores de Back-end (Node.js/Python), melhorando o desempenho da aplicação (RNF01).
            \item Comunicação Interna: Embora a criptografia seja mais crítica na comunicação externa (cliente-servidor), a comunicação interna entre os serviços (Back-end e PostgreSQL/AWS RDS) também pode ser configurada para usar conexões criptografadas, assegurando a segurança end-to-end em toda a infraestrutura.
        \end{itemize}

        Em suma, a obrigatoriedade do protocolo HTTPS é uma medida fundamental que blinda a aplicação contra as vulnerabilidades mais comuns da web, protegendo informações sensíveis em conformidade com as melhores práticas de segurança da informação (RNF04).
    
\end{alineas}
    

% ---
\subsection{Observância à Legislação}
% ---
Embora o sistema trate principalmente de dados do negócio (estoque, vendas), o CarneUp está em observância com a Lei Geral de Proteção de Dados (LGPD) no tratamento dos dados pessoais de seus usuários (proprietários, gerentes, caixas).
\begin{itemize}
    \item Princípio da Finalidade: Coleta apenas dos dados estritamente necessários para o funcionamento do sistema (nome, e-mail, senha criptografada).
    \item Logs e Rastreabilidade: Manutenção de logs para auditoria de acesso e alterações, conforme exigido por regulamentos de segurança.
    \item Consentimento: O registro de usuário administrador implica na aceitação da política de privacidade e uso do sistema.
    \item
    
\end{itemize}
% ---
\section{Modelo de Banco de Dados}
% ---
% ---
\subsection{Modelo Entidade Relacionamento (MER / DER)}
% ---
O modelo será centrado nas entidades de Estoque, Vendas e Fornecedores.
\begin{itemize}
    \item Entidade Estoque (Peça): Armazena dados do produto em estoque (código de barras, peso, valor de custo, fornecedor, validade).
    \item Entidade Venda: Armazena o registro de cada transação (peça vendida, valor final, data/hora, operador).
    \item Entidade Fornecedor: Armazena dados de contato e histórico do fornecedor.
    
\end{itemize}


% ---
\subsection{Diagrama Entidade Relacionamento (DER)}
% ---

  \begin{figure}[htb]
	\begin{center}
	    \includegraphics[scale=0.5]{Imagens/DER.jpg}
        \caption{\label - Diagrama Entidade Relacionamento (DER)}
	\end{center}
\end{figure}



% ---
\section{Duração / Cronograma}
% ---
O projeto terá a duração total de 10 meses (Julho de 2025 a Maio de 2026), conforme a distribuição em Sprints e objetivos da seção 3.2.1.1.

% ---
\subsection{Análise da Duração do Projeto}
% ---
A duração foi definida priorizando a conclusão do MVP (Mínimo Produto Viável - Sprints 4-6) dentro do período inicial para obter feedback rápido, e dedicando a maior parte do tempo à implementação de funcionalidades críticas de negócio (rastreabilidade, alertas de validade) e à garantia de qualidade e manutenibilidade (Testes, Logs, Documentação Final). O cronograma é gerido de forma flexível através do Scrum.



% ---
\section{Dicionário de Dados}

  \begin{figure}[]
	\begin{center}
	    \includegraphics[scale=0.8]{Imagens/Dicionario_de_Dados.png}
        \caption{\label - Dicionário de Dados}
	\end{center}
\end{figure}


% ---
% ---
%\include{5 Viabilidade financeira}
% ---
\chapter{Viabilidade Financeira}\label{cap_exemplos}

A análise de viabilidade financeira simula a sustentabilidade do projeto em um período de 2,5 meses, considerando custos de mão de obra (MO) e infraestrutura.

% ---
\section{Custos}
% ---
A estimativa de custos de Mão de Obra (MO) é baseada em uma média de taxa horária (R\$ 50,00) para 5 desenvolvedores, limitados a 80 horas/mês cada, durante a nova duração total de 2,5 meses.



% ---
\subsection{Detalhamento dos Custos de Desenvolvimento (Mão de Obra)}
% ---

Foi calculado com base em 5 membros da equipe, trabalhando uma média de 80 horas/mês, durante os 2,5 meses.
    \begin{itemize}
        
    \item Esforço Total: 1.000 horas (5 membros x 80 horas/mês x 2,5 meses).

    \item  Custo Horário Médio: R\$ 50,00.

    \item Custo Total de MO: R\$ 50.000,00.

    \end{itemize}

% ---
\subsection{Custos de Infraestrutura}
% ---
Os custos operacionais de infraestrutura (manutenção e hospedagem) são projetados a partir da utilização de serviços cloud (AWS), que oferecem um modelo de pagamento por uso, minimizando o investimento inicial:
\begin{itemize}
    
\item Custo Mensal Estimado (AWS + Domínio): R\$ 400,00.

\item Custo Acumulado (2,5 meses de Desenvolvimento): R\$ 1.000,00.

\end{itemize}
Investimento Inicial Necessário
O investimento inicial total para o desenvolvimento e entrega do Projeto Final é de R\$ 51.000,00.



% ---
\section{Receitas}
% ---
CarneUp é baseado em Software as a Service (SaaS), focado em receita recorrente e escalabilidade\cite{aws_saas}.
\begin{itemize}
    \item Modelo: Assinatura mensal por açougue.
    \item Preço por Assinatura: R\$ 250,00 mensais por unidade (açougue).
\end{itemize}
A receita é escalável e diretamente ligada à aquisição de novos clientes após a Entrega Final (Mês 3 em diante).




% ---
\section{Cenário Realista}
% ---
Projeta uma taxa de adoção prudente, levando em consideração os desafios de implementação de novos sistemas em pequenos e médios varejistas.

\begin{itemize}
    \item Adoção: 5 novos clientes no Mês 3 (Novembro/2025), com crescimento de 5 clientes por mês nos períodos subsequentes.
    \item Ponto de Equilíbrio (Break-even Point): O custo inicial de R\$ 51.000,00 é recuperado no Mês 7 (Março/2026). Este é um prazo excelente, dado o alto investimento inicial em MO.
\end{itemize}


% ---
\section{Cenário Otimista}
% ---

Considera uma alta aceitação do produto no mercado, impulsionada pela facilidade de uso e pelo foco no problema de rastreabilidade do açougue.

\begin{itemize}
    \item Adoção: 10 novos clientes no Mês 3, com crescimento de 10 clientes por mês nos períodos subsequentes.
    \item Ponto de Equilíbrio (Break-even Point): Atingido no Mês 5 (Janeiro/2026).
\end{itemize}
% ---
\section{Cenário Pessimista}
% ---
Pressupõe uma adoção lenta, devido à inércia do mercado ou dificuldades na validação da proposta de valor.
\begin{itemize}
    \item Adoção: 2 novos clientes no Mês 3, com crescimento de apenas 2 clientes por mês nos períodos subsequentes.
    \item Ponto de Equilíbrio (Break-even Point): Atingido no Mês 12 (Outubro/2026).
\end{itemize}
% ---
%\include{6 Considerações Finais}
% ---
\chapter{Considerações Finais}\label{cap_exemplos}

O desenvolvimento do CarneUp foi concluído, resultando em uma aplicação que resolve o problema central da gestão de açougues de pequeno e médio porte: a rastreabilidade e o controle de validade do estoque de carnes. O projeto demonstrou a viabilidade técnica da integração de tecnologias modernas para atender a um nicho de mercado específico.

% ---
\section{Dificuldades, Escolhas e Descartes}
% ---

As dificuldades enfrentadas pelo projeto não foram de ordem de integração de hardware, mas sim desafios de gestão e complexidade de dados:

\begin{itemize}
    \item Restrição Temporal e Marcos Fixos: A principal dificuldade foi a condensação do cronograma para apenas 2,5 meses, imposta pelos Marcos Fixos de Entrega Acadêmica (20/09, 10/10 e 31/10/2025). Essa restrição exigiu um controle rigoroso do escopo.
    \item Modelagem de Diagramas e Banco de Dados (MER/DER): Encontramos problemas na fase de modelagem, pois a rastreabilidade completa da carne exige um controle de estoque complexo. Tivemos que refinar repetidamente os diagramas (MER/DER) para garantir que o sistema rastreasse as informações essenciais.
\end{itemize}

Escolhas Cruciais

As escolhas tecnológicas e metodológicas foram determinantes para o sucesso do projeto no prazo estabelecido:
\begin{itemize}
    \item Metodologia Scrum: A adoção do Scrum permitiu a entrega incremental em Sprints curtas, garantindo que os marcos fixos fossem atingidos (Desenho, POC e MVP) através de priorização ágil.
   
    \item PostgreSQL: O uso do PostgreSQL foi essencial para a segurança dos dados, oferecendo a robustez e a integridade transacional exigidas para dados financeiros e de estoque.
\end{itemize}

Funcionalidades Descartadas

Para garantir a entrega do MVP dentro do prazo rigoroso e focar na funcionalidade principal do açougue, algumas funcionalidades foram deliberadamente adiadas para a Fase 2:
\begin{itemize}
    \item Módulo de Contabilidade Completa: O foco foi mantido no core business (Venda, Estoque e Custo da Peça). Módulos avançados de geração de balancetes ou integração contábil foram adiados.
    \item Múltiplas Filiais: O sistema foi dimensionado para um único açougue para simplificar a lógica de estoque e as permissões, garantindo que o tempo fosse focado na rastreabilidade por lote.
\end{itemize}



% ----------------------------------------------------------
% Finaliza a parte no bookmark do PDF
% para que se inicie o bookmark na raiz
% e adiciona espaço de parte no Sumário
% ----------------------------------------------------------


% ----------------------------------------------------------
% ELEMENTOS PÓS-TEXTUAIS
% ----------------------------------------------------------
\postextual
% ----------------------------------------------------------

% ----------------------------------------------------------
% Referências bibliográficas
% ----------------------------------------------------------
\bibliography{Referencias}



\end{document}
