
%\documentclass[
	% -- opções da classe memoir --
%	12pt,				% tamanho da fonte
%	openright,			% capítulos começam em pág ímpar (insere página vazia caso preciso)
%	twoside,			% para impressão em verso e anverso. Oposto a oneside
%	a4paper,			% tamanho do papel. 
	% -- opções do pacote babel --
%	english,			% idioma adicional para hifenização
%	french,				% idioma adicional para hifenização
%	spanish,			% idioma adicional para hifenização
%	brazil				% o último idioma é o principal do documento
%	]{abntex2}
% ---	

\documentclass[
    % -- opções da classe memoir --
    12pt,               % tamanho da fonte
    openany,            % capítulos começam em qualquer pág
    oneside,            % para impressão só frente (digital)
    a4paper,            % tamanho do papel. 
    % -- opções do pacote babel --
    english,            % idioma adicional para hifenização
    french,             % idioma adicional para hifenização
    spanish,            % idioma adicional para hifenização
    brazil              % o último idioma é o principal do documento
    ]{abntex2}

% ---
% Pacotes básicos 
% ---
\usepackage{lmodern}			% Usa a fonte Latin Modern			
\usepackage[T1]{fontenc}		% Selecao de codigos de fonte.
\usepackage[utf8]{inputenc}		% Codificacao do documento (conversão automática dos acentos)
\usepackage{lastpage}			% Usado pela Ficha catalográfica
\usepackage{indentfirst}		% Indenta o primeiro parágrafo de cada seção.
\usepackage{xcolor}             % Controle das cores (mais moderno)
\usepackage{graphicx}			% Inclusão de gráficos
\usepackage{microtype} 			% para melhorias de justificação
\usepackage{longtable} 			% Adicionado para tabelas longas
\usepackage{tabularx} 			% Adicionado para tabelas que seajustam
\usepackage{xurl} 				% Adicionado para quebrar URLs
% ---
		
% ---
% Pacotes de citações
% ---
\usepackage[brazilian,hyperpageref]{backref}	 % Paginas com as citações na bibl
\usepackage[alf]{abntex2cite}	% Citações padrão ABNT
\usepackage{ragged2e}

% --- 
% CONFIGURAÇÕES DE PACOTES
% --- 

% ---
% Configurações do pacote backref
% Usado sem a opção hyperpageref de backref
\renewcommand{\backrefpagesname}{Citado na(s) página(s):~}
% Texto padrão antes do número das páginas
\renewcommand{\backref}{}
% Define os textos da citação
\renewcommand*{\backrefalt}[4]{
	\ifcase #1 %
		Nenhuma citação no texto.%
	\or
		Citado na página #2.%
	\else
		Citado #1 vezes nas páginas #2.%
	\fi}%
	
% Pacotes para tabelas (colunas de largura fixa e alinhamento)
\usepackage{array}
\usepackage{booktabs}
\usepackage{ragged2e} % Para alinhar texto justificado/à esquerda em colunas p{}
\usepackage{longtable} % Essencial para tabelas longas
\usepackage{hyperref}  % Para links (sempre o último a ser carregado)
	
	
% ---

% ---
% Informações de dados para CAPA e FOLHA DE ROSTO
% ---
\titulo{CarneUp: Sistema de Gestão de Açougue de Pequeno e Médio Porte}


\autor{BRENO DIAS OLIVEIRA SP3015645\\FELIPE DALBOSCO 
PALUDO SP3123766\\GUSTAVO GOUVEA ANDRADE SP3076725\\PEDRO AUGUSTO SILVA SANTOS SP3098559\\RAQUEL CORREIA DA SILVA 
SP3098567\\WELLINGTON OLIVEIRA DE SOUSA SP3161307}

\local{São Paulo - SP - Brasil}
\data{2025}
\orientador{Marcelo Tavares de Santana}

\instituicao{%
   IFSP- Instituto Federal de Educação, Ciência e Tecnologia
 Câmpus São Paulo
  \par
   Tecnologia em Análise e Desenvolvimento de Sistemas}

% O preambulo deve conter o tipo do trabalho, o objetivo, 
% o nome da instituição e a área de concentração 
\preambulo{Trabalho de Projeto Integrado I apresentado ao curso de Tecnologia em Análise e Desenvolvimento de Sistemas do Instituto Federal de Educação, Ciência e Tecnologia de São Paulo, como parte dos requisitos para a conclusão da disciplina.}
% ---


% ---
% Configurações de aparência do PDF final
% ---

% alterando o aspecto da cor azul
\definecolor{blue}{RGB}{41,5,195}

% informações do PDF
\makeatletter
\hypersetup{
     	%pagebackref=true,
		pdftitle={\@title}, 
		pdfauthor={\@author},
    	pdfsubject={\imprimirpreambulo},
	    pdfcreator={LaTeX with abnTeX2},
		pdfkeywords={abnt}{latex}{abntex}{abntex2}{trabalho acadêmico}, 
		colorlinks=true,       		% false: boxed links; true: colored links
    	linkcolor=blue,          	% color of internal links
    	citecolor=blue,        		% color of links to bibliography
    	filecolor=magenta,      		% color of file links
		urlcolor=blue,
		bookmarksdepth=4
}
\makeatother
% --- 

% --- 
% Espaçamentos entre linhas e parágrafos 
% --- 

% O tamanho do parágrafo é dado por:
\setlength{\parindent}{1.3cm}

% Controle do espaçamento entre um parágrafo e outro:
\setlength{\parskip}{0.2cm}  % tente também \onelineskip

% ---
% compila o indice
% ---
\makeindex
% ---

% ----
% Início do documento
% ----
\begin{document}

% Retira espaço extra obsoleto entre as frases.
\frenchspacing 

% ----------------------------------------------------------
% ELEMENTOS PRÉ-TEXTUAIS
% ----------------------------------------------------------
% \pretextual

% ---
% Capa
% ---
\imprimircapa
% ---

% ---
% Folha de rosto
% (o * indica que haverá a ficha bibliográfica)
% ---
\imprimirfolhaderosto*
% ---

% ---
% RESUMOS
% ---

% resumo em português
% \setlength{\absparsep}{18pt} % <<< CORREÇÃO: Comando comentado, pois é desnecessário com parágrafo único
\begin{resumo}
Este trabalho propõe o desenvolvimento do CarneUp, um sistema de gerenciamento de açougues focado em digitalizar e otimizar as operações de venda, compra e controle de estoque de carnes[cite: 13]. A ausência de controle digital em muitas boutiques e açougues de pequeno e médio porte, onde as vendas e o gerenciamento de estoque ainda são feitos de forma analógica em cadernos, gera ineficiência e perda de informações cruciais[cite: 14]. O CarneUp visa solucionar este problema, oferecendo uma plataforma robusta e intuitiva para o gerenciamento completo do negócio, desde o registro de vendas com captura automática de peso/valor via código de barras, até o controle de validade e a rastreabilidade por fornecedor[cite: 15]. O sistema é projetado para açougues de pequeno e médio porte, fornecendo ferramentas essenciais para aumentar a precisão do estoque, melhorar o controle financeiro por peça e reduzir o desperdício, garantindo assim a eficiência e a competitividade do negócio[cite: 16].

\textbf{Palavras-chaves}: Gestão de açougue, Controle de estoque de carnes, Sistema de vendas, Software de açougue[cite: 17].
\end{resumo}

% resumo em inglês
\begin{resumo}[Abstract]
 \begin{otherlanguage*}{english}
   This work proposes the development of CarneUp, a butcher shop management system focused on digitizing and optimizing sales, purchasing, and meat stock control operations[cite: 19]. The lack of digital control in many small and medium-sized butcher shops, where sales and inventory management are still carried out analogically in notebooks, leads to inefficiency and loss of crucial information[cite: 20]. CarneUp aims to solve this problem by offering a robust and intuitive platform for complete business management, from sales registration with automatic weight/value capture via barcode, to validity control and traceability by supplier[cite: 21]. The system is designed for small and medium-sized butcher shops, providing essential tools to increase inventory accuracy, improve financial control per piece, and reduce waste, thereby ensuring business efficiency and competitiveness[cite: 22].
 
   \vspace{\onelineskip}
 
   \noindent 
   \textbf{Key-words}: Butcher shop management, Meat inventory control, Sales system, Butcher shop software[cite: 23].
 \end{otherlanguage*}
\end{resumo}

% ---
% inserir lista de ilustrações
% ---
\pdfbookmark[0]{\listfigurename}{lof}
\listoffigures*
\cleardoublepage
% ---

% ---
% inserir lista de tabelas
% ---
\pdfbookmark[0]{\listtablename}{lot}
\listoftables*
\cleardoublepage
% ---

% ---
% inserir lista de abreviaturas e siglas
% ---
\begin{siglas}
  \item[API] Application Programming Interface
  \item[AWS] Amazon Web Services
  \item[CRUD] Create, Read, Update, Delete
  \item[FDD] Feature Driven Development
  \item[MER] Modelo Entidade-Relacionamento
  \item[MVP] Produto Mínimo Viável
  \item[PDV] Ponto de Venda
  \item[POC] Prova de Conceito
  \item[RF] Requisito Funcional
  \item[RNF] Requisito Não Funcional
  \item[SaaS] Software as a Service
  \item[Scrum] (Metodologia Ágil)
  \item[SSL/TLS] Secure Sockets Layer/Transport Layer Security
\end{siglas}
% ---

% ---
% inserir o sumario
% ---
\pdfbookmark[0]{\contentsname}{toc}
\tableofcontents*
\cleardoublepage
% ---

% ----------------------------------------------------------
% ELEMENTOS TEXTUAIS
% ----------------------------------------------------------
\textual

% ----------------------------------------------------------
% CAPÍTULO 1: INTRODUÇÃO
% ----------------------------------------------------------
\chapter*[Introdução]{Introdução}

O principal objetivo do projeto \textit{CarneUp} é desenvolver um sistema de gerenciamento digital focado nas operações de açougues de pequeno e médio porte. Este sistema visa permitir a digitalização e a otimização da gestão de estoque e do processo de vendas no Ponto de Venda (PDV).

O problema central que motiva este projeto é a gestão predominantemente analógica e ineficiente encontrada em muitos desses estabelecimentos. Atualmente, operações críticas como o controle de vendas, o registro de compras e o acompanhamento de estoque são realizados manualmente em cadernos ou planilhas. Esta abordagem é não apenas propensa a erros, mas também impede uma análise precisa da margem de lucro por corte e um controle detalhado por fornecedor ou validade. Consequentemente, a falta de um sistema digital resulta em desperdício de estoque vencido e na perda de dados valiosos para a tomada de decisão estratégica.

Diante desse cenário, a relevância do \textit{CarneUp} reside na necessidade urgente de modernização deste setor. A solução proposta visa digitalizar e centralizar a gestão do açougue, oferecendo uma aplicação web robusta e com foco na facilidade de uso para o público-alvo. Ao automatizar o registro de vendas via código de barras e implementar um controle de validade e de custos por fornecedor, o sistema impacta diretamente a eficiência operacional, o controle financeiro e a redução de desperdício.

\chapter {Escopo e Objetivos}

O Escopo do projeto abrange a criação de um sistema web focado nas operações de retaguarda (back-office) e no ponto de venda (PDV) de açougues. O sistema tem como foco principal o controle de estoque e o controle financeiro detalhado de cada peça de carne.

\textbf{O projeto abrange:}
\begin{itemize}
    \item Gestão de Vendas (PDV) com integração de código de barras.
    \item Controle de Estoque (Entrada/Saída), incluindo fornecedor, valor de custo e validade.
    \item Emissão de alertas de validade.
    \item Geração de relatórios básicos de vendas.
    \item Gestão de usuários e controle de sessão (Autenticação). % <<< ESTE ITEM
    \item Aplicação de padrões de acessibilidade (WCAG) na interface web.
\end{itemize}

Os objetivos específicos do projeto incluem:

\begin{itemize}
    \item Gerenciar Vendas: Permitir o registro rápido e eficiente de vendas de carnes.
    \item Controle de Estoque: Fornecer uma visão em tempo real do estoque de carnes, incluindo o controle das peças vendidas, suas validades e fornecedores.
    \item Digitalização de Pesagem: Integrar-se com códigos de barras gerados pela balança para capturar automaticamente o valor e o peso da peça no momento da venda.
    \item Otimização de Compras: Registrar o valor pago nas carnes e seus fornecedores para análises de custo.
\end{itemize}

\section{Requisitos do Sistema}

Para alcançar os objetivos propostos, o \textit{CarneUp} atende a um conjunto de requisitos funcionais (RF) e não funcionais (RNF e RN). Os requisitos funcionais estão detalhados na Tabela 1, agrupados de acordo com os principais componentes de gestão definidos nos diagramas de arquitetura: \textbf{Gestão de Vendas}, \textbf{Gestão de Produtos (Estoque)}, \textbf{Gestão de Compras} e \textbf{Gestão de Autenticação}.

\begin{longtable}{|>{\Centering}m{1.5cm}|>{\RaggedRight}m{13cm}|}
    \caption{Requisitos Funcionais (RF)}
    \label{tab:requisitos_funcionais}\\
    \hline
    \textbf{Código} & \textbf{Descrição} \\ 
    \hline
    \endfirsthead

    \multicolumn{2}{c}%
    {{\bfseries \tablename\ \thetable{} -- continuação}} \\
    \hline
    \textbf{Código} & \textbf{Descrição} \\ 
    \hline
    \endhead
    \hline
    \multicolumn{2}{|r|}{{Continua...}} \\
    \hline
    \endfoot
    \hline
    \endlastfoot

    RF01 & O sistema deve permitir o registro de vendas através da leitura do código de barras gerado pela balança. \\ 
    \hline
    RF02 & O sistema deve calcular automaticamente o valor total da venda ao capturar o peso e o valor unitário do item lido via código de barras. \\ 
    \hline
    RF08 & O sistema deve gerar relatórios diários de vendas, detalhando a quantidade e o valor total vendido. \\ 
    \hline
    RF03 & O sistema deve permitir o registro de novas entradas de estoque (compras), incluindo valor de custo, fornecedor e validade. \\ 
    \hline
    RF05 & O sistema deve calcular a margem de lucro de cada peça de carne vendida (Valor de Venda - Valor de Custo). \\ 
    \hline
    RF06 & O sistema deve emitir um alerta visual e via e-mail para o administrador quando a validade de uma peça de estoque estiver a 7 dias ou menos do vencimento. \\ 
    \hline
    RF07 & O sistema deve permitir o descarte de peças de estoque (baixa) e registrar o motivo do descarte (ex: vencimento, perda de qualidade). \\ 
    \hline
    RF04 & O sistema deve permitir a busca e visualização de peças de estoque por fornecedor. \\ 
    \hline
    RF09 & O sistema deve gerenciar o cadastro completo de fornecedores e seus respectivos contatos. \\
    \hline
    RF10 & O sistema deve controlar o acesso através de autenticação de usuário (login/senha) e gerenciar o ciclo de vida da sessão. \\ 
    \hline
    RF11 & O sistema deve permitir o cadastro de diferentes perfis de usuários (ex: Operador de Caixa, Gerente) com níveis de permissão distintos. \\
    \hline
\end{longtable}

\subsection{Regras de Negócio}

As regras de negócio (RN) representam as restrições, políticas e condições operacionais que ditam o funcionamento interno do sistema e garantem sua aderência aos processos específicos de um açougue. Elas são cruciais para manter a integridade dos dados, controlar o fluxo de estoque e assegurar a precisão financeira.

A Tabela \ref{tab:requisitos_rn} detalha as regras obrigatórias do sistema \textit{CarneUp}. É importante notar que cada regra está diretamente ligada a um ou mais Requisitos Funcionais (RF) ou Não Funcionais (RNF) definidos anteriormente, estabelecendo as condições sob as quais as funcionalidades de venda, estoque e custo são executadas.

\begin{longtable}{|>{\Centering}m{1.5cm}|>{\RaggedRight}m{9cm}|>{\Centering}m{4cm}|}
\caption{Regras de Negócio (RN)}
\label{tab:requisitos_rn}\\ % <-- Rótulo corrigido para ser o único usado
 \hline
 \textbf{Código} & \textbf{Descrição} & \textbf{Requisito Relacionado} \\
 \hline
 \endfirsthead

 \multicolumn{3}{c}%
 {{\bfseries \tablename\ \thetable{} -- continuação}} \\
 \hline
 \textbf{Código} & \textbf{Descrição} & \textbf{Requisito Relacionado} \\
 \hline
 \endhead
 \hline
 \multicolumn{3}{|r|}{{Continua...}} \\
 \hline
 \endfoot
 \hline
 \endlastfoot

 RN01 & A venda só pode ser registrada se o código de barras lido for reconhecido e o item tiver estoque disponível. & RF01, RF02 \\
 \hline
 RN02 & O valor de custo por peça deve ser registrado no momento da entrada no estoque para cálculo de margem de lucro. & RF03, RF05 \\
 \hline
 RN03 & Peças com validade expirada devem ser automaticamente bloqueadas para venda e movidas para a lista de descarte. & RF07, RNF01 \\
 \hline
 RN04 & O sistema deve manter o histórico de fornecedores para cada peça de carne no estoque para fins de análise de custo e consulta. & RF04 \\
 \hline
 RN05 & A peça de carne vendida deve ter seu estoque subtraído em tempo real no momento do registro da venda. & RF02, RNF01 \\
 \hline
\end{longtable}

Os Requisitos Não Funcionais (RNF) especificam os critérios que podem ser usados para julgar a operação de um sistema, em vez de comportamentos específicos. Diferentemente dos Requisitos Funcionais (RF), que definem o que o sistema deve fazer, os RNFs definem quão bem ele deve realizar suas funções, estabelecendo atributos de qualidade, usabilidade, desempenho e segurança.

Estes atributos são cruciais para o sistema \textit{CarneUp}, pois garantem que ele seja rápido o suficiente para um Ponto de Venda (PDV), escalável para acompanhar o crescimento do açougue e seguro para proteger dados financeiros. A Tabela \ref{tab:requisitos_rnf} detalha estes requisitos, organizados por código, seu módulo de impacto e a descrição que quantifica ou qualifica a expectativa de qualidade do software.

\begin{longtable}{|>{\Centering}m{1.5cm}|>{\Centering}m{3cm}|>{\RaggedRight}m{10cm}|}
 \caption{Requisitos Não Funcionais (RNF)}
 \label{tab:requisitos_rnf}\\
 \hline
\textbf{Código} & \textbf{Módulo} & \textbf{Descrição} \\
\hline
 \endfirsthead

 \multicolumn{3}{c}%
 {{\bfseries \tablename\ \thetable{} -- continuação}} \\
 \hline
\textbf{Código} & \textbf{Módulo} & \textbf{Descrição} \\
 \hline
 \endhead
 \hline
 \multicolumn{3}{|r|}{{Continua...}} \\
 \hline
 \endfoot
 \hline
 \endlastfoot

 RNF01 & Performance & O registro de venda e a baixa de estoque devem ser processados em no máximo 2 segundos para visar o fluxo rápido de caixa. \\
 \hline
 RNF02 & Escalabilidade & O sistema deve ser capaz de suportar picos de até 10 usuários simultâneos (caixas + gerentes) sem degradação perceptível de performance. \\
 \hline
 RNF03 & Usabilidade & A interface de registro de vendas deve ser intuitiva e otimizada para telas de toque e leitores de código de barras. \\
 \hline
 RNF04 & Segurança & Todos os dados sensíveis (informações financeiras, cadastros) devem ser armazenados de forma criptografada (ex: senhas via bcrypt, dados via SSL). \\
 \hline
 RNF05 & Manutenibilidade & O código deve seguir uma convenção de código (\textit{Code Convention}) e utilizar logs detalhados para facilitar a identificação e correção de erros. \\
 \hline
\end{longtable}

\section{Problema e Solução Proposta}

O problema central abordado pelo \textit{CarneUp} é a gestão predominantemente analógica e ineficiente encontrada em açougues de pequeno e médio porte. Nesses estabelecimentos, operações críticas como o controle de vendas, o registro de compras, o controle de validade e o acompanhamento de estoque são realizados manualmente, seja em cadernos ou planilhas.

Esta prática manual é altamente suscetível a erros e acarreta diversas ineficiências operacionais e financeiras: dificulta o controle de custos por fornecedor, impede uma análise precisa da margem de lucro por corte e leva a perdas diretas por estoque vencido. A ausência de um sistema digital, portanto, resulta na falta de dados confiáveis para a tomada de decisões estratégicas.

Para endereçar essa lacuna de gestão, a solução proposta é o \textit{CarneUp}. Trata-se de uma aplicação web desenhada para digitalizar e centralizar a gestão de estoque e vendas do açougue. Com foco na facilidade de uso para o público-alvo, o sistema prioriza a eficiência operacional no Ponto de Venda (PDV) e o controle detalhado das peças de carne, resolvendo diretamente os desafios de controle de custo, validade e registro de vendas.

As funcionalidades chave do \textit{CarneUp} são:

\begin{itemize}
    \item Registro de Vendas Otimizado: Utilização de códigos de barras da balança para preenchimento automático de valor e peso da peça vendida.
    \item Controle de Custo por Fornecedor: Registro de dados como fornecedor, valor de compra e validade de cada lote de carne.
    \item Controle de Estoque Inteligente: Emissão de alertas de validade próxima e visualização clara das peças em estoque.
\end{itemize}


\section{Justificativa}

A relevância do \textit{CarneUp} se justifica pela necessidade de modernização do setor de açougues de pequeno e médio porte. Ao fornecer uma solução digital acessível, o \textit{CarneUp} contribui diretamente para:

\begin{itemize}
    \item Controle Financeiro: O registro do valor pago por peça permite uma análise de custo precisa, otimizando a margem de lucro e identificando os cortes mais rentáveis.
    
    \item Redução de Desperdício: O controle de validade digitalizado ajuda a reduzir as perdas de estoque devido ao vencimento.
    
    \item Eficiência Operacional: A automação do registro de vendas via código de barras acelera o atendimento e minimiza a chance de erros no caixa.
\end{itemize}

A inexistência de um sistema digital que combine controle de estoque detalhado (fornecedor, valor de compra, validade) com uma interface de PDV eficiente para açougues menores torna o \textit{CarneUp} uma ferramenta estratégica para a competitividade desses negócios.

% ---
\section{Análise da Concorrência}
% ---
O CarneUp se posiciona contra sistemas de gestão (PDV) mais genéricos, que não possuem a granularidade necessária para o controle de carnes com base em peso variável, validade e fornecedor, sendo adaptados para o segmento de açougues.

\subsection{Concorrente 1: ConnectPlug}

O ConnectPlug é um sistema de PDV (Ponto de Venda) e gestão comercial mais abrangente, atendendo diversos segmentos, incluindo açougues. Ele oferece funcionalidades como frente de caixa, gestão de estoque e emissão de notas fiscais. \cite{connectplug}

\begin{itemize}
    \item Foco: Solução completa de gestão comercial.

    \item Diferencial (para o CarneUp): Pode ser adaptado para açougues, mas pode não ter o foco e a profundidade no **controle de custos por peça** (valor de compra, fornecedor, validade individualizada) que o CarneUp busca.
\end{itemize}

\subsection{Concorrente 2: Aliar Sistemas}

A Aliar Sistemas offers soluções específicas para o varejo de carnes, com foco em açougues e frigoríficos. Seu sistema inclui gestão de produção, **controle de lotes** e controle de estoque. \cite{aliar}

\begin{itemize}
    \item Foco: Soluções específicas para o setor de carnes.

    \item Por focar em frigoríficos e grandes açougues, pode ter um custo elevado e complexidade excessiva para açougues de pequeno porte, que é o alvo do CarneUp.
\end{itemize}

\subsection{Concorrente 3: SOFTClass}

A SOFTClass oferece software de gestão para diversos segmentos do varejo, incluindo açougues. Possui módulos de frente de caixa, estoque e financeiro.\cite{softclass}

\begin{itemize}
    \item Foco: Solução de gestão empresarial para varejo.

    \item Semelhante ao ConnectPlug, pode ser mais um sistema de PDV adaptado, carecendo da simplicidade e do foco no ciclo de vida da carne (compra/validade/fornecedor) que é o diferencial do CarneUp.
\end{itemize}


\subsection{Comparativo}
O quadro abaixo resume as principais funcionalidades e destaca o diferencial do CarneUp.

\begin{table}[h]
    \centering
    \caption{Comparativo}
    \label{tab:placeholder_label}
    \begin{tabularx}{\linewidth}{|>{\RaggedRight}p{4cm}|c|c|c|c|}
        \hline
        \textbf{Funcionalidade} & \textbf{CarneUp} & \textbf{ConnectPlug} & \textbf{Aliar Sistemas} & \textbf{SOFTClass} \\ \hline
        Registro de Vendas & X & X & X & X \\ \hline
        Controle Básico de Estoque & X & X & X & X \\ \hline
        Captura de Valor/Peso por Código de Barras da Balança & X & X & X & X \\ \hline
        \textbf{Controle Detalhado (Fornecedor e Custo por Peça)} & X & X & - & - \\ \hline
        Alertas de Validade por Peça/Lote & X & - & - & - \\ \hline
        Foco em Pequenos/Médios Açougues e Facilidade de Uso & X & - & - & X \\ \hline
    \end{tabularx}
\end{table}

% ----------------------------------------------------------
% CAPÍTULO 2: GESTÃO DO PROJETO (SEÇÃO 3 NO ORIGINAL)
% * <Felipe Paludo> 23:07:07 10 Nov 2025 UTC-0300:
% REFEITO
% ----------------------------------------------------------
\chapter{Gestão do Projeto}\label{cap_exemplos}
% ACRESCENTADO PARAGRAFO INTRODUTORIO NO CAPITULO 2
A presente seção detalha o planejamento, a organização e o controle das atividades essenciais para a execução do projeto, este capítulo estabelece a estrutura da equipe, define os papéis e responsabilidades dos membros, e descreve a metodologia ágil adotada, incluindo a estruturação do cronograma e os ciclos de desenvolvimento. Por fim, são apresentadas as ferramentas de versionamento e o acesso ao código-fonte.


% ---
\section{Organização da Equipe}
% ---

O projeto será desenvolvido por uma equipe de cinco membros, com papéis definidos para cobrir as áreas de desenvolvimento, administração de dados e gestão, conforme apresentado na Tabela 2.

% ---
%\subsection{Responsabilidades / Papéis / Atividades}
% ---

\begin{table}[h!]
    \centering
    \caption{Papéis, Integrantes e Responsabilidades Chave}
    \label{tab:papeis_equipe}
    
    \begin{tabular}{|>{\Centering}m{2.5cm}|>{\Centering}m{3.5cm}|>{\RaggedRight}m{7cm}|}
        \hline
        \textbf{Papel} & \textbf{Integrante} & \textbf{Responsabilidades Chave} \\ 
        \hline
        Product Owner & Raquel Correia da Silva & Definição e priorização do Product Backlog, contato com o cliente, validação das entregas. \\ 
        \hline
        Scrum Master & Felipe Dalbosco Paludo & Visar a aplicação correta da metodologia Scrum, remover impedimentos, facilitar as reuniões. \\ 
        \hline
        Desenvolvedor Full Stack & Gustavo Gouvea Andrade & Implementação das funcionalidades no front-end e back-end, integração com banco de dados. \\ 
        \hline
        Desenvolvedor Front-end & Breno Dias Oliveira & Desenvolvimento da interface do usuário, usabilidade e responsividade. \\ 
        \hline
        DBA / Desenvolvedor Back-end & Pedro Auguso Silva Santos & Modelagem e Administração do Banco de Dados, desenvolvimento de APIs e lógica de negócios. \\ 
        \hline
        DBA / Desenvolvedor Back-end & Wellington Oliveira de Sousa & Modelagem e Administração do Banco de Dados, desenvolvimento da documentação LateX. \\ 
        \hline
    \end{tabular}
\end{table}

% ---
\section{Metodologias de gestão e desenvolvimento}
% ---
O projeto foi planejado para ter uma duração total de aproximadamente 2,5 meses, iniciando em 6 de Setembro de 2025 e com previsão de conclusão em 31 de Outubro de 2025. Essa duração é ditada pelos Marcos Fixos de Entrega.

O desenvolvimento do CarneUp adota a metodologia ágil Scrum, organizada em ciclos de trabalho curtos chamados Sprints. Dada a restrição temporal imposta pelos marcos de entrega acadêmica, o projeto foi dividido em 4 Sprints principais com duração variável (aproximadamente 10 a 14 dias úteis) para focar na entrega dos marcos fixos. O foco não é no número de Sprints, mas sim no cumprimento dos marcos de entrega, apresentado na tabela 3, que guiam o ritmo do projeto.

\begin{table}[h!]
    \centering
    \caption{Marcos de Entrega do Projeto \textit{CarneUp}}
    \label{tab:marcos_entrega}
    \begin{tabular}{|>{\Centering}m{2.5cm}|>{\RaggedRight}m{11cm}|}
        \hline
        \textbf{Data} & \textbf{Marco de Entrega / Atividade} \\ 
        \hline
        \textbf{06/09/2025} & Início do Desenvolvimento Geral e da Documentação. \\ 
        \hline
        \textbf{20/09/2025} & Entrega do Desenho da Aplicação. \\ 
        \hline
        \textbf{10/10/2025} & Apresentação da Prova de Conceito (POC). \\ 
        \hline
        \textbf{31/10/2025} & Entrega do Projeto Final (MVP) e sua Documentação. \\ 
        \hline
    \end{tabular}
\end{table}

% ---
\section{Repositório da aplicação}
% ---

O projeto utilizará o GitHub como plataforma de versionamento de código e colaboração. O GitHub permite o controle detalhado das alterações de código (commits), facilitando o trabalho em equipe, a revisão de código (Pull Requests) e o rastreamento de problemas (Issues).

Link: https://github.com/wellingtonwos/ProjetoExtensaoI
Acesso: O repositório será privado durante o desenvolvimento e pode ser tornado público (ou manter-se privado com acesso via convite) após a Entrega Final. Para fins de avaliação, os colaboradores deverão ser convidados a ter acesso "Read" ou "Collaborator".


% ----------------------------------------------------------
% CAPÍTULO 3: DESENVOLVIMENTO (SEÇÃO 4 NO ORIGINAL)
% ----------------------------------------------------------
\chapter{Desenvolvimento Do Projeto}\label{cap:desenvolvimento}

O projeto \textit{CarneUp} abrange o desenvolvimento de uma aplicação web completa, projetada especificamente para digitalizar e otimizar a gestão de açougues de pequeno e médio porte, com foco nas operações de Vendas, Compras e Estoque. A arquitetura e as tecnologias foram selecionadas para garantir a rastreabilidade detalhada do produto e a alta performance no Ponto de Venda (PDV), endereçando diretamente a ineficiência da gestão analógica.

\section{Arquitetura e Modularidade do Sistema}

O \textit{CarneUp} foi concebido sob uma arquitetura de aplicação web em três camadas (Front-end, Back-end e Banco de Dados), visando clareza na separação de responsabilidades, manutenibilidade e escalabilidade (RNF02).

\subsection{Módulos de Negócio}

O Back-end é o core transacional do sistema e está logicamente dividido em módulos que se comunicam via API RESTful, cada um responsável por um pilar da gestão:

\begin{itemize}
    \item \textbf{Módulo de Gestão de Vendas (PDV):} Focado na eficiência do caixa (RNF01), este módulo gerencia o registro rápido de vendas. Sua principal característica é a integração com a leitura de código de barras da balança, permitindo a captura automática do item, peso e valor, e garantindo a baixa de estoque em tempo real (RF01, RF02).
    \item \textbf{Módulo de Estoque e Rastreabilidade:} É o elemento central do sistema. Ele permite o registro da entrada de estoque com informações cruciais para a análise de custo e rastreabilidade: valor de custo, fornecedor e data de validade (RF03). O módulo também é responsável pela emissão de alertas automáticos quando a validade das peças está próxima (RF06), sendo vital na estratégia de redução de desperdício.
    \item \textbf{Módulo de Gestão de Compras e Fornecedores:} Gerencia o cadastro completo dos fornecedores (RF09) e mantém o histórico de compra de cada lote de carne, permitindo o cálculo da margem de lucro por peça (RF05).
    \item \textbf{Módulo de Autenticação e Usuários:} Controla o acesso ao sistema via login/senha e implementa o gerenciamento de diferentes perfis de usuários (Operador de Caixa, Gerente), garantindo que apenas usuários autorizados tenham acesso às funcionalidades específicas (RF10, RF11).
\end{itemize}

\section{Escolhas Tecnológicas (Tech Stack)}

A seleção da pilha tecnológica foi baseada na busca por performance, segurança e robustez.

\begin{itemize}
    \item \textbf{Front-end (Apresentação):} Foi escolhido o \textbf{React} para a construção da interface. Esta biblioteca permite a criação de uma experiência de usuário reativa, modular e otimizada para telas de toque e leitores de código de barras, o que atende ao requisito de usabilidade no PDV (RNF03).
    \item \textbf{Back-end (Aplicação):} A lógica de negócios é implementada em \textbf{Java com o framework Spring Boot}. Esta combinação é ideal para sistemas transacionais de missão crítica, como controle de estoque e vendas, oferecendo alta segurança e performance na execução das regras de negócio (RNF01).
    \item \textbf{Banco de Dados (Dados):} O \textbf{PostgreSQL} foi definido como o Sistema Gerenciador de Banco de Dados (SGBD) principal. Sua reputação de robustez, integridade transacional e confiabilidade é fundamental para armazenar dados financeiros e de estoque de forma segura e duradoura (RNF04).
\end{itemize}

\section{Infraestrutura e Segurança na Nuvem}

O projeto será implantado na \textbf{Amazon Web Services (AWS)}, que oferece o modelo de Software as a Service (SaaS) e garante a infraestrutura necessária para suportar picos de uso e a expansão futura.

\begin{itemize}
    \item \textbf{Hospedagem e Escalabilidade:} A plataforma AWS garante que o sistema possa suportar picos de até 10 usuários simultâneos (RNF02) sem degradação de performance, por meio de serviços de computação elástica.
    \item \textbf{Segurança de Dados em Trânsito:} A comunicação entre o Front-end e o Back-end será totalmente blindada pelo protocolo \textbf{HTTPS}, com terminação SSL/TLS configurada no Load Balancer da AWS. Isso é mandatório para proteger credenciais de acesso (bcrypt) e dados financeiros contra interceptação (RNF04).
    \item \textbf{Manutenibilidade:} A utilização de serviços de logging nativos do Spring Boot (SLF4J/Logback) será implementada para rastrear transações e facilitar a identificação e correção de erros, conforme o requisito de manutenibilidade (RNF05).
\end{itemize}

% ---

\section{Histórias de Usuário}

Uma História de Usuário segue o formato padronizado "Como um [papel do usuário], eu quero [realizar uma tarefa], para que [obter um benefício]", transformando requisitos complexos em narrativas simples e orientadas ao valor. No contexto do \textit{CarneUp}, as Histórias mapeiam diretamente a jornada do produto no açougue, atacando os pontos de dor identificados:
\begin{itemize}
    \item \textbf{Eficiência no PDV:} Focando no operador de caixa para garantir a agilidade e precisão nas vendas (ex: US01).
    \item \textbf{Rastreabilidade e Prevenção de Perdas:} Atendendo ao gerente para controlar validade, fornecedor e descarte (ex: US02, US03, US05).
    \item \textbf{Análise Estratégica:} Apoiando o proprietário na visualização do desempenho financeiro através da margem de lucro (ex: US04).
\end{itemize}

As histórias a seguir detalham o escopo do Mínimo Produto Viável (MVP), garantindo que as entregas agreguem o máximo valor aos usuários em cada ciclo de desenvolvimento.

\subsection{Descrição das Histórias de Usuário}
% ---
\begin{itemize}
    \item US01: Registrar Venda com Código de Barras (RF01, RF02)
        \begin{itemize}
            \item Descrição: Como um operador de caixa, quero registrar uma venda lendo o código de barras da balança, para que o sistema preencha automaticamente o item, peso e valor, agilizando o atendimento.
            \item Critérios de Aceitação: O código de barras deve ser lido pelo leitor, o sistema deve exibir o item e valor unitário.
O operador deve confirmar a venda e o estoque deve ser atualizado.
        \end{itemize}
\end{itemize}

\begin{itemize}
    \item US02: Inclusão de Estoque com Rastreabilidade (RF03, RF04)
        \begin{itemize}
            \item Descrição: Como um gerente de compras, quero registrar a entrada de novas peças de carne, incluindo o fornecedor, valor de custo e data de validade, para que eu possa ter controle de rastreabilidade e calcular a margem de lucro.
            \item Critérios de Aceitação: Deve haver um formulário com campos obrigatórios para Fornecedor, Data de Validade e Valor de Custo.
O registro deve ser armazenado no banco de dados com todas as informações.
        \end{itemize}
\end{itemize}

\begin{itemize}
    \item US03: Alerta de Validade Próxima (RF06, RNF05)
        \begin{itemize}
            \item Descrição: Como um gerente, quero receber um alerta no dashboard e por e-mail quando um item estiver a 7 dias do vencimento, para que eu possa tomar medidas (promoção/descarte) antes da perda total do produto.
            \item Critérios de Aceitação: O alerta deve ser exibido no dashboard principal com destaque.
Um e-mail deve ser enviado para o e-mail do gerente.
        \end{itemize}
\end{itemize}

\begin{itemize}
    \item US04: Visualização de Margem de Lucro (RF05)
        \begin{itemize}
            \item Descrição: Como um proprietário, quero visualizar a margem de lucro por peça de carne vendida, para identificar quais cortes são mais rentáveis.
            \item Critérios de Aceitação: Deve haver uma tela de relatório que liste as vendas e mostre a diferença (em R\$ e \% ) entre o valor de venda e o valor de custo.
        \end{itemize}
        
\end{itemize}

\begin{itemize}
    \item US05: Gerenciamento de Descarte (RF07)
        \begin{itemize}
            \item Descrição: Como um gerente de estoque, quero registrar o descarte de uma peça, indicando o motivo (vencimento, estrago), para que o estoque seja baixado e o motivo da perda seja registrado para análise.
            \item Critérios de Aceitação: A baixa de estoque deve ocorrer.
O sistema deve exigir a seleção do motivo e a data do descarte.
        \end{itemize}
\end{itemize}

% ---
\section{Segurança, Privacidade e Legislação}

A segurança e a privacidade dos dados representam um pilar não funcional crítico para o sucesso e a longevidade do sistema \textit{CarneUp}. Em um ambiente transacional como o Ponto de Venda (PDV) de um açougue, a integridade dos dados financeiros (custos, vendas, margem de lucro) e a confidencialidade das credenciais de acesso (operadores e gerentes) são inegociáveis. Um sistema de gestão deve ser inerentemente resistente a vulnerabilidades, garantindo que a informação seja protegida em todas as suas fases: desde o trânsito da máquina do cliente até o servidor, até o seu armazenamento permanente no banco de dados.

O projeto aborda estas exigências por meio de uma estratégia de segurança em múltiplas camadas, focada em mitigar os riscos mais comuns da web, conforme detalhado no Requisito Não Funcional RNF04. A implementação técnica integra o uso de algoritmos robustos de \textit{hash} para proteção de senhas e a criptografia de comunicação \textit{end-to-end}, aproveitando os recursos nativos da infraestrutura de \textit{cloud} da AWS para assegurar o mais alto nível de proteção contra interceptação e adulteração. Tais medidas são fundamentais para construir a confiança do usuário e manter a conformidade com as melhores práticas de segurança da informação no ambiente de varejo.
\subsection{Critérios de Segurança e Privacidade}

A estratégia de segurança do \textit{CarneUp} está fundamentada em dois pilares essenciais: proteção dos dados em repouso (armazenados) e proteção dos dados em trânsito (durante a comunicação). O Requisito Não Funcional RNF04 exige o uso de técnicas de criptografia para todos os dados sensíveis, garantindo a integridade e a confidencialidade do sistema.

\subsubsection{Proteção de Dados em Repouso}

Para garantir a confidencialidade das credenciais de acesso, o sistema adotará a seguinte medida para dados armazenados:
\begin{itemize}
    \item \textbf{Criptografia de Senhas:} Utilização do algoritmo de hash \textbf{bcrypt} para armazenar as senhas dos usuários (administradores e caixas) no banco de dados. Esta técnica de \textit{hashing} salgado impede a recuperação da senha original mesmo em caso de comprometimento da base de dados (RNF04).
\end{itemize}

\subsubsection{Criptografia em Trânsito: SSL/TLS (HTTPS)}

O protocolo HTTPS (\textit{HyperText Transfer Protocol Secure}) é essencial para o projeto \textit{CarneUp}, pois garante a segurança das comunicações entre o navegador do usuário e o servidor de aplicação (\textit{Back-end}). O HTTPS é o protocolo HTTP combinado com a camada de segurança SSL/TLS (\textit{Secure Sockets Layer/Transport Layer Security}), criando um link criptografado que protege os dados transmitidos contra interceptação e adulteração.

\paragraph{Critérios de Segurança e Conformidade}
A implementação do SSL/TLS no \textit{CarneUp} atende a critérios cruciais para um sistema de gestão:
\begin{itemize}
    \item \textbf{Proteção de Credenciais:} As senhas são protegidas durante o trânsito da máquina do cliente até o servidor, impedindo ataques de \textit{sniffing} ou \textit{Man-in-the-Middle}.
    \item \textbf{Integridade dos Dados de Venda:} Garante que os dados financeiros críticos (valor final da venda, custo do produto ou atualização de estoque) não possam ser modificados por terceiros maliciosos durante a transmissão.
    \item \textbf{Confiança do Usuário:} O ícone de cadeado na barra de endereço (\url{HTTPS}) reforça a confiabilidade e o compromisso da aplicação com a segurança dos dados.
\end{itemize}

\paragraph{Implantação na Infraestrutura AWS}
Na arquitetura do \textit{CarneUp}, a proteção HTTPS será implementada utilizando os serviços nativos da Amazon Web Services (AWS), garantindo alta disponibilidade e gerenciamento facilitado:
\begin{itemize}
    \item \textbf{Certificado SSL/TLS:} Será utilizado um certificado SSL/TLS (emitido via AWS Certificate Manager - ACM, ou adquirido) para autenticar a identidade do domínio.
    \item \textbf{Terminação SSL (\textit{Offloading}):} A terminação SSL será configurada no Application Load Balancer (ALB), o ponto de entrada do tráfego. Isso permite que a carga de criptografia/descriptografia seja removida dos servidores de \textit{Back-end} (Spring Boot), melhorando o desempenho da aplicação (RNF01).
    \item \textbf{Comunicação Interna:} A comunicação interna entre os serviços (\textit{Back-end} e PostgreSQL/AWS RDS) será configurada para usar conexões criptografadas, assegurando a segurança \textit{end-to-end} em toda a infraestrutura.
\end{itemize}

Em suma, a obrigatoriedade do protocolo HTTPS é uma medida fundamental que blinda a aplicação contra as vulnerabilidades mais comuns da web, protegendo informações sensíveis em conformidade com as melhores práticas de segurança da informação (RNF04).
    

\subsection{Observância à Legislação}

Embora o sistema \textit{CarneUp} trate primariamente de dados do negócio (estoque, vendas e custos), ele processa dados pessoais de seus usuários internos (proprietários, gerentes e caixas), atuando como operador e controlador desses dados. Dessa forma, o desenvolvimento adere integralmente aos princípios e diretrizes da \textbf{Lei Geral de Proteção de Dados (LGPD) – Lei nº 13.709/2018}. A observância legal não se limita apenas à proteção, mas à construção de um sistema ético e transparente.

\paragraph{Princípios de Proteção e Transparência}
O projeto materializa a conformidade com a LGPD nos seguintes pontos de design:

\begin{itemize}
    \item \textbf{Princípio da Finalidade e Necessidade:} A coleta de dados é limitada estritamente ao necessário para a autenticação e gestão de permissões do sistema (nome, e-mail e credenciais de login). Nenhum dado sensível (como dados biométricos ou financeiros pessoais) é processado, aplicando o conceito de \textbf{minimização de dados}.
    
    
    \item \textbf{Direitos do Titular:} Dada a natureza da aplicação (sistema interno de gestão), a arquitetura é projetada para permitir que o administrador do açougue (controlador primário) possa, a qualquer momento, atender a solicitações dos titulares de dados, como a \textbf{exclusão de conta} de um ex-funcionário ou a \textbf{consulta} aos seus dados cadastrais.
\end{itemize}

\section{Modelo de Banco de Dados}

O modelo de banco de dados é a espinha dorsal do \textit{CarneUp}, sendo o responsável por garantir a integridade, a consistência e a rastreabilidade dos dados, requisitos críticos para a gestão de perecíveis. A escolha por um modelo relacional é mandatória para suportar as transações de alta frequência no Ponto de Venda (PDV) e permitir cálculos financeiros complexos, como a margem de lucro por peça. A tecnologia \textbf{PostgreSQL} (conforme definido na arquitetura) é ideal para este modelo, dada a sua robustez, consistência transacional e capacidade de gerenciar volumes de dados de inventário.

\subsection{Modelo Entidade Relacionamento (MER / DER)}

O Modelo Entidade Relacionamento (MER) do \textit{CarneUp} foi concebido para atender diretamente às Regras de Negócio e Histórias de Usuário, centrando-se nos fluxos de Estoque, Vendas e Compras. A estrutura relacional assegura que cada transação seja ligada de forma unívoca a uma peça de estoque, permitindo auditoria completa, cálculos precisos de custo e o gerenciamento eficaz da validade do produto. O modelo é composto pelas seguintes entidades principais, cujas responsabilidades espelham os requisitos do projeto:

\begin{itemize}
    \item \textbf{Entidade Estoque (Peça):} É a entidade central para a rastreabilidade. Armazena dados essenciais do produto em estoque, incluindo o \textit{código de barras} (chave para o PDV - US01), \textit{peso}, \textit{data de validade} (crítico para US03/RN03), \textit{valor de custo} (base para RF05/US04) e a chave estrangeira do Fornecedor.
    \item \textbf{Entidade Venda:} Representa a transação completa. Armazena os metadados da transação (valor total, data/hora e operador de caixa), sendo a entidade-mãe para o detalhamento dos itens vendidos.
    \item \textbf{Entidade ItemVenda (Nova Entidade de Relacionamento):} Esta entidade conecta a Venda à Peça de Estoque. É vital, pois é onde se registra o valor final de venda do item e se realiza a baixa de estoque em tempo real (RN05), facilitando o cálculo da margem de lucro de cada item na transação (US04).
    \item \textbf{Entidade Fornecedor:} Armazena os dados de contato e histórico de compras. É ligada à Entidade Estoque, permitindo a rastreabilidade da origem do produto (RF04/US02).
\end{itemize}

\subsection{Diagrama Entidade Relacionamento (DER)}

O Diagrama Entidade Relacionamento (DER) a seguir (Figura \ref{fig:der}) representa a estrutura lógica do banco de dados, ilustrando as entidades, atributos e os relacionamentos definidos no MER.

\begin{figure}[htb]
	\centering
	    \includegraphics[width=0.9\linewidth]{Imagens/DER.jpg}
         \caption{Diagrama Entidade Relacionamento (DER)}
	    \label{fig:der}
	\end{figure}
% ---
% <<< CORREÇÃO: Seção 4.8 e 4.8.1 removidas por serem contraditórias e redundantes.
% A duração correta (2,5 meses) já está definida no Capítulo 3.
% ---

% ---
%\section{Dicionário de Dados}

%  \begin{figure}[htb]
%	\centering
%	    \includegraphics[width=\linewidth]{Imagens/Dicionario_de_Dados.png} % <<< CORREÇÃO: scale substituído por width=\linewidth para corrigir overflow
%        \caption{Dicionário de Dados} % <<< CORREÇÃO: Sintaxe
%	    \label{fig:dicionario_dados} % <<< CORREÇÃO: Sintaxe
%	\end{figure}


% ---
% ----------------------------------------------------------
% CAPÍTULO 4: VIABILIDADE FINANCEIRA
% ----------------------------------------------------------
\chapter{Viabilidade Financeira}\label{cap:viabilidade}

O presente capítulo tem como objetivo delinear a viabilidade econômico-financeira do projeto \textsf{CarneUp}, elemento fundamental para garantir a sustentabilidade de longo prazo e a escalabilidade do sistema após a sua implantação. Esta análise transcende a perspectiva puramente técnica, focando na conversão do Produto Mínimo Viável (MVP) em um modelo de negócio rentável e competitivo no mercado de açougues de pequeno e médio porte.

Este estudo de caso se concentra na quantificação do Investimento Inicial requerido para o desenvolvimento, simulando a alocação de recursos durante os 2,5 meses de duração do projeto. A análise é estruturada nas seguintes etapas:

% ---
\section{Custos}\label{sec:custos}

A estimativa de custos representa o Investimento Inicial (Capital de Giro) necessário para a fase de desenvolvimento e entrega do Produto Mínimo Viável (MVP) do \textsf{CarneUp}. Estes custos foram categorizados em duas vertentes principais: Mão de Obra (MO) e Infraestrutura, conforme detalhado nas subseções a seguir.

\subsection{Detalhamento dos Custos de Desenvolvimento (Mão de Obra)}\label{sub:custos_mo}

O custo de Mão de Obra (MO) é a componente de maior peso no investimento inicial, sendo diretamente proporcional ao esforço total da equipe de desenvolvimento. O cálculo baseou-se na estimativa de horas de trabalho necessárias para o período de 2,5 meses, aplicando-se uma taxa horária média ponderada.

\begin{itemize}
    \item Esforço Total Estimado: 1.000 horas (5 desenvolvedores $\times$ 80 horas/mês $\times$ 2,5 meses).
    \item Custo Horário Médio (\textit{Loaded Rate}): R\$ 50,00.
    \item Custo Total de MO: R\$ 50.000,00.
\end{itemize}

\subsection{Custos de Infraestrutura e Operacionais}\label{sub:custos_infra}

Os custos de infraestrutura consideram a utilização de serviços em nuvem (AWS) e domínios essenciais para o ambiente de desenvolvimento e testes. A escolha pelo modelo \textit{pay-as-you-go} minimiza o investimento de capital (Capex) em hardware, concentrando-o em custos operacionais (Opex).

\begin{itemize}
    \item Custo Mensal Estimado (AWS + Domínio): R\$ 400,00.
    \item Custo Acumulado (2,5 meses de Desenvolvimento): R\$ 1.000,00.
\end{itemize}

\subsection{Investimento Inicial Consolidado}\label{sub:investimento_consolidado}

O investimento total necessário para a conclusão e entrega do MVP é consolidado na Tabela \ref{tab:custos_iniciais}.

% --- Tabela formatada em ABNT ---
\begin{table}[!htb]
    \centering
    \caption{Consolidação dos Custos Iniciais do Projeto}
    \label{tab:custos_iniciais}
    \begin{tabular}{|l|c|c|}
        \hline
        \textbf{Natureza do Custo} & \textbf{Valor (R\$)} & \textbf{Participação (\%)} \\
        \hline\hline
        Mão de Obra (Desenvolvimento) & 50.000,00 & 98,04\% \\
        \hline
        Infraestrutura (AWS, Domínio) & 1.000,00 & 1,96\% \\
        \hline\hline
        \textbf{INVESTIMENTO INICIAL TOTAL} & \textbf{51.000,00} & \textbf{100,00\%} \\
        \hline
    \end{tabular}
    \flushleft{\footnotesize Fonte: Elaborado pelo autor (2025).}
\end{table}

Conforme apresentado na Tabela \ref{tab:custos_iniciais}, o Investimento Inicial Total do projeto é de R\$ 51.000,00. O custo com Mão de Obra representa a parcela predominante deste valor, indicando a natureza intensiva em capital humano do desenvolvimento de software.

\subsection{Custos de Infraestrutura}\label{sub:custos_infra}

Os custos operacionais de infraestrutura (manutenção e hospedagem) são projetados a partir da utilização de serviços \textit{cloud} (AWS), que oferecem um modelo de pagamento por uso, minimizando o investimento inicial:

\begin{itemize}
    \item Custo Mensal Estimado (AWS + Domínio): R\$ 400,00.
    \item Custo Acumulado (2,5 meses de Desenvolvimento): R\$ 1.000,00.
\end{itemize}

O Investimento Inicial Total necessário para o desenvolvimento e entrega do Projeto Final é de R\$ 51.000,00.

\section{Receitas}\label{sec:receitas}

O modelo de monetização do \textsf{CarneUp} é baseado no conceito de \textit{Software as a Service} (SaaS), fundamentado na receita recorrente e na escalabilidade intrínseca à arquitetura multi-tenant \cite{aws_saas}. Esta estratégia está alinhada com a Proposta de Valor do sistema, que entrega rastreabilidade e gestão de estoque em troca de uma taxa de serviço mensal.

A Tabela \ref{tab:modelo_receita} detalha a estrutura de precificação adotada para o mercado-alvo de açougues de pequeno e médio porte.

\begin{table}[!htb]
    \centering
    \caption{Estrutura de Receita Recorrente (SaaS)}
    \label{tab:modelo_receita}
    \begin{tabular}{|l|l|}
        \hline
        \textbf{Componente} & \textbf{Detalhe} \\
        \hline\hline
        Modelo de Cobrança & Assinatura Mensal (Recorrente) \\
        \hline
        Unidade de Cobrança & Por Unidade (Açougue/Loja) \\
        \hline
        Preço por Assinatura & R\$ 250,00/mês \\
        \hline
    \end{tabular}
    \flushleft{\footnotesize Fonte: Elaborado pelo autor (2025).}
\end{table}

A natureza escalável do modelo SaaS implica que a Receita Operacional Mensal (ROM) é diretamente proporcional à aquisição e retenção de novos clientes. A geração de receita líquida inicia-se no Mês 3 (Entrega Final do MVP), sendo o fator de crescimento principal para a recuperação do investimento inicial.

\section{Análise de Cenários}\label{sec:cenarios}

Para validar a sustentabilidade financeira do projeto, foi realizada uma Análise de Sensibilidade, simulando três cenários distintos de adoção do produto pelo mercado-alvo. O objetivo primário é determinar o Ponto de Equilíbrio (\textit{Break-even Point}), que representa o momento em que a Receita Acumulada iguala o Investimento Inicial de R\$ 51.000,00, marcando o início da recuperação de capital.

\subsection{Cenário Realista}\label{sub:cenario_realista}

Este cenário projeta uma taxa de adoção prudente, alinhada com os desafios típicos de implementação de novos sistemas (curva de aprendizado e inércia) em pequenos e médios varejistas.

\begin{itemize}
    \item Adoção Base: 5 novos clientes no Mês 3 (Novembro/2025).
    \item Crescimento: Acréscimo de 5 novos clientes por mês nos períodos subsequentes (crescimento linear).
    \item Ponto de Equilíbrio (\textit{Break-even Point}): O investimento é recuperado no \textbf{Mês 7 (Março/2026)}.
\end{itemize}
O prazo de recuperação de capital de sete meses é considerado satisfatório, dado o alto percentual de investimento direcionado à Mão de Obra.

\subsection{Cenário Otimista}\label{sub:cenario_otimista}

O cenário otimista considera uma alta taxa de aceitação do mercado (\textit{Early Adopters}), impulsionada pela facilidade de uso do \textsf{CarneUp} e pelo foco assertivo na solução do problema de rastreabilidade (Proposta de Valor).

\begin{itemize}
    \item Adoção Base: 10 novos clientes no Mês 3 (Novembro/2025).
    \item Crescimento: Acréscimo de 10 novos clientes por mês nos períodos subsequentes.
    \item Ponto de Equilíbrio (\textit{Break-even Point}): Atingido no \textbf{Mês 5 (Janeiro/2026)}.
\end{itemize}

\subsection{Cenário Pessimista}\label{sub:cenario_pessimista}

O cenário pessimista pressupõe uma adoção lenta e cautelosa, decorrente de possíveis barreiras de entrada, resistência à mudança ou dificuldades inesperadas na validação da proposta de valor.

\begin{itemize}
    \item Adoção Base: 2 novos clientes no Mês 3 (Novembro/2025).
    \item Crescimento: Acréscimo de 2 novos clientes por mês nos períodos subsequentes.
    \item Ponto de Equilíbrio (\textit{Break-even Point}): Atingido no \textbf{Mês 12 (Outubro/2026)}.
\end{itemize}

\subsection{Consolidação da Análise de Sensibilidade}\label{sub:tabela_cenarios}

A Tabela \ref{tab:analise_sensibilidade} resume os parâmetros de crescimento e os respectivos pontos de equilíbrio, fornecendo uma visão consolidada do risco de mercado e da projeção de retorno.

% --- Tabela formatada em ABNT (Correção: Colunas de largura fixa para alinhamento) ---
\begin{table}[!htb]
    \centering
    \caption{Resumo da Análise de Sensibilidade e Ponto de Equilíbrio}
    \label{tab:analise_sensibilidade}
    % Colunas: 1 (Cenário) alinhada à esquerda, 3 colunas de dados com largura fixa e conteúdo centralizado.
    \begin{tabular}{p{2.5cm} >{\centering\arraybackslash}p{3.5cm} >{\centering\arraybackslash}p{3cm} >{\centering\arraybackslash}p{3cm}}
        \hline % Linha superior
        \textbf{Cenário} & \textbf{Adoção Mensal Base (Clientes)} & \textbf{Crescimento Mensal ($\Delta$)} & \textbf{Ponto de Equilíbrio (Mês)} \\
        \hline\hline % Linha dupla para separar cabeçalho
        Pessimista & 2 & +2 & 12 (Out/2026) \\
        Realista & 5 & +5 & 7 (Mar/2026) \\
        Otimista & 10 & +10 & 5 (Jan/2026) \\
        \hline % Linha inferior
    \end{tabular}
    \flushleft{\footnotesize Fonte: Elaborado pelo autor (2025).}
\end{table}

% ----------------------------------------------------------
% CAPÍTULO 5: CONSIDERAÇÕES FINAIS
% ----------------------------------------------------------
\chapter{Considerações Finais}\label{cap:conclusao}

O presente projeto alcançou seu objetivo geral, que era desenvolver o \textsf{CarneUp}, um Sistema de Gestão focado na rastreabilidade e no controle de perecibilidade de estoque para açougues de pequeno e médio porte.

Com a entrega do Produto Mínimo Viável (MVP), o projeto demonstrou a viabilidade técnica da integração de tecnologias modernas para atender ao problema de ineficiência na gestão analógica. A solução implementada propõe-se a atender às exigências do mercado e fornece um controle de validade e lote mais eficaz, validando a premissa de que a tecnologia pode mitigar riscos operacionais no core business do varejo de carnes.

\section{Conclusão dos Resultados e Análise Crítica}\label{sec:analise_critica}

Esta seção sintetiza os principais resultados alcançados, as decisões estratégicas tomadas e o alinhamento do projeto com sua proposta de valor, incluindo a validação financeira.

\subsection{Alinhamento com a Proposta de Valor e Viabilidade}\label{sub:alinhamento_viabilidade}

O \textsf{CarneUp} não apenas cumpriu seus requisitos técnicos, como também validou uma alta viabilidade econômica. Conforme detalhado no Capítulo \ref{cap:viabilidade}, a análise de sensibilidade projeta a recuperação do Investimento Inicial de R\$ 51.000,00 em um prazo excelente de sete meses no Cenário Realista.

Essa performance é um indicativo de que a solução é economicamente sustentável e possui um modelo de negócios (SaaS) com potencial de escalabilidade. A decisão de utilizar o PostgreSQL foi fundamental para oferecer a robustez e a integridade transacional exigidas para dados financeiros e de estoque, aspectos que buscam assegurar a confiança do usuário no sistema.

\subsection{Dificuldades e Lições Aprendidas}\label{sub:licoes_aprendidas}

O desenvolvimento foi marcado por desafios que exigiram adaptação metodológica e rigor na gestão de escopo.

\begin{itemize}
    \item Restrição Temporal e Escopo: A principal dificuldade foi a condensação do cronograma para apenas 2,5 meses, imposta pelos Marcos Fixos de Entrega Acadêmica (20/09, 10/10 e 31/10/2025). Essa restrição exigiu uma priorização agressiva e um controle rigoroso do escopo.
    \item Modelagem de Dados Complexa: A rastreabilidade por lote e validade demandou um esforço significativo na fase de modelagem, com refinamentos repetidos nos diagramas (MER/DER). Isso evidenciou a complexidade inerente à representação de processos de estoque perecível.
\end{itemize}

A adoção da Metodologia \textit{Scrum} permitiu a entrega incremental em Sprints curtas, um fator que foi determinante para a conclusão do MVP dentro do prazo.

\section{Limitações e Trabalhos Futuros}\label{sec:trabalhos_futuros}

As limitações impostas pelo prazo rigoroso resultaram em funcionalidades que foram deliberadamente postergadas, estabelecendo o escopo para a Fase 2 do projeto, que será objeto de Trabalhos Futuros.

\begin{itemize}
    \item Módulo de Contabilidade Completa: O foco foi mantido no core business (Venda, Estoque e Custo da Peça). Módulos avançados de geração de balancetes ou integração contábil foram adiados.
    \item Suporte a Múltiplas Filiais: O sistema foi dimensionado inicialmente para um único açougue. A expansão para uma arquitetura multi-filial será o foco principal para a otimização da escalabilidade do produto no futuro.
\end{itemize}

O desenvolvimento subsequente deve incluir a implementação dessas funcionalidades, a otimização contínua da interface do usuário (UI) baseada em *feedback* de clientes e a implementação de métricas de retenção para o modelo SaaS.

% ----------------------------------------------------------
% ELEMENTOS PÓS-TEXTUAIS
% ----------------------------------------------------------
\postextual
% ----------------------------------------------------------

% ----------------------------------------------------------
% Referências bibliográficas
% ----------------------------------------------------------
\bibliography{Referencias}
\end{document}
