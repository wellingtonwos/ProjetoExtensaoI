% ---
%\documentclass[
	% -- opções da classe memoir --
%	12pt,				% tamanho da fonte
%	openright,			% capítulos começam em pág ímpar (insere página vazia caso preciso)
%	twoside,			% para impressão em verso e anverso. Oposto a oneside
%	a4paper,			% tamanho do papel. 
	% -- opções do pacote babel --
%	english,			% idioma adicional para hifenização
%	french,				% idioma adicional para hifenização
%	spanish,			% idioma adicional para hifenização
%	brazil				% o último idioma é o principal do documento
%	]{abntex2}
% ---	

\documentclass[
    % -- opções da classe memoir --
    12pt,               % tamanho da fonte
    openany,            % capítulos começam em qualquer pág
    oneside,            % para impressão só frente (digital)
    a4paper,            % tamanho do papel. 
    % -- opções do pacote babel --
    english,            % idioma adicional para hifenização
    french,             % idioma adicional para hifenização
    spanish,            % idioma adicional para hifenização
    brazil              % o último idioma é o principal do documento
    ]{abntex2}

% ---
% Pacotes básicos 
% ---
\usepackage{lmodern}			% Usa a fonte Latin Modern			
\usepackage[T1]{fontenc}		% Selecao de codigos de fonte.
\usepackage[utf8]{inputenc}		% Codificacao do documento (conversão automática dos acentos)
\usepackage{lastpage}			% Usado pela Ficha catalográfica
\usepackage{indentfirst}		% Indenta o primeiro parágrafo de cada seção.
\usepackage{xcolor}             % Controle das cores (mais moderno)
\usepackage{graphicx}			% Inclusão de gráficos
\usepackage{microtype} 			% para melhorias de justificação
\usepackage{longtable} 			% Adicionado para tabelas longas
\usepackage{tabularx} 			% Adicionado para tabelas que seajustam
\usepackage{xurl} 				% Adicionado para quebrar URLs
% ---
		
% ---
% Pacotes de citações
% ---
\usepackage[brazilian,hyperpageref]{backref}	 % Paginas com as citações na bibl
\usepackage[alf]{abntex2cite}	% Citações padrão ABNT
\usepackage{ragged2e}

% --- 
% CONFIGURAÇÕES DE PACOTES
% --- 

% ---
% Configurações do pacote backref
% Usado sem a opção hyperpageref de backref
\renewcommand{\backrefpagesname}{Citado na(s) página(s):~}
% Texto padrão antes do número das páginas
\renewcommand{\backref}{}
% Define os textos da citação
\renewcommand*{\backrefalt}[4]{
	\ifcase #1 %
		Nenhuma citação no texto.%
	\or
		Citado na página #2.%
	\else
		Citado #1 vezes nas páginas #2.%
	\fi}%
	
% Pacotes para tabelas (colunas de largura fixa e alinhamento)
\usepackage{array}
\usepackage{booktabs}
\usepackage{ragged2e} % Para alinhar texto justificado/à esquerda em colunas p{}
\usepackage{longtable} % Essencial para tabelas longas
\usepackage{hyperref}  % Para links (sempre o último a ser carregado)
	
	
% ---

% ---
% Informações de dados para CAPA e FOLHA DE ROSTO
% ---
\titulo{CarneUp: Sistema de Gestão de Açougue de Pequeno e Médio Porte}


\autor{BRENO DIAS OLIVEIRA SP3015645\\FELIPE DALBOSCO 
PALUDO SP3123766\\GUSTAVO GOUVEA ANDRADE SP3076725\\PEDRO AUGUSTO SILVA SANTOS SP3098559\\RAQUEL CORREIA DA SILVA 
SP3098567\\WELLINGTON OLIVEIRA DE SOUSA SP3161307}

\local{São Paulo - SP - Brasil}
\data{2025}
\orientador{Marcelo Tavares de Santana}

\instituicao{%
   IFSP- Instituto Federal de Educação, Ciência e Tecnologia
 Câmpus São Paulo
  \par
   Tecnologia em Análise e Desenvolvimento de Sistemas}

% O preambulo deve conter o tipo do trabalho, o objetivo, 
% o nome da instituição e a área de concentração 
\preambulo{Trabalho de Projeto Integrado I apresentado ao curso de Tecnologia em Análise e Desenvolvimento de Sistemas do Instituto Federal de Educação, Ciência e Tecnologia de São Paulo, como parte dos requisitos para a conclusão da disciplina.}
% ---


% ---
% Configurações de aparência do PDF final
% ---

% alterando o aspecto da cor azul
\definecolor{blue}{RGB}{41,5,195}

% informações do PDF
\makeatletter
\hypersetup{
     	%pagebackref=true,
		pdftitle={\@title}, 
		pdfauthor={\@author},
    	pdfsubject={\imprimirpreambulo},
	    pdfcreator={LaTeX with abnTeX2},
		pdfkeywords={abnt}{latex}{abntex}{abntex2}{trabalho acadêmico}, 
		colorlinks=true,       		% false: boxed links; true: colored links
    	linkcolor=blue,          	% color of internal links
    	citecolor=blue,        		% color of links to bibliography
    	filecolor=magenta,      		% color of file links
		urlcolor=blue,
		bookmarksdepth=4
}
\makeatother
% --- 

% --- 
% Espaçamentos entre linhas e parágrafos 
% --- 

% O tamanho do parágrafo é dado por:
\setlength{\parindent}{1.3cm}

% Controle do espaçamento entre um parágrafo e outro:
\setlength{\parskip}{0.2cm}  % tente também \onelineskip

% ---
% compila o indice
% ---
\makeindex
% ---

% ----
% Início do documento
% ----
\begin{document}

% Retira espaço extra obsoleto entre as frases.
\frenchspacing 

% ----------------------------------------------------------
% ELEMENTOS PRÉ-TEXTUAIS
% ----------------------------------------------------------
% \pretextual

% ---
% Capa
% ---
\imprimircapa
% ---

% ---
% Folha de rosto
% (o * indica que haverá a ficha bibliográfica)
% ---
\imprimirfolhaderosto*
% ---

% ---
% RESUMOS
% ---

% resumo em português
% \setlength{\absparsep}{18pt} % <<< CORREÇÃO: Comando comentado, pois é desnecessário com parágrafo único
\begin{resumo}
Este trabalho propõe o desenvolvimento do CarneUp, um sistema de gerenciamento de açougues focado em digitalizar e otimizar as operações de venda, compra e controle de estoque de carnes[cite: 13]. A ausência de controle digital em muitas boutiques e açougues de pequeno e médio porte, onde as vendas e o gerenciamento de estoque ainda são feitos de forma analógica em cadernos, gera ineficiência e perda de informações cruciais[cite: 14]. O CarneUp visa solucionar este problema, oferecendo uma plataforma robusta e intuitiva para o gerenciamento completo do negócio, desde o registro de vendas com captura automática de peso/valor via código de barras, até o controle de validade e a rastreabilidade por fornecedor[cite: 15]. O sistema é projetado para açougues de pequeno e médio porte, fornecendo ferramentas essenciais para aumentar a precisão do estoque, melhorar o controle financeiro por peça e reduzir o desperdício, garantindo assim a eficiência e a competitividade do negócio[cite: 16].

\textbf{Palavras-chaves}: Gestão de açougue, Controle de estoque de carnes, Sistema de vendas, Software de açougue[cite: 17].
\end{resumo}

% resumo em inglês
\begin{resumo}[Abstract]
 \begin{otherlanguage*}{english}
   This work proposes the development of CarneUp, a butcher shop management system focused on digitizing and optimizing sales, purchasing, and meat stock control operations[cite: 19]. The lack of digital control in many small and medium-sized butcher shops, where sales and inventory management are still carried out analogically in notebooks, leads to inefficiency and loss of crucial information[cite: 20]. CarneUp aims to solve this problem by offering a robust and intuitive platform for complete business management, from sales registration with automatic weight/value capture via barcode, to validity control and traceability by supplier[cite: 21]. The system is designed for small and medium-sized butcher shops, providing essential tools to increase inventory accuracy, improve financial control per piece, and reduce waste, thereby ensuring business efficiency and competitiveness[cite: 22].
 
   \vspace{\onelineskip}
 
   \noindent 
   \textbf{Key-words}: Butcher shop management, Meat inventory control, Sales system, Butcher shop software[cite: 23].
 \end{otherlanguage*}
\end{resumo}

% ---
% inserir lista de ilustrações
% ---
\pdfbookmark[0]{\listfigurename}{lof}
\listoffigures*
\cleardoublepage
% ---

% ---
% inserir lista de tabelas
% ---
\pdfbookmark[0]{\listtablename}{lot}
\listoftables*
\cleardoublepage
% ---

% ---
% inserir lista de abreviaturas e siglas
% ---
\begin{siglas}
  \item[API] Application Programming Interface
  \item[AWS] Amazon Web Services
  \item[CRUD] Create, Read, Update, Delete
  \item[FDD] Feature Driven Development
  \item[MER] Modelo Entidade-Relacionamento
  \item[MVP] Produto Mínimo Viável
  \item[PDV] Ponto de Venda
  \item[POC] Prova de Conceito
  \item[RF] Requisito Funcional
  \item[RNF] Requisito Não Funcional
  \item[SaaS] Software as a Service
  \item[Scrum] (Metodologia Ágil)
  \item[SSL/TLS] Secure Sockets Layer/Transport Layer Security
\end{siglas}
% ---

% ---
% inserir o sumario
% ---
\pdfbookmark[0]{\contentsname}{toc}
\tableofcontents*
\cleardoublepage
% ---

% ----------------------------------------------------------
% ELEMENTOS TEXTUAIS
% ----------------------------------------------------------
\textual

% ----------------------------------------------------------
% CAPÍTULO 1: INTRODUÇÃO
% ----------------------------------------------------------
\chapter[Introdução]{Introdução}

O principal objetivo do projeto \textit{CarneUp} é desenvolver um sistema de gerenciamento digital focado nas operações de açougues de pequeno e médio porte. Este sistema visa permitir a digitalização e a otimização da gestão de estoque e do processo de vendas no Ponto de Venda (PDV).

O problema central que motiva este projeto é a gestão predominantemente analógica e ineficiente encontrada em muitos desses estabelecimentos. Atualmente, operações críticas como o controle de vendas, o registro de compras e o acompanhamento de estoque são realizados manualmente em cadernos ou planilhas. Esta abordagem é não apenas propensa a erros, mas também impede uma análise precisa da margem de lucro por corte e um controle detalhado por fornecedor ou validade. Consequentemente, a falta de um sistema digital resulta em desperdício de estoque vencido e na perda de dados valiosos para a tomada de decisão estratégica.

Diante desse cenário, a relevância do \textit{CarneUp} reside na necessidade urgente de modernização deste setor. A solução proposta visa digitalizar e centralizar a gestão do açougue, oferecendo uma aplicação web robusta e com foco na facilidade de uso para o público-alvo. Ao automatizar o registro de vendas via código de barras e implementar um controle de validade e de custos por fornecedor, o sistema impacta diretamente a eficiência operacional, o controle financeiro e a redução de desperdício.

\section{Escopo e Objetivos}

O Escopo do projeto abrange a criação de um sistema web focado nas operações de retaguarda (back-office) e no ponto de venda (PDV) de açougues. O sistema tem como foco principal o controle de estoque e o controle financeiro detalhado de cada peça de carne.

\textbf{O projeto abrange:}
\begin{itemize}
    \item Gestão de Vendas (PDV) com integração de código de barras.
    \item Controle de Estoque (Entrada/Saída), incluindo fornecedor, valor de custo e validade.
    \item Emissão de alertas de validade.
    \item Geração de relatórios básicos de vendas.
    \item Gestão de usuários e controle de sessão (Autenticação). % <<< ESTE ITEM
    \item Aplicação de padrões de acessibilidade (WCAG) na interface web.
\end{itemize}

Os objetivos específicos do projeto incluem:

\begin{itemize}
    \item Gerenciar Vendas: Permitir o registro rápido e eficiente de vendas de carnes.
    \item Controle de Estoque: Fornecer uma visão em tempo real do estoque de carnes, incluindo o controle das peças vendidas, suas validades e fornecedores.
    \item Digitalização de Pesagem: Integrar-se com códigos de barras gerados pela balança para capturar automaticamente o valor e o peso da peça no momento da venda.
    \item Otimização de Compras: Registrar o valor pago nas carnes e seus fornecedores para análises de custo.
\end{itemize}

\section{Requisitos do Sistema}

Para alcançar os objetivos propostos, o \textit{CarneUp} atende a um conjunto de requisitos funcionais (RF) e não funcionais (RNF e RN). Os requisitos funcionais estão detalhados na Tabela 1, agrupados de acordo com os principais componentes de gestão definidos nos diagramas de arquitetura: \textbf{Gestão de Vendas}, \textbf{Gestão de Produtos (Estoque)}, \textbf{Gestão de Compras} e \textbf{Gestão de Autenticação}.

\begin{longtable}{|>{\Centering}m{1.5cm}|>{\RaggedRight}m{13cm}|}
    \caption{Requisitos Funcionais (RF)}
    \label{tab:requisitos_funcionais}\\
    \hline
    \textbf{Código} & \textbf{Descrição} \\ 
    \hline
    \endfirsthead

    \multicolumn{2}{c}%
    {{\bfseries \tablename\ \thetable{} -- continuação}} \\
    \hline
    \textbf{Código} & \textbf{Descrição} \\ 
    \hline
    \endhead
    \hline
    \multicolumn{2}{|r|}{{Continua...}} \\
    \hline
    \endfoot
    \hline
    \endlastfoot

    % --- Requisitos de Gestão de Vendas (Componente) ---
    RF01 & O sistema deve permitir o registro de vendas através da leitura do código de barras gerado pela balança. \\ 
    \hline
    RF02 & O sistema deve calcular automaticamente o valor total da venda ao capturar o peso e o valor unitário do item lido via código de barras. \\ 
    \hline
    RF08 & O sistema deve gerar relatórios diários de vendas, detalhando a quantidade e o valor total vendido. \\ 
    \hline

    % --- Requisitos de Gestão de Produtos (Estoque) (Componente) ---
    RF03 & O sistema deve permitir o registro de novas entradas de estoque (compras), incluindo valor de custo, fornecedor e validade. \\ 
    \hline
    RF05 & O sistema deve calcular a margem de lucro de cada peça de carne vendida (Valor de Venda - Valor de Custo). \\ 
    \hline
    RF06 & O sistema deve emitir um alerta visual e via e-mail para o administrador quando a validade de uma peça de estoque estiver a 7 dias ou menos do vencimento. \\ 
    \hline
    RF07 & O sistema deve permitir o descarte de peças de estoque (baixa) e registrar o motivo do descarte (ex: vencimento, perda de qualidade). \\ 
    \hline

    % --- Requisitos de Gestão de Compras (Componente) ---
    RF04 & O sistema deve permitir a busca e visualização de peças de estoque por fornecedor. \\ 
    \hline
    RF09 & O sistema deve gerenciar o cadastro completo de fornecedores e seus respectivos contatos. \\ 
    \hline

    % --- Requisitos de Gestão de Autenticação (Componente) [CORREÇÃO] ---
    RF10 & O sistema deve controlar o acesso através de autenticação de usuário (login/senha) e gerenciar o ciclo de vida da sessão. \\ 
    \hline
    RF11 & O sistema deve permitir o cadastro de diferentes perfis de usuários (ex: Operador de Caixa, Gerente) com níveis de permissão distintos. \\
    \hline
\end{longtable}

A Tabela 2 apresenta as regras de negócio obrigatórias, que se referem a regras e restrições de operação.

\begin{longtable}{|>{\Centering}m{1.5cm}|>{\RaggedRight}m{9cm}|>{\Centering}m{4cm}|}
    \caption{Requisitos Não Funcionais (RN) - Regras de Negócio}
    \label{tab:requisitos_rn}\\
    \hline
    \textbf{Código} & \textbf{Descrição} & \textbf{Requisito Relacionado} \\ 
    \hline
    \endfirsthead

    \multicolumn{3}{c}%
    {{\bfseries \tablename\ \thetable{} -- continuação}} \\
    \hline
    \textbf{Código} & \textbf{Descrição} & \textbf{Requisito Relacionado} \\ 
    \hline
    \endhead
    \hline
    \multicolumn{3}{|r|}{{Continua...}} \\
    \hline
    \endfoot
    \hline
    \endlastfoot
    
    RN01 & A venda só pode ser registrada se o código de barras lido for reconhecido e o item tiver estoque disponível. & RF01, RF02 \\ 
    \hline
    RN02 & O valor de custo por peça deve ser registrado no momento da entrada no estoque para cálculo de margem de lucro. & RF03, RF05 \\ 
    \hline
    RN03 & Peças com validade expirada devem ser automaticamente bloqueadas para venda e movidas para a lista de descarte. & RF07, RNF01 \\ 
    \hline
    RN04 & O sistema deve manter o histórico de fornecedores para cada peça de carne no estoque para fins de análise de custo e consulta. & RF04 \\ 
    \hline
    RN05 & A peça de carne vendida deve ter seu estoque subtraído em tempo real no momento do registro da venda. & RF02, RNF01 \\ 
    \hline
\end{longtable}

A Tabela 3 detalha os requisitos não funcionais que definem atributos de qualidade do software, como performance, usabilidade e segurança.

\begin{longtable}{|>{\Centering}m{1.5cm}|>{\Centering}m{3cm}|>{\RaggedRight}m{10cm}|}
    \caption{Requisitos Não Funcionais (RNF) - Atributos de Qualidade}
    \label{tab:requisitos_rnf}\\
    \hline
    \textbf{Código} & \textbf{Módulo} & \textbf{Descrição} \\ 
    \hline
    \endfirsthead

    \multicolumn{3}{c}%
    {{\bfseries \tablename\ \thetable{} -- continuação}} \\
    \hline
    \textbf{Código} & \textbf{Módulo} & \textbf{Descrição} \\ 
    \hline
    \endhead
    \hline
    \multicolumn{3}{|r|}{{Continua...}} \\
    \hline
    \endfoot
    \hline
    \endlastfoot

    RNF01 & Performance & O registro de venda e a baixa de estoque devem ser processados em no máximo 2 segundos para visar o fluxo rápido de caixa. \\ 
    \hline
    RNF02 & Escalabilidade & O sistema deve ser capaz de suportar picos de até 10 usuários simultâneos (caixas + gerentes) sem degradação perceptível de performance. \\ 
    \hline
    RNF03 & Usabilidade & A interface de registro de vendas deve ser intuitiva e otimizada para telas de toque e leitores de código de barras. \\ 
    \hline
    RNF04 & Segurança & Todos os dados sensíveis (informações financeiras, cadastros) devem ser armazenados de forma criptografada (ex: senhas via bcrypt, dados via SSL). \\ 
    \hline
    RNF05 & Manutenibilidade & O código deve seguir uma convenção de código (\textit{Code Convention}) e utilizar logs detalhados para facilitar a identificação e correção de erros. \\ 
    \hline
\end{longtable}

\section{Problema e Solução Proposta}

O problema central abordado pelo \textit{CarneUp} é a gestão predominantemente analógica e ineficiente encontrada em açougues de pequeno e médio porte. Nesses estabelecimentos, operações críticas como o controle de vendas, o registro de compras, o controle de validade e o acompanhamento de estoque são realizados manualmente, seja em cadernos ou planilhas.

Esta prática manual é altamente suscetível a erros e acarreta diversas ineficiências operacionais e financeiras: dificulta o controle de custos por fornecedor, impede uma análise precisa da margem de lucro por corte e leva a perdas diretas por estoque vencido. A ausência de um sistema digital, portanto, resulta na falta de dados confiáveis para a tomada de decisões estratégicas.

Para endereçar essa lacuna de gestão, a solução proposta é o \textit{CarneUp}. Trata-se de uma aplicação web desenhada para digitalizar e centralizar a gestão de estoque e vendas do açougue. Com foco na facilidade de uso para o público-alvo, o sistema prioriza a eficiência operacional no Ponto de Venda (PDV) e o controle detalhado das peças de carne, resolvendo diretamente os desafios de controle de custo, validade e registro de vendas.

As funcionalidades chave do \textit{CarneUp} são:

\begin{itemize}
    \item Registro de Vendas Otimizado: Utilização de códigos de barras da balança para preenchimento automático de valor e peso da peça vendida.
    \item Controle de Custo por Fornecedor: Registro de dados como fornecedor, valor de compra e validade de cada lote de carne.
    \item Controle de Estoque Inteligente: Emissão de alertas de validade próxima e visualização clara das peças em estoque.
\end{itemize}


\section{Justificativa}

A relevância do \textit{CarneUp} se justifica pela necessidade de modernização do setor de açougues de pequeno e médio porte. Ao fornecer uma solução digital acessível, o \textit{CarneUp} contribui diretamente para:

\begin{itemize}
    \item Controle Financeiro: O registro do valor pago por peça permite uma análise de custo precisa, otimizando a margem de lucro e identificando os cortes mais rentáveis.
    
    \item Redução de Desperdício: O controle de validade digitalizado ajuda a reduzir as perdas de estoque devido ao vencimento.
    
    \item Eficiência Operacional: A automação do registro de vendas via código de barras acelera o atendimento e minimiza a chance de erros no caixa.
\end{itemize}

A inexistência de um sistema digital que combine controle de estoque detalhado (fornecedor, valor de compra, validade) com uma interface de PDV eficiente para açougues menores torna o \textit{CarneUp} uma ferramenta estratégica para a competitividade desses negócios.

% ---
\section{Análise da Concorrência}
% ---
O CarneUp se posiciona contra sistemas de gestão (PDV) mais genéricos, que não possuem a granularidade necessária para o controle de carnes com base em peso variável, validade e fornecedor, sendo adaptados para o segmento de açougues.

\subsection{Concorrente 1: ConnectPlug}

O ConnectPlug é um sistema de PDV (Ponto de Venda) e gestão comercial mais abrangente, atendendo diversos segmentos, incluindo açougues. Ele oferece funcionalidades como frente de caixa, gestão de estoque e emissão de notas fiscais. \cite{connectplug}

\begin{itemize}
    \item Foco: Solução completa de gestão comercial.

    \item Diferencial (para o CarneUp): Pode ser adaptado para açougues, mas pode não ter o foco e a profundidade no **controle de custos por peça** (valor de compra, fornecedor, validade individualizada) que o CarneUp busca.
\end{itemize}

\subsection{Concorrente 2: Aliar Sistemas}

A Aliar Sistemas offers soluções específicas para o varejo de carnes, com foco em açougues e frigoríficos. Seu sistema inclui gestão de produção, **controle de lotes** e controle de estoque. \cite{aliar}

\begin{itemize}
    \item Foco: Soluções específicas para o setor de carnes.

    \item Por focar em frigoríficos e grandes açougues, pode ter um custo elevado e complexidade excessiva para açougues de pequeno porte, que é o alvo do CarneUp.
\end{itemize}

\subsection{Concorrente 3: SOFTClass}

A SOFTClass oferece software de gestão para diversos segmentos do varejo, incluindo açougues. Possui módulos de frente de caixa, estoque e financeiro.\cite{softclass}

\begin{itemize}
    \item Foco: Solução de gestão empresarial para varejo.

    \item Semelhante ao ConnectPlug, pode ser mais um sistema de PDV adaptado, carecendo da simplicidade e do foco no ciclo de vida da carne (compra/validade/fornecedor) que é o diferencial do CarneUp.
\end{itemize}


\subsection{Comparativo}
O quadro abaixo resume as principais funcionalidades e destaca o diferencial do CarneUp.

\begin{table}[h]
    \centering
    \caption{Comparativo}
    \label{tab:placeholder_label}
    \begin{tabularx}{\linewidth}{|>{\RaggedRight}p{4cm}|c|c|c|c|}
        \hline
        \textbf{Funcionalidade} & \textbf{CarneUp} & \textbf{ConnectPlug} & \textbf{Aliar Sistemas} & \textbf{SOFTClass} \\ \hline
        Registro de Vendas & X & X & X & X \\ \hline
        Controle Básico de Estoque & X & X & X & X \\ \hline
        Captura de Valor/Peso por Código de Barras da Balança & X & X & X & X \\ \hline
        \textbf{Controle Detalhado (Fornecedor e Custo por Peça)} & X & X & - & - \\ \hline
        Alertas de Validade por Peça/Lote & X & - & - & - \\ \hline
        Foco em Pequenos/Médios Açougues e Facilidade de Uso & X & - & - & X \\ \hline
    \end{tabularx}
\end{table}

% ----------------------------------------------------------
% CAPÍTULO 2: GESTÃO DO PROJETO (SEÇÃO 3 NO ORIGINAL)
% * <Felipe Paludo> 23:07:07 10 Nov 2025 UTC-0300:
% REFEITO
% ----------------------------------------------------------
\chapter{Gestão do Projeto}\label{cap_exemplos}
% ACRESCENTADO PARAGRAFO INTRODUTORIO NO CAPITULO 2
A presente seção detalha o planejamento, a organização e o controle das atividades essenciais para a execução do projeto, este capítulo estabelece a estrutura da equipe, define os papéis e responsabilidades dos membros, e descreve a metodologia ágil adotada, incluindo a estruturação do cronograma e os ciclos de desenvolvimento. Por fim, são apresentadas as ferramentas de versionamento e o acesso ao código-fonte.


% ---
\section{Organização da Equipe}
% ---

O projeto será desenvolvido por uma equipe de cinco membros, com papéis definidos para cobrir as áreas de desenvolvimento, administração de dados e gestão, conforme apresentado na Tabela 2.

% ---
%\subsection{Responsabilidades / Papéis / Atividades}
% ---

\begin{table}[h!]
    \centering
    \caption{Papéis, Integrantes e Responsabilidades Chave}
    \label{tab:papeis_equipe}
    
    \begin{tabular}{|>{\Centering}m{2.5cm}|>{\Centering}m{3.5cm}|>{\RaggedRight}m{7cm}|}
        \hline
        \textbf{Papel} & \textbf{Integrante} & \textbf{Responsabilidades Chave} \\ 
        \hline
        Product Owner & Raquel Correia da Silva & Definição e priorização do Product Backlog, contato com o cliente, validação das entregas. \\ 
        \hline
        Scrum Master & Felipe Dalbosco Paludo & Visar a aplicação correta da metodologia Scrum, remover impedimentos, facilitar as reuniões. \\ 
        \hline
        Desenvolvedor Full Stack & Gustavo Gouvea Andrade & Implementação das funcionalidades no front-end e back-end, integração com banco de dados. \\ 
        \hline
        Desenvolvedor Front-end & Breno Dias Oliveira & Desenvolvimento da interface do usuário, usabilidade e responsividade. \\ 
        \hline
        DBA / Desenvolvedor Back-end & Pedro Auguso Silva Santos & Modelagem e Administração do Banco de Dados, desenvolvimento de APIs e lógica de negócios. \\ 
        \hline
        DBA / Desenvolvedor Back-end & Wellington Oliveira de Sousa & Modelagem e Administração do Banco de Dados, desenvolvimento da documentação LateX. \\ 
        \hline
    \end{tabular}
\end{table}
%FIGURA 1 SUBSTITUIDA POR TABELA
% ---
 % \begin{figure}[htb]
	%\begin{center}
	 %   \includegraphics[scale=0.5]{Imagens/Responsabilidades_Papeis_Atividades.png}
       % \caption{\label Responsabilidades / Papéis / Atividades}
	%\end{center}
%\end{figure}


% ---
\section{Metodologias de gestão e desenvolvimento}
% ---
%ALTERAÇÕES NA SECÇÃO 2 PEDIDAS PELO PROFESSOR, TEXTO ANTIGO ESTA COMENTADO
O projeto foi planejado para ter uma duração total de aproximadamente 2,5 meses, iniciando em 6 de Setembro de 2025 e com previsão de conclusão em 31 de Outubro de 2025. Essa duração é ditada pelos Marcos Fixos de Entrega.

O desenvolvimento do CarneUp adota a metodologia ágil Scrum, organizada em ciclos de trabalho curtos chamados Sprints. Dada a restrição temporal imposta pelos marcos de entrega acadêmica, o projeto foi dividido em 4 Sprints principais com duração variável (aproximadamente 10 a 14 dias úteis) para focar na entrega dos marcos fixos. O foco não é no número de Sprints, mas sim no cumprimento dos marcos de entrega, apresentado na tabela 3, que guiam o ritmo do projeto.

%O projeto utilizará a metodologia Scrum para o desenvolvimento e a gestão de projetos.

% ---
%---\subsection{Scrum / Sprints }
% ---
%O Scrum é uma metodologia ágil que permite à equipe se auto-organizar e fazer mudanças de forma rápida, em resposta a novos requisitos ou feedback do cliente. Ele se baseia em ciclos de trabalho chamados Sprints e em três papéis principais (Scrum Master, Product Owner e Time de Desenvolvimento) \cite{scrum20}, garantindo clareza e foco na entrega de valor contínuo. A escolha do Scrum se deve à sua capacidade de lidar com a evolução dos requisitos e garantir entregas incrementais do sistema CarneUp ao longo do tempo. 

%O desenvolvimento do CarneUp adota a metodologia ágil Scrum, organizada em ciclos de trabalho curtos chamados Sprints. Dada a restrição temporal imposta pelos marcos de entrega acadêmica, o projeto foi dividido em 4 Sprints principais com duração variável (aproximadamente 10 a 14 dias úteis) para focar na entrega dos marcos fixos, conforme a tabela detalhada na seção 3.2.1.1.
%O foco não é no número de Sprints, mas sim no cumprimento dos seguintes marcos de entrega que guiam o ritmo do projeto:


% Tabela dos Marcos de Entrega
\begin{table}[h!]
    \centering
    \caption{Marcos de Entrega do Projeto \textit{CarneUp}}
    \label{tab:marcos_entrega}
    
    % Definição das colunas: m{2.5cm} para Data (centralizado) e m{11cm} para Atividade (alinhado à esquerda)
    \begin{tabular}{|>{\Centering}m{2.5cm}|>{\RaggedRight}m{11cm}|}
        \hline
        \textbf{Data} & \textbf{Marco de Entrega / Atividade} \\ 
        \hline
        \textbf{06/09/2025} & Início do Desenvolvimento Geral e da Documentação. \\ 
        \hline
        \textbf{20/09/2025} & Entrega do Desenho da Aplicação. \\ 
        \hline
        \textbf{10/10/2025} & Apresentação da Prova de Conceito (POC). \\ 
        \hline
        \textbf{31/10/2025} & Entrega do Projeto Final (MVP) e sua Documentação. \\ 
        \hline
    \end{tabular}
\end{table}

% ---
\section{Repositório da aplicação}
% ---

% ---
%SUBSEÇÃO ELIMINADA
%\subsection{Definição do Repositório da Aplicação}
% ---

O projeto utilizará o GitHub como plataforma de versionamento de código e colaboração. O GitHub permite o controle detalhado das alterações de código (commits), facilitando o trabalho em equipe, a revisão de código (Pull Requests) e o rastreamento de problemas (Issues).


% ---
%SUBSEÇÃO ELIMINADA
%\subsection{Link do Repositório e Especificações para Acesso}
% ---
%Plataforma: GitHub

Link: https://github.com/wellingtonwos/ProjetoExtensaoI
Acesso: O repositório será privado durante o desenvolvimento e pode ser tornado público (ou manter-se privado com acesso via convite) após a Entrega Final. Para fins de avaliação, os colaboradores deverão ser convidados a ter acesso "Read" ou "Collaborator".


% ----------------------------------------------------------
% CAPÍTULO 3: DESENVOLVIMENTO (SEÇÃO 4 NO ORIGINAL)
% ----------------------------------------------------------
\chapter{Desenvolvimento Do Projeto}\label{cap:desenvolvimento} % <<< CORREÇÃO: Label única


% ---
\section{Escopo do Projeto}
% ---

CarneUp abrange o desenvolvimento de uma aplicação web completa para a gestão de 
açougues de pequeno e médio porte, com foco nas operações de Vendas, Compras e Estoque.
% ---
\subsection{Regras de Negócio}
% ---
  \begin{figure}[htb]
	\centering
	    \includegraphics[width=0.9\linewidth]{Imagens/Regras_de_Negocio.png} % <<< CORREÇÃO: scale substituído
        \caption{Regras de Negócio} % <<< CORREÇÃO: Sintaxe
	    \label{fig:regras_negocio} % <<< CORREÇÃO: Sintaxe
	\end{figure}

% ---
\subsection{Requisitos Funcionais}
% ---

  \begin{figure}[htb]
	\centering
	    \includegraphics[width=0.7\linewidth]{Imagens/Requisitos_Funcionais.png} % <<< CORREÇÃO: scale substituído
        \caption{Requisitos Funcionais} % <<< CORREÇÃO: Sintaxe
	    \label{fig:req_funcionais} % <<< CORREÇÃO: Sintaxe
	\end{figure}

% ---
\subsection{Requisitos Não Funcionais}
% ---

.
\begin{figure}[htb]
	\centering
	    \includegraphics[width=0.7\linewidth]{Imagens/Requisitos_Nao_Funcionais.png} % <<< CORREÇÃO: scale substituído
        \caption{Requisitos Não Funcionais} % <<< CORREÇÃO: Sintaxe
	    \label{fig:req_nao_funcionais} % <<< CORREÇÃO: Sintaxe
	\end{figure}


% ---
\section{Histórias de Usuário}
% ---
Como a metodologia de desenvolvimento escolhida é o Scrum, as funcionalidades serão expressas através de Histórias de Usuário.
% ---
\subsection{Descrição das Histórias de Usuário}
% ---
\begin{itemize}
    \item US01: Registrar Venda com Código de Barras (RF01, RF02)
        \begin{itemize}
            \item Descrição: Como um operador de caixa, quero registrar uma venda lendo o código de barras da balança, para que o sistema preencha automaticamente o item, peso e valor, agilizando o atendimento.
            \item Critérios de Aceitação: O código de barras deve ser lido pelo leitor, o sistema deve exibir o item e valor unitário.
O operador deve confirmar a venda e o estoque deve ser atualizado.
        \end{itemize}
\end{itemize}

% <<< CORREÇÃO: Bloco US01 duplicado removido daqui

\begin{itemize}
    \item US02: Inclusão de Estoque com Rastreabilidade (RF03, RF04)
        \begin{itemize}
            \item Descrição: Como um gerente de compras, quero registrar a entrada de novas peças de carne, incluindo o fornecedor, valor de custo e data de validade, para que eu possa ter controle de rastreabilidade e calcular a margem de lucro.
            \item Critérios de Aceitação: Deve haver um formulário com campos obrigatórios para Fornecedor, Data de Validade e Valor de Custo.
O registro deve ser armazenado no banco de dados com todas as informações.
        \end{itemize}
\end{itemize}

\begin{itemize}
    \item US03: Alerta de Validade Próxima (RF06, RNF05)
        \begin{itemize}
            \item Descrição: Como um gerente, quero receber um alerta no dashboard e por e-mail quando um item estiver a 7 dias do vencimento, para que eu possa tomar medidas (promoção/descarte) antes da perda total do produto.
            \item Critérios de Aceitação: O alerta deve ser exibido no dashboard principal com destaque.
Um e-mail deve ser enviado para o e-mail do gerente.
        \end{itemize}
\end{itemize}

\begin{itemize}
    \item US04: Visualização de Margem de Lucro (RF05)
        \begin{itemize}
            \item Descrição: Como um proprietário, quero visualizar a margem de lucro por peça de carne vendida, para identificar quais cortes são mais rentáveis.
            \item Critérios de Aceitação: Deve haver uma tela de relatório que liste as vendas e mostre a diferença (em R\$ e \% ) entre o valor de venda e o valor de custo.
        \end{itemize}
        
\end{itemize}

\begin{itemize}
    \item US05: Gerenciamento de Descarte (RF07)
        \begin{itemize}
            \item Descrição: Como um gerente de estoque, quero registrar o descarte de uma peça, indicando o motivo (vencimento, estrago), para que o estoque seja baixado e o motivo da perda seja registrado para análise.
            \item Critérios de Aceitação: A baixa de estoque deve ocorrer.
O sistema deve exigir a seleção do motivo e a data do descarte.
        \end{itemize}
\end{itemize}

% ---
\section{Arquitetura}
% ---

% <<< CORREÇÃO: Texto padronizado para Java/Spring Boot
O CarneUp utilizará uma arquitetura de aplicação web de três camadas (Front-end, Back-end e Banco de Dados), implantada na nuvem da AWS.
O Front-end será desenvolvido em React e se comunicará com o Back-end via API RESTful.
O Back-end será construído utilizando Java com Spring Boot.
O banco de dados PostgreSQL será o coração do sistema para garantir a integridade transacional dos dados de estoque e venda.
A escolha da AWS garante escalabilidade e alta disponibilidade.

% ---
\subsection{Diagrama da Arquitetura}
% ---
% Adicionar diagrama se houver
% \begin{figure}[htb]
% 	\centering
% 	\includegraphics[width=0.9\linewidth]{Imagens/Diagrama_Arquitetura.png}
% 	\caption{Diagrama da Arquitetura}
% 	\label{fig:arquitetura}
% \end{figure}

% ---
\section{Tecnologias}
% ---

Foram escolhidas para garantir robustez, performance e facilidade de manutenção.
% ---
\subsection{Front-end}
% ---

\begin{itemize}
    \item React: Biblioteca JavaScript para construção de interfaces de usuário reativas e componentizadas.
Permite a criação de um Front-end rápido e modular (RNF03).
    
    \item Bootstrap: Framework de Front-end para design responsivo e consistente.
Acelera o desenvolvimento da interface.
\end{itemize}

% ---
\subsection{Back-end}
% ---

\begin{itemize}
    \item A linguagem principal do Back-end será Java, utilizando o framework Spring Boot.
O Spring Boot é ideal para construir microsserviços e sistemas transacionais de alta segurança e performance, como o módulo de registro de vendas e controle de estoque.
\end{itemize}

% ---
\subsection{Banco de Dados}
% ---
\begin{itemize}
    \item PostgreSQL: Sistema de gerenciamento de banco de dados relacional robusto e confiável, fundamental para a integridade dos dados de vendas e estoque (RNF04).
\end{itemize}
% ---
\subsection{Infraestrutura}
% ---

\begin{itemize}
    \item AWS (Amazon Web Services): Plataforma de computação em nuvem que fornecerá os serviços de hospedagem e deployment.
\end{itemize}

% ---
\section{Logs}
% ---

Será utilizada a biblioteca de logging nativa do Spring Boot (como SLF4J/Logback) para registrar logs de erro, transação e alertas de validade (RF05).
% ---
\section{Segurança, Privacidade e Legislação}
% ---


% ---
\subsection{Critérios de Segurança e Privacidade}
% ---
\begin{itemize}
    \item Criptografia de Senhas: Utilização de bcrypt para armazenar as senhas dos usuários (administradores/caixas) no banco de dados (RNF04).
\end{itemize}
% ---
\subsubsection{Criptografia em Trânsito: SSL/HTTPS}
% ---
O protocolo HTTPS (HyperText Transfer Protocol Secure) é essencial para o projeto CarneUp, pois garante a segurança das comunicações entre o navegador do usuário (cliente) e o servidor de aplicação (Back-end).
O HTTPS é, fundamentalmente, o protocolo HTTP combinado com a camada de segurança SSL/TLS (Secure Sockets Layer/Transport Layer Security).
Essa combinação cria um link criptografado que protege os dados transmitidos contra interceptação e adulteração, garantindo a confidencialidade e a integridade das informações.
\begin{alineas}
    \item Critérios de Segurança e Conformidade
            A implementação do SSL/TLS no CarneUp atende a critérios cruciais para um sistema de gestão:
    \begin{itemize}
        \item Proteção de Credenciais: Durante o login e autenticação, as senhas (mesmo que criptografadas no banco de dados) são protegidas durante o trânsito da máquina do cliente até o servidor, impedindo ataques de sniffing ou Man-in-the-Middle.
        \item Integridade dos Dados de Venda: Garante que os dados financeiros críticos — como o valor final de uma venda, o valor de custo de um produto ou a atualização de estoque — não possam ser modificados por terceiros maliciosos durante a transmissão.
        \item Confiança do Usuário: O ícone de cadeado na barra de endereço (\url{HTTPS}) reforça a confiabilidade e o compromisso da aplicação com a segurança dos dados. % <<< CORREÇÃO: \usepackage{xurl} vai consertar o overflow
    \end{itemize}

    \item Implantação na Infraestrutura AWS
        Na arquitetura do CarneUp, a proteção HTTPS será implementada utilizando os serviços nativos da Amazon Web Services (AWS), garantindo alta disponibilidade e gerenciamento facilitado:

        \begin{itemize}
            \item Certificado SSL/TLS: Será utilizado um certificado SSL/TLS (emitido via AWS Certificate Manager - ACM, ou adquirido) para autenticar a identidade do domínio do CarneUp.
            \item Terminação SSL: A terminação (ou offloading) SSL será configurada em um Load Balancer (como o Application Load Balancer - ALB) ou no serviço de distribuição de conteúdo (AWS CloudFront, se utilizado), que são pontos de entrada para o tráfego.
            Isso permite que a carga de criptografia/descriptografia seja removida dos servidores de Back-end (Spring Boot), melhorando o desempenho da aplicação (RNF01). % <<< CORREÇÃO: Node.js/Python trocado por Spring Boot
            \item Comunicação Interna: Embora a criptografia seja mais crítica na comunicação externa (cliente-servidor), a comunicação interna entre os serviços (Back-end e PostgreSQL/AWS RDS) também pode ser configurada para usar conexões criptografadas, assegurando a segurança end-to-end em toda a infraestrutura.
        \end{itemize}

        Em suma, a obrigatoriedade do protocolo HTTPS é uma medida fundamental que blinda a aplicação contra as vulnerabilidades mais comuns da web, protegendo informações sensíveis em conformidade com as melhores práticas de segurança da informação (RNF04).
\end{alineas}
    

% ---
\subsection{Observância à Legislação}
% ---
Embora o sistema trate principalmente de dados do negócio (estoque, vendas), o CarneUp está em observância com a Lei Geral de Proteção de Dados (LGPD) no tratamento dos dados pessoais de seus usuários (proprietários, gerentes, caixas).
\begin{itemize}
    \item Princípio da Finalidade: Coleta apenas dos dados estritamente necessários para o funcionamento do sistema (nome, e-mail, senha criptografada).
    \item Logs e Rastreabilidade: Manutenção de logs para auditoria de acesso e alterações, conforme exigido por regulamentos de segurança.
    \item Consentimento: O registro de usuário administrador implica na aceitação da política de privacidade e uso do sistema.
    \item
    
\end{itemize}
% ---
\section{Modelo de Banco de Dados}
% ---
% ---
\subsection{Modelo Entidade Relacionamento (MER / DER)}
% ---
O modelo será centrado nas entidades de Estoque, Vendas e Fornecedores.
\begin{itemize}
    \item Entidade Estoque (Peça): Armazena dados do produto em estoque (código de barras, peso, valor de custo, fornecedor, validade).
    \item Entidade Venda: Armazena o registro de cada transação (peça vendida, valor final, data/hora, operador).
    \item Entidade Fornecedor: Armazena dados de contato e histórico do fornecedor.
\end{itemize}


% ---
\subsection{Diagrama Entidade Relacionamento (DER)}
% ---

  \begin{figure}[htb]
	\centering
	    \includegraphics[width=0.9\linewidth]{Imagens/DER.jpg} % <<< CORREÇÃO: scale substituído
        \caption{Diagrama Entidade Relacionamento (DER)} % <<< CORREÇÃO: Sintaxe
	    \label{fig:der} % <<< CORREÇÃO: Sintaxe
	\end{figure}



% ---
% <<< CORREÇÃO: Seção 4.8 e 4.8.1 removidas por serem contraditórias e redundantes.
% A duração correta (2,5 meses) já está definida no Capítulo 3.
% ---

% ---
\section{Dicionário de Dados}

  \begin{figure}[htb]
	\centering
	    \includegraphics[width=\linewidth]{Imagens/Dicionario_de_Dados.png} % <<< CORREÇÃO: scale substituído por width=\linewidth para corrigir overflow
        \caption{Dicionário de Dados} % <<< CORREÇÃO: Sintaxe
	    \label{fig:dicionario_dados} % <<< CORREÇÃO: Sintaxe
	\end{figure}


% ---
% ----------------------------------------------------------
% CAPÍTULO 4: VIABILIDADE (SEÇÃO 5 NO ORIGINAL)
% ----------------------------------------------------------
\chapter{Viabilidade Financeira}\label{cap:viabilidade} % <<< CORREÇÃO: Label única

A análise de viabilidade financeira simula a sustentabilidade do projeto em um período de 2,5 meses, considerando custos de mão de obra (MO) e infraestrutura.
% ---
\section{Custos}
% ---
A estimativa de custos de Mão de Obra (MO) é baseada em uma média de taxa horária (R\$ 50,00) para 5 desenvolvedores, limitados a 80 horas/mês cada, durante a nova duração total de 2,5 meses.
% ---
\subsection{Detalhamento dos Custos de Desenvolvimento (Mão de Obra)}
% ---

Foi calculado com base em 5 membros da equipe, trabalhando uma média de 80 horas/mês, durante os 2,5 meses.
\begin{itemize}
        
    \item Esforço Total: 1.000 horas (5 membros x 80 horas/mês x 2,5 meses).
    \item  Custo Horário Médio: R\$ 50,00.
    \item Custo Total de MO: R\$ 50.000,00.
\end{itemize}

% ---
\subsection{Custos de Infraestrutura}
% ---
Os custos operacionais de infraestrutura (manutenção e hospedagem) são projetados a partir da utilização de serviços cloud (AWS), que oferecem um modelo de pagamento por uso, minimizando o investimento inicial:
\begin{itemize}
    
\item Custo Mensal Estimado (AWS + Domínio): R\$ 400,00.
\item Custo Acumulado (2,5 meses de Desenvolvimento): R\$ 1.000,00.

\end{itemize}
Investimento Inicial Necessário
O investimento inicial total para o desenvolvimento e entrega do Projeto Final é de R\$ 51.000,00.
% ---
\section{Receitas}
% ---
CarneUp é baseado em Software as a Service (SaaS), focado em receita recorrente e escalabilidade\cite{aws_saas}.
\begin{itemize}
    \item Modelo: Assinatura mensal por açougue.
    \item Preço por Assinatura: R\$ 250,00 mensais por unidade (açougue).
\end{itemize}
A receita é escalável e diretamente ligada à aquisição de novos clientes após a Entrega Final (Mês 3 em diante).
% ---
\section{Cenário Realista}
% ---
Projeta uma taxa de adoção prudente, levando em consideração os desafios de implementação de novos sistemas em pequenos e médios varejistas.
\begin{itemize}
    \item Adoção: 5 novos clientes no Mês 3 (Novembro/2025), com crescimento de 5 clientes por mês nos períodos subsequentes.
    \item Ponto de Equilíbrio (Break-even Point): O custo inicial de R\$ 51.000,00 é recuperado no Mês 7 (Março/2026).
Este é um prazo excelente, dado o alto investimento inicial em MO.
\end{itemize}


% ---
\section{Cenário Otimista}
% ---

Considera uma alta aceitação do produto no mercado, impulsionada pela facilidade de uso e pelo foco no problema de rastreabilidade do açougue.
\begin{itemize}
    \item Adoção: 10 novos clientes no Mês 3, com crescimento de 10 clientes por mês nos períodos subsequentes.
    \item Ponto de Equilíbrio (Break-even Point): Atingido no Mês 5 (Janeiro/2026).
\end{itemize}
% ---
\section{Cenário Pessimista}
% ---
Pressupõe uma adoção lenta, devido à inércia do mercado ou dificuldades na validação da proposta de valor.
\begin{itemize}
    \item Adoção: 2 novos clientes no Mês 3, com crescimento de apenas 2 clientes por mês nos períodos subsequentes.
    \item Ponto de Equilíbrio (Break-even Point): Atingido no Mês 12 (Outubro/2026).
\end{itemize}
% ---

% ----------------------------------------------------------
% CAPÍTULO 5: CONCLUSÃO (SEÇÃO 6 NO ORIGINAL)
% ----------------------------------------------------------
\chapter{Considerações Finais}\label{cap:conclusao} % <<< CORREÇÃO: Label única

O desenvolvimento do CarneUp foi concluído, resultando em uma aplicação que resolve o problema central da gestão de açougues de pequeno e médio porte: a rastreabilidade e o controle de validade do estoque de carnes.
O projeto demonstrou a viabilidade técnica da integração de tecnologias modernas para atender a um nicho de mercado específico.
% ---
\section{Dificuldades, Escolhas e Descartes}
% ---

As dificuldades enfrentadas pelo projeto não foram de ordem de integração de hardware, mas sim desafios de gestão e complexidade de dados:

\begin{itemize}
    \item Restrição Temporal e Marcos Fixos: A principal dificuldade foi a condensação do cronograma para apenas 2,5 meses, imposta pelos Marcos Fixos de Entrega Acadêmica (20/09, 10/10 e 31/10/2025).
Essa restrição exigiu um controle rigoroso do escopo.
    \item Modelagem de Diagramas e Banco de Dados (MER/DER): Encontramos problemas na fase de modelagem, pois a rastreabilidade completa da carne exige um controle de estoque complexo.
Tivemos que refinar repetidamente os diagramas (MER/DER) para garantir que o sistema rastreasse as informações essenciais.
\end{itemize}

Escolhas Cruciais

As escolhas tecnológicas e metodológicas foram determinantes para o sucesso do projeto no prazo estabelecido:
\begin{itemize}
    \item Metodologia Scrum: A adoção do Scrum permitiu a entrega incremental em Sprints curtas, garantindo que os marcos fixos fossem atingidos (Desenho, POC e MVP) através de priorização ágil.
    \item PostgreSQL: O uso do PostgreSQL foi essencial para a segurança dos dados, oferecendo a robustez e a integridade transacional exigidas para dados financeiros e de estoque.
\end{itemize}

Funcionalidades Descartadas

Para garantir a entrega do MVP dentro do prazo rigoroso e focar na funcionalidade principal do açougue, algumas funcionalidades foram deliberadamente adiadas para a Fase 2:
\begin{itemize}
    \item Módulo de Contabilidade Completa: O foco foi mantido no core business (Venda, Estoque e Custo da Peça).
Módulos avançados de geração de balancetes ou integração contábil foram adiados.
    \item Múltiplas Filiais: O sistema foi dimensionado para um único açougue para simplificar a lógica de estoque e as permissões, garantindo que o tempo fosse focado na rastreabilidade por lote.
\end{itemize}



% ----------------------------------------------------------
% Finaliza a parte no bookmark do PDF
% para que se inicie o bookmark na raiz
% e adiciona espaço de parte no Sumário
% ----------------------------------------------------------


% ----------------------------------------------------------
% ELEMENTOS PÓS-TEXTUAIS
% ----------------------------------------------------------
\postextual
% ----------------------------------------------------------

% ----------------------------------------------------------
% Referências bibliográficas
% ----------------------------------------------------------
\bibliography{Referencias}



\end{document}
