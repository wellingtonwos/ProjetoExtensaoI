
%\documentclass[
	% -- opções da classe memoir --
%	12pt,				% tamanho da fonte
%	openright,			% capítulos começam em pág ímpar (insere página vazia caso preciso)
%	twoside,			% para impressão em verso e anverso. Oposto a oneside
%	a4paper,			% tamanho do papel. 
	% -- opções do pacote babel --
%	english,			% idioma adicional para hifenização
%	french,				% idioma adicional para hifenização
%	spanish,			% idioma adicional para hifenização
%	brazil				% o último idioma é o principal do documento
%	]{abntex2}
% ---	

\documentclass[
    % -- opções da classe memoir --
    12pt,               % tamanho da fonte
    openany,            % capítulos começam em qualquer pág
    oneside,            % para impressão só frente (digital)
    a4paper,            % tamanho do papel. 
    % -- opções do pacote babel --
    english,            % idioma adicional para hifenização
    french,             % idioma adicional para hifenização
    spanish,            % idioma adicional para hifenização
    brazil              % o último idioma é o principal do documento
    ]{abntex2}

% ---
% Pacotes básicos 
% ---
\usepackage{lmodern}			% Usa a fonte Latin Modern			
\usepackage[T1]{fontenc}		% Selecao de codigos de fonte.
\usepackage[utf8]{inputenc}		% Codificacao do documento (conversão automática dos acentos)
\usepackage{lastpage}			% Usado pela Ficha catalográfica
\usepackage{indentfirst}		% Indenta o primeiro parágrafo de cada seção.   
\usepackage{xcolor}             % Controle das cores (mais moderno)
\usepackage{graphicx}			% Inclusão de gráficos
\usepackage{microtype} 			% para melhorias de justificação
\usepackage{longtable} 			% Adicionado para tabelas longas
\usepackage{tabularx} 			% Adicionado para tabelas que seajustam
\usepackage{xurl} 				% Adicionado para quebrar URLs
% ---
		
% ---
% Pacotes de citações
% ---
\usepackage[brazilian,hyperpageref]{backref}	 % Paginas com as citações na bibl
\usepackage[alf]{abntex2cite}	% Citações padrão ABNT
\usepackage{ragged2e}

% --- 
% CONFIGURAÇÕES DE PACOTES
% --- 

% ---
% Configurações do pacote backref
% Usado sem a opção hyperpageref de backref
\renewcommand{\backrefpagesname}{Citado na(s) página(s):~}
% Texto padrão antes do número das páginas
\renewcommand{\backref}{}
% Define os textos da citação
\renewcommand*{\backrefalt}[4]{
	\ifcase #1 %
		Nenhuma citação no texto.%
	\or
		Citado na página #2.%
	\else
		Citado #1 vezes nas páginas #2.%
	\fi}%
	
% Pacotes para tabelas (colunas de largura fixa e alinhamento)
\usepackage{array}
\usepackage{booktabs}
\usepackage{ragged2e} % Para alinhar texto justificado/à esquerda em colunas p{}
\usepackage{longtable} % Essencial para tabelas longas
\usepackage{hyperref}  % Para links (sempre o último a ser carregado)
	
	
% ---

% ---
% Informações de dados para CAPA e FOLHA DE ROSTO
% ---
\titulo{CarneUp: Sistema de Gestão de Açougue de Pequeno e Médio Porte}


\autor{BRENO DIAS OLIVEIRA SP3015645\\FELIPE DALBOSCO 
PALUDO SP3123766\\GUSTAVO GOUVEA ANDRADE SP3076725\\PEDRO AUGUSTO SILVA SANTOS SP3098559\\RAQUEL CORREIA DA SILVA 
SP3098567\\WELLINGTON OLIVEIRA DE SOUSA SP3161307}

\local{São Paulo - SP - Brasil}
\data{2025}
\orientador{Marcelo Tavares de Santana}

\instituicao{%
   IFSP- Instituto Federal de Educação, Ciência e Tecnologia
 Câmpus São Paulo
  \par
   Tecnologia em Análise e Desenvolvimento de Sistemas}

% O preambulo deve conter o tipo do trabalho, o objetivo, 
% o nome da instituição e a área de concentração 
\preambulo{Trabalho de Projeto Integrado I apresentado ao curso de Tecnologia em Análise e Desenvolvimento de Sistemas do Instituto Federal de Educação, Ciência e Tecnologia de São Paulo, como parte dos requisitos para a conclusão da disciplina.}
% ---


% ---
% Configurações de aparência do PDF final
% ---

% alterando o aspecto da cor azul
\definecolor{blue}{RGB}{41,5,195}

% informações do PDF
\makeatletter
\hypersetup{
     	%pagebackref=true,
		pdftitle={\@title}, 
		pdfauthor={\@author},
    	pdfsubject={\imprimirpreambulo},
	    pdfcreator={LaTeX with abnTeX2},
		pdfkeywords={abnt}{latex}{abntex}{abntex2}{trabalho acadêmico}, 
		colorlinks=true,       		% false: boxed links; true: colored links
    	linkcolor=blue,          	% color of internal links
    	citecolor=blue,        		% color of links to bibliography
    	filecolor=magenta,      		% color of file links
		urlcolor=blue,
		bookmarksdepth=4
}
\makeatother
% --- 

% --- 
% Espaçamentos entre linhas e parágrafos 
% --- 

% O tamanho do parágrafo é dado por:
\setlength{\parindent}{1.3cm}

% Controle do espaçamento entre um parágrafo e outro:
\setlength{\parskip}{0.2cm}  % tente também \onelineskip

% ---
% compila o indice
% ---
\makeindex
% ---

% ----
% Início do documento
% ----
\begin{document}

% Retira espaço extra obsoleto entre as frases.
\frenchspacing 

% ----------------------------------------------------------
% ELEMENTOS PRÉ-TEXTUAIS
% ----------------------------------------------------------
% \pretextual

% ---
% Capa
% ---
\imprimircapa
% ---

% ---
% Folha de rosto
% (o * indica que haverá a ficha bibliográfica)
% ---
\imprimirfolhaderosto*
% ---

% ---
% RESUMOS
% ---

% resumo em português
% \setlength{\absparsep}{18pt} % <<< CORREÇÃO: Comando comentado, pois é desnecessário com parágrafo único
\begin{resumo}
Este trabalho propõe o desenvolvimento do CarneUp, um sistema de gerenciamento de açougues focado em digitalizar e otimizar as operações de venda, compra e controle de estoque de carnes[cite: 13]. A ausência de controle digital em muitas boutiques e açougues de pequeno e médio porte, onde as vendas e o gerenciamento de estoque ainda são feitos de forma analógica em cadernos, gera ineficiência e perda de informações cruciais[cite: 14]. O CarneUp visa solucionar este problema, oferecendo uma plataforma robusta e intuitiva para o gerenciamento completo do negócio, desde o registro de vendas com captura automática de peso/valor via código de barras, até o controle de validade e a rastreabilidade por fornecedor[cite: 15]. O sistema é projetado para açougues de pequeno e médio porte, fornecendo ferramentas essenciais para aumentar a precisão do estoque, melhorar o controle financeiro por peça e reduzir o desperdício, garantindo assim a eficiência e a competitividade do negócio[cite: 16].

\textbf{Palavras-chaves}: Gestão de açougue, Controle de estoque de carnes, Sistema de vendas, Software de açougue[cite: 17].
\end{resumo}

% resumo em inglês
\begin{resumo}[Abstract]
 \begin{otherlanguage*}{english}
   This work proposes the development of CarneUp, a butcher shop management system focused on digitizing and optimizing sales, purchasing, and meat stock control operations[cite: 19]. The lack of digital control in many small and medium-sized butcher shops, where sales and inventory management are still carried out analogically in notebooks, leads to inefficiency and loss of crucial information[cite: 20]. CarneUp aims to solve this problem by offering a robust and intuitive platform for complete business management, from sales registration with automatic weight/value capture via barcode, to validity control and traceability by supplier[cite: 21]. The system is designed for small and medium-sized butcher shops, providing essential tools to increase inventory accuracy, improve financial control per piece, and reduce waste, thereby ensuring business efficiency and competitiveness[cite: 22].
 
   \vspace{\onelineskip}
 
   \noindent 
   \textbf{Key-words}: Butcher shop management, Meat inventory control, Sales system, Butcher shop software[cite: 23].
 \end{otherlanguage*}
\end{resumo}

% ---
% inserir lista de ilustrações
% ---
\pdfbookmark[0]{\listfigurename}{lof}
\listoffigures*
\cleardoublepage
% ---

% ---
% inserir lista de tabelas
% ---
\pdfbookmark[0]{\listtablename}{lot}
\listoftables*
\cleardoublepage
% ---

% ---
% inserir lista de abreviaturas e siglas
% ---
\begin{siglas}
  \item[API] Application Programming Interface
  \item[AWS] Amazon Web Services
  \item[CRUD] Create, Read, Update, Delete
  \item[FDD] Feature Driven Development
  \item[MER] Modelo Entidade-Relacionamento
  \item[MVP] Produto Mínimo Viável
  \item[PDV] Ponto de Venda
  \item[POC] Prova de Conceito
  \item[RF] Requisito Funcional
  \item[RNF] Requisito Não Funcional
  \item[SaaS] Software as a Service
  \item[Scrum] (Metodologia Ágil)
  \item[SSL/TLS] Secure Sockets Layer/Transport Layer Security
\end{siglas}
% ---

% ---
% inserir o sumario
% ---
\pdfbookmark[0]{\contentsname}{toc}
\tableofcontents*
\cleardoublepage
% ---

% ----------------------------------------------------------
% ELEMENTOS TEXTUAIS
% ----------------------------------------------------------
\textual

% ----------------------------------------------------------
% CAPÍTULO 1: INTRODUÇÃO
% ----------------------------------------------------------
\chapter*[Introdução]{Introdução}

O principal objetivo do projeto \textit{CarneUp} é desenvolver um sistema de gerenciamento digital focado nas operações de açougues de pequeno e médio porte. Este sistema visa permitir a digitalização e a otimização da gestão de estoque e do processo de vendas no Ponto de Venda (PDV).

O problema central que motiva este projeto é a gestão predominantemente analógica e ineficiente encontrada em muitos desses estabelecimentos. Atualmente, operações críticas como o controle de vendas, o registro de compras e o acompanhamento de estoque são realizados manualmente em cadernos ou planilhas. Esta abordagem é não apenas propensa a erros, mas também impede uma análise precisa da margem de lucro por corte e um controle detalhado por fornecedor ou validade. Consequentemente, a falta de um sistema digital resulta em desperdício de estoque vencido e na perda de dados valiosos para a tomada de decisão estratégica.

Diante desse cenário, a relevância do \textit{CarneUp} reside na necessidade urgente de modernização deste setor. A solução proposta visa digitalizar e centralizar a gestão do açougue, oferecendo uma aplicação web robusta e com foco na facilidade de uso para o público-alvo. Ao automatizar o registro de vendas via código de barras e implementar um controle de validade e de custos por fornecedor, o sistema impacta diretamente a eficiência operacional, o controle financeiro e a redução de desperdício.

\chapter {Escopo e Objetivos}

O Escopo do projeto abrange a criação de um sistema web focado nas operações de retaguarda (back-office) e no ponto de venda (PDV) de açougues que vendem carnes e outros produtos. O sistema tem como foco principal o controle de estoque de itens perecíveis e não perecíveis e o controle financeiro detalhado de cada peça de carne.

\textbf{O projeto abrange:}
\begin{itemize}
    \item Gestão de Vendas (PDV) com integração de código de barras. É possível realizar vendas mesmo sem o código de barras, inserindo o item manualmente.
    \item Controle de Estoque (Entrada/Saída), incluindo marca, valor de custo e validade (para perecíveis).
    \item Emissão de alertas de validade.
    \item Geração de relatórios básicos de vendas, com filtros por período.
    \item Gestão de usuários: Perfis de Administrador e Operador.
    \item Aplicação de padrões de acessibilidade (WCAG) na interface web.
\end{itemize}

Os objetivos específicos do projeto incluem:

\begin{itemize}
    \item Gerenciar Vendas: Permitir o registro rápido e eficiente de vendas de carnes e outros produtos.
    \item Controle de Estoque: Fornecer uma visão em tempo real do estoque, incluindo o controle das peças vendidas, suas validades (quando aplicável) e marcas.
    \item Digitalização de Pesagem: Integrar-se com códigos de barras gerados pela balança para capturar automaticamente o valor e o peso da peça no momento da venda.
    \item Otimização de Compras: Registrar o valor pago nos itens e suas marcas para análises de custo.
\end{itemize}

\section{Requisitos do Sistema}

Para alcançar os objetivos propostos, o \textit{CarneUp} atende a um conjunto de requisitos funcionais (RF) e não funcionais (RNF e RN). Os requisitos funcionais estão detalhados na Tabela 1, agrupados de acordo com os principais componentes de gestão definidos nos diagramas de arquitetura: \textbf{Gestão de Vendas}, \textbf{Gestão de Produtos (Estoque)}, \textbf{Gestão de Compras/Marca} e \textbf{Gestão de Autenticação}.

\begin{longtable}{|>{\Centering}m{1.5cm}|>{\RaggedRight}m{13cm}|}
    \caption{Requisitos Funcionais (RF)}
    \label{tab:requisitos_funcionais}\\
    \hline
    \textbf{Código} & \textbf{Descrição} \\ 
    \hline
    \endfirsthead

    \multicolumn{2}{c}%
    {{\bfseries \tablename\ \thetable{} -- continuação}} \\
    \hline
    \textbf{Código} & \textbf{Descrição} \\ 
    \hline
    \endhead
    \hline
    \multicolumn{2}{|r|}{{Continua...}} \\
    \hline
    \endfoot
    \hline
    \endlastfoot

    RF01 & O sistema deve permitir o registro de vendas através da leitura do código de barras gerado pela balança ou pela inserção manual do item. \\ 
    \hline
    RF02 & O sistema deve calcular automaticamente o valor total da venda ao capturar o peso e o valor unitário do item lido via código de barras ou inserido manualmente. \\ 
    \hline
    RF03 & O sistema deve permitir o registro de novas entradas de estoque (compras), incluindo valor de custo, marca e validade (para perecíveis). \\  
    \hline
    RF04 & O sistema deve calcular a margem de lucro de cada peça de carne/item vendido (Valor de Venda - Valor de Custo). \\
    \hline
    RF05 & O sistema deve emitir um alerta visual no painel para o administrador quando a validade de uma peça de estoque estiver a 7 dias ou menos do vencimento. \\ 
    \hline
    RF06 & O sistema deve permitir o descarte de peças de estoque (baixa) em caso de vencimento ou perda de qualidade. \\ 
    \hline
    RF07 & O sistema deve permitir a busca e visualização de peças de estoque por marca. \\ 
    \hline
    RF08 & O sistema deve gerenciar o cadastro completo de marcas. \\
    \hline
    RF09 & O sistema deve permitir a visualização de relatórios de vendas, com filtros por período e itens. \\
    \hline
    RF10 & O sistema deve controlar o acesso através de autenticação de usuário (login/senha) e gerenciar o ciclo de vida da sessão. \\ 
    \hline
    RF11 & O sistema deve permitir o cadastro de diferentes perfis de usuários: \textbf{Administrador} e \textbf{Operador}. \\
    \hline
\end{longtable}

\subsection{Regras de Negócio}

As regras de negócio (RN) representam as restrições, políticas e condições operacionais que ditam o funcionamento interno do sistema e garantem sua aderência aos processos específicos de um açougue.

A Tabela \ref{tab:requisitos_rn} detalha as regras obrigatórias do sistema \textit{CarneUp}.

\begin{longtable}{|>{\Centering}m{1.5cm}|>{\RaggedRight}m{9cm}|>{\Centering}m{4cm}|}
\caption{Regras de Negócio (RN)}
\label{tab:requisitos_rn}\\ % <-- Rótulo corrigido para ser o único usado
    \hline
    \textbf{Código} & \textbf{Descrição} & \textbf{Requisito Relacionado} \\
    \hline
    \endfirsthead

    \multicolumn{3}{c}%
    {{\bfseries \tablename\ \thetable{} -- continuação}} \\
    \hline
    \textbf{Código} & \textbf{Descrição} & \textbf{Requisito Relacionado} \\
    \hline
    \endhead
    \hline
    \multicolumn{3}{|r|}{{Continua...}} \\
    \hline
    \endfoot
    \hline
    \endlastfoot

    RN01 & A venda pode ser registrada pela leitura do código de barras ou pela inserção manual do item, mas só será concluída se o item tiver estoque disponível. & RF01, RF02 \\
    \hline
    RN02 & O valor de custo por peça/item deve ser registrado no momento da entrada no estoque para cálculo de margem de lucro. & RF03, RF04 \\
    \hline
    RN03 & Peças/itens com validade expirada devem ser automaticamente bloqueados para venda e movidos para a lista de descarte. & RF06, RNF01 \\
    \hline
    RN04 & O sistema deve manter o histórico de marcas para cada peça de carne/item no estoque para fins de análise de custo e consulta. & RF07 \\
    \hline
\end{longtable}

Os Requisitos Não Funcionais (RNF) especificam os critérios que podem ser usados para julgar a operação de um sistema, em vez de comportamentos específicos.

\begin{longtable}{|>{\Centering}m{1.5cm}|>{\Centering}m{3cm}|>{\RaggedRight}m{10cm}|}
    \caption{Requisitos Não Funcionais (RNF)}
    \label{tab:requisitos_rnf}\\
    \hline
\textbf{Código} & \textbf{Módulo} & \textbf{Descrição} \\
\hline
    \endfirsthead

    \multicolumn{3}{c}%
    {{\bfseries \tablename\ \thetable{} -- continuação}} \\
    \hline
\textbf{Código} & \textbf{Módulo} & \textbf{Descrição} \\
    \hline
    \endhead
    \hline
    \multicolumn{3}{|r|}{{Continua...}} \\
    \hline
    \endfoot
    \hline
    \endlastfoot

    RNF01 & Performance & O registro de venda e a baixa de estoque devem ser processados em no máximo 2 segundos para visar o fluxo rápido de caixa. \\
    \hline
    RNF02 & Escalabilidade & O sistema deve ser capaz de suportar picos de até 10 usuários simultâneos (Operadores + Administradores) sem degradação perceptível de performance. \\
    \hline
    RNF03 & Usabilidade & A interface de registro de vendas deve ser intuitiva e otimizada para telas de toque e leitores de código de barras. \\
    \hline
    RNF04 & Segurança & Todos os dados sensíveis (informações financeiras, cadastros) devem ser armazenados de forma segura (ex: senhas via bcrypt, dados via SSL/TLS). \\
    \hline
    RNF05 & Manutenibilidade & O código deve seguir uma convenção de código (\textit{Code Convention}) para facilitar a identificação e correção de erros. \\
    \hline
\end{longtable}

\section{Problema e Solução Proposta}

O problema central abordado pelo \textit{CarneUp} é a gestão predominantemente analógica e ineficiente encontrada em açougues de pequeno e médio porte. Nestes estabelecimentos, a falta de controle de estoque detalhado (por marca, validade) e a dificuldade na mensuração da margem de lucro por item resultam em desperdício de estoque e perda de dados valiosos.

Para endereçar essa lacuna, a solução proposta é o \textit{CarneUp}: uma aplicação web desenhada para digitalizar e centralizar a gestão de estoque e vendas de todos os produtos do açougue.

As funcionalidades chave do \textit{CarneUp} são:

\begin{itemize}
    \item Registro de Vendas Otimizado: Utilização de códigos de barras da balança (ou entrada manual) para preenchimento automático de valor e peso da peça vendida.
    \item Controle de Custo por Marca: Registro de dados como marca, valor de compra e validade de cada lote.
    \item Controle de Estoque Inteligente: Emissão de alertas visuais de validade próxima e visualização clara das peças em estoque.
\end{itemize}


\section{Justificativa}

A relevância do \textit{CarneUp} se justifica pela necessidade de modernização do setor. Ao fornecer uma solução digital acessível, o \textit{CarneUp} contribui diretamente para:

\begin{itemize}
    \item Controle Financeiro: O registro do valor pago por peça/item permite uma análise de custo precisa, otimizando a margem de lucro e identificando os itens mais rentáveis.
    
    \item Redução de Desperdício: O controle de validade digitalizado ajuda a reduzir as perdas de estoque devido ao vencimento.
    
    \item Eficiência Operacional: A automação do registro de vendas via código de barras acelera o atendimento e minimiza a chance de erros no caixa.
\end{itemize}

% ---
\section{Análise da Concorrência}
% ---
O CarneUp se posiciona contra sistemas de gestão (PDV) mais genéricos, adaptados para o segmento de açougues, mas que não possuem a granularidade necessária para o controle de carnes com base em peso variável, validade e marca.

\subsection{Concorrente 1: ConnectPlug}

O ConnectPlug é um sistema de PDV (Ponto de Venda) e gestão comercial mais abrangente, atendendo diversos segmentos, incluindo açougues. Ele oferece funcionalidades como frente de caixa, gestão de estoque e emissão de notas fiscais. \cite{connectplug}

\begin{itemize}
    \item Foco: Solução completa de gestão comercial.

    \item Diferencial (para o CarneUp): Pode ser adaptado para açougues, mas pode não ter o foco e a profundidade no **controle de custos por peça** (valor de compra, marca, validade individualizada) que o CarneUp busca.
\end{itemize}

\subsection{Concorrente 2: Aliar Sistemas}

A Aliar Sistemas oferece soluções específicas para o varejo de carnes, com foco em açougues e frigoríficos. Seu sistema inclui gestão de produção, **controle de lotes** e controle de estoque. \cite{aliar}

\begin{itemize}
    \item Foco: Soluções específicas para o setor de carnes.

    \item Por focar em frigoríficos e grandes açougues, pode ter um custo elevado e complexidade excessiva para açougues de pequeno porte, que é o alvo do CarneUp.
\end{itemize}

\subsection{Concorrente 3: SOFTClass}

A SOFTClass oferece software de gestão para diversos segmentos do varejo, incluindo açougues. Possui módulos de frente de caixa, estoque e financeiro.\cite{softclass}

\begin{itemize}
    \item Foco: Solução de gestão empresarial para varejo.

    \item Semelhante ao ConnectPlug, pode ser mais um sistema de PDV adaptado, carecendo da simplicidade e do foco no ciclo de vida da carne (compra/validade/marca) que é o diferencial do CarneUp.
\end{itemize}


\subsection{Comparativo}
O quadro abaixo resume as principais funcionalidades e destaca o diferencial do CarneUp.

\begin{table}[h]
    \centering
    \caption{Comparativo}
    \label{tab:placeholder_label}
    \begin{tabularx}{\linewidth}{|>{\RaggedRight}p{4cm}|c|c|c|c|}
        \hline
        \textbf{Funcionalidade} & \textbf{CarneUp} & \textbf{ConnectPlug} & \textbf{Aliar Sistemas} & \textbf{SOFTClass} \\ \hline
        Registro de Vendas (Manual ou Cód. Barras) & X & X & X & X \\ \hline
        Controle Básico de Estoque & X & X & X & X \\ \hline
        Captura de Valor/Peso por Código de Barras da Balança & X & X & X & X \\ \hline
        \textbf{Controle Detalhado (Marca e Custo por Peça)} & X & X & - & - \\ \hline
        Alertas de Validade por Peça/Lote & X & - & - & - \\ \hline
        Foco em Pequenos/Médios Açougues e Facilidade de Uso & X & - & - & X \\ \hline
    \end{tabularx}
\end{table}

% ----------------------------------------------------------
% CAPÍTULO 2: GESTÃO DO PROJETO (SEÇÃO 3 NO ORIGINAL)
% * <Felipe Paludo> 23:07:07 10 Nov 2025 UTC-0300:
% REFEITO
% ----------------------------------------------------------
\chapter{Gestão do Projeto}\label{cap_exemplos}
% ACRESCENTADO PARAGRAFO INTRODUTORIO NO CAPITULO 2
A presente seção detalha o planejamento, a organização e o controle das atividades essenciais para a execução do projeto. Este capítulo estabelece a estrutura da equipe, define os papéis e responsabilidades dos membros, e descreve a metodologia ágil adotada, incluindo a estruturação do cronograma e os ciclos de desenvolvimento. Por fim, são apresentadas as ferramentas de versionamento e o acesso ao código-fonte.


% ---
\section{Organização da Equipe}
% ---

O projeto será desenvolvido por uma equipe de cinco membros, com papéis definidos para cobrir as áreas de desenvolvimento, administração de dados e gestão, conforme apresentado na Tabela 2.

% ---
%\subsection{Responsabilidades / Papéis / Atividades}
% ---

\begin{table}[h!]
    \centering
    \caption{Papéis, Integrantes e Responsabilidades Chave}
    \label{tab:papeis_equipe}
    
    \begin{tabular}{|>{\Centering}m{2.5cm}|>{\Centering}m{3.5cm}|>{\RaggedRight}m{7cm}|}
        \hline
        \textbf{Papel} & \textbf{Integrante} & \textbf{Responsabilidades Chave} \\ 
        \hline
        Product Owner & Raquel Correia da Silva & Definição e priorização do Product Backlog, contato com o cliente, validação das entregas. \\ 
        \hline
        Scrum Master & Felipe Dalbosco Paludo & Visar a aplicação correta da metodologia Scrum, remover impedimentos, facilitar as reuniões. \\ 
        \hline
        Desenvolvedor Full Stack & Gustavo Gouvea Andrade & Implementação das funcionalidades no front-end e back-end, integração com banco de dados. \\ 
        \hline
        Desenvolvedor Front-end & Pedro Augusto Silva Santos & Desenvolvimento da interface do usuário, usabilidade e responsividade. \\ % CORRIGIDO
        \hline
        DBA / Desenvolvedor Back-end & Breno Dias Oliveira & Modelagem e Administração do Banco de Dados, desenvolvimento de APIs e lógica de negócios. \\ % CORRIGIDO
        \hline
        DBA / Desenvolvedor Back-end & Wellington Oliveira de Sousa & Modelagem e Administração do Banco de Dados, desenvolvimento da documentação LateX. \\ 
        \hline
    \end{tabular}
\end{table}

% ---
\section{Metodologias de gestão e desenvolvimento}
% ---
O projeto foi planejado para ter uma duração total de aproximadamente 2,5 meses, iniciando em 6 de Setembro de 2025 e com previsão de conclusão em 31 de Outubro de 2025. Essa duração é ditada pelos Marcos Fixos de Entrega.

O desenvolvimento do CarneUp adota a metodologia ágil Scrum, organizada em ciclos de trabalho curtos chamados Sprints. Dada a restrição temporal imposta pelos marcos de entrega acadêmica (Projeto Integrado - PI), o projeto foi dividido em 4 Sprints principais com duração variável (aproximadamente 10 a 14 dias úteis) para focar na entrega dos marcos fixos. O foco não é no número de Sprints, mas sim no cumprimento dos marcos de entrega, apresentado na tabela 3, que guiam o ritmo do projeto.

\begin{table}[h!]
    \centering
    \caption{Marcos de Entrega do Projeto \textit{CarneUp}}
    \label{tab:marcos_entrega}
    \begin{tabular}{|>{\Centering}m{2.5cm}|>{\RaggedRight}m{11cm}|}
        \hline
        \textbf{Data} & \textbf{Marco de Entrega / Atividade} \\ 
        \hline
        \textbf{06/09/2025} & Início do Desenvolvimento Geral e da Documentação. \\ 
        \hline
        \textbf{20/09/2025} & Entrega do Desenho da Aplicação. \\ 
        \hline
        \textbf{10/10/2025} & Apresentação da Prova de Conceito (POC). \\ 
        \hline
        \textbf{31/10/2025} & Entrega do Projeto Final (MVP) e sua Documentação. \\ 
        \hline
    \end{tabular}
\end{table}

% ---
\section{Repositório da aplicação}
% ---

O projeto utilizará o GitHub como plataforma de versionamento de código e colaboração. O GitHub permite o controle detalhado das alterações de código (commits), facilitando o trabalho em equipe, a revisão de código (Pull Requests) e o rastreamento de problemas (Issues).

Link: https://github.com/wellingtonwos/ProjetoExtensaoI
Acesso: O repositório será privado durante o desenvolvimento e pode ser tornado público (ou manter-se privado com acesso via convite) após a Entrega Final. Para fins de avaliação, os colaboradores deverão ser convidados a ter acesso "Read" ou "Collaborator".
% ----------------------------------------------------------
% CAPÍTULO 3: DESENVOLVIMENTO (SEÇÃO 4 NO ORIGINAL)
% ----------------------------------------------------------
\chapter{Desenvolvimento Do Projeto}\label{cap:desenvolvimento}

O projeto \textit{CarneUp} abrange o desenvolvimento de uma aplicação web completa, projetada especificamente para digitalizar e otimizar a gestão de açougues de pequeno e médio porte, com foco nas operações de Vendas, Compras e Estoque.

\section{Arquitetura e Modularidade do Sistema}

O \textit{CarneUp} foi concebido sob uma arquitetura de aplicação web em três camadas (Front-end, Back-end e Banco de Dados), visando clareza na separação de responsabilidades, manutenibilidade e escalabilidade (RNF02).

\subsection{Módulos de Negócio}

O Back-end é o core transacional do sistema e está logicamente dividido em módulos que se comunicam via API RESTful:

\begin{itemize}
    \item \textbf{Módulo de Gestão de Vendas (PDV):} Focado na eficiência do caixa (RNF01), gerencia o registro rápido de vendas. Sua principal característica é a integração com a leitura de código de barras da balança (ou inserção manual do item), permitindo a captura automática do item, peso e valor, e garantindo a baixa de estoque em tempo real (RF01, RF02).
    \item \textbf{Módulo de Estoque e Rastreabilidade:} Permite o registro da entrada de estoque com informações cruciais para a análise de custo e rastreabilidade: valor de custo, marca e data de validade (RF03). O módulo é responsável pela emissão de alertas visuais quando a validade das peças está próxima (RF05).
    \item \textbf{Módulo de Gestão de Marcas e Compras:} Gerencia o cadastro completo das marcas (RF08) e mantém o histórico de compra de cada lote, permitindo o cálculo da margem de lucro por peça (RF04).
    \item \textbf{Módulo de Autenticação e Usuários:} Controla o acesso ao sistema via login/senha e implementa o gerenciamento dos perfis \textbf{Administrador} e \textbf{Operador}, garantindo que apenas usuários autorizados tenham acesso às funcionalidades específicas (RF10, RF11).
\end{itemize}

\section{Escolhas Tecnológicas (Tech Stack)}

A seleção da pilha tecnológica foi baseada na busca por performance, segurança e robustez.

\begin{itemize}
    \item \textbf{Front-end (Apresentação):} Foi escolhido o \textbf{React} para a construção da interface. Esta biblioteca permite a criação de uma experiência de usuário reativa, modular e otimizada para telas de toque e leitores de código de barras, o que atende ao requisito de usabilidade no PDV (RNF03).
    \item \textbf{Back-end (Aplicação):} A lógica de negócios é implementada em \textbf{Java com o framework Spring Boot}. Esta combinação é ideal para sistemas transacionais de missão crítica, oferecendo alta segurança e performance na execução das regras de negócio (RNF01).
    \item \textbf{Banco de Dados (Dados):} O \textbf{PostgreSQL} foi definido como o Sistema Gerenciador de Banco de Dados (SGBD) principal. Sua reputação de robustez, integridade transacional e confiabilidade é fundamental para armazenar dados financeiros e de estoque de forma segura e duradoura (RNF04).
\end{itemize}

\section{Infraestrutura e Segurança na Nuvem}

O projeto será implantado na \textbf{Amazon Web Services (AWS)}, que oferece o modelo de Software as a Service (SaaS) e garante a infraestrutura necessária para suportar picos de uso e a expansão futura.

\begin{itemize}
    \item \textbf{Hospedagem e Escalabilidade:} A plataforma AWS garante que o sistema possa suportar picos de até 10 usuários simultâneos (RNF02) sem degradação de performance, por meio de serviços de computação elástica.
    \item \textbf{Segurança de Dados em Trânsito:} A comunicação entre o Front-end e o Back-end será totalmente blindada pelo protocolo \textbf{HTTPS}, com terminação SSL/TLS configurada no Load Balancer da AWS. Isso é mandatório para proteger credenciais de acesso e dados financeiros (RNF04).
    \item \textbf{Manutenibilidade:} A utilização de serviços de logging nativos do Spring Boot (SLF4J/Logback) será implementada para rastrear transações e facilitar a identificação e correção de erros, conforme o requisito de manutenibilidade (RNF05).
\end{itemize}

% ---

\section{Histórias de Usuário}

As Histórias de Usuário a seguir detalham o escopo do Mínimo Produto Viável (MVP), garantindo que as entregas agreguem o máximo valor aos usuários em cada ciclo de desenvolvimento.

\subsection{Descrição das Histórias de Usuário}
% ---
\begin{itemize}
    \item US01: Registrar Venda com ou sem Código de Barras (RF01, RF02)
        \begin{itemize}
            \item Descrição: Como um operador de caixa, quero registrar uma venda lendo o código de barras da balança (ou inserindo o item manualmente), para que o sistema preencha automaticamente o item, peso e valor, agilizando o atendimento.
            \item Critérios de Aceitação: O código de barras deve ser lido pelo leitor, ou o item deve ser selecionado manualmente. O sistema deve exibir o item e valor unitário. O operador deve confirmar a venda e o estoque deve ser atualizado.
        \end{itemize}
\end{itemize}

\begin{itemize}
    \item US02: Inclusão de Estoque com Rastreabilidade (RF03, RF07)
        \begin{itemize}
            \item Descrição: Como um administrador, quero registrar a entrada de novas peças/itens, incluindo a marca, valor de custo e data de validade (para perecíveis), para que eu possa ter controle de rastreabilidade e calcular a margem de lucro.
            \item Critérios de Aceitação: Deve haver um formulário com campos obrigatórios para Marca, Valor de Custo. O campo Data de Validade deve ser obrigatório para itens perecíveis. O registro deve ser armazenado no banco de dados com todas as informações.
        \end{itemize}
\end{itemize}

\begin{itemize}
    \item US03: Alerta Visual de Validade Próxima (RF05)
        \begin{itemize}
            \item Descrição: Como um administrador, quero receber um alerta visual no dashboard quando um item estiver a 7 dias do vencimento, para que eu possa tomar medidas (promoção/descarte) antes da perda total do produto.
            \item Critérios de Aceitação: O alerta deve ser exibido no dashboard principal com destaque visual.
        \end{itemize}
\end{itemize}

\begin{itemize}
    \item US04: Visualização de Margem de Lucro (RF04)
        \begin{itemize}
            \item Descrição: Como um administrador, quero visualizar a margem de lucro por peça de carne/item vendido, para identificar quais itens são mais rentáveis.
            \item Critérios de Aceitação: Deve haver uma tela de relatório que liste as vendas e mostre a diferença (em R\$ e \% ) entre o valor de venda e o valor de custo.
        \end{itemize}
        
\end{itemize}

\begin{itemize}
    \item US05: Gerenciamento de Descarte (RF06)
        \begin{itemize}
            \item Descrição: Como um administrador, quero registrar o descarte de uma peça/item (por vencimento ou estrago), para que o estoque seja baixado.
            \item Critérios de Aceitação: A baixa de estoque deve ocorrer. O sistema deve registrar a data do descarte.
        \end{itemize}
\end{itemize}

% ---
\section{Segurança, Privacidade e Legislação}

A segurança e a privacidade dos dados representam um pilar não funcional crítico para o sucesso e a longevidade do sistema \textit{CarneUp}.

\subsection{Critérios de Segurança e Privacidade}

A estratégia de segurança do \textit{CarneUp} está fundamentada em dois pilares essenciais: proteção dos dados em repouso (armazenados) e proteção dos dados em trânsito (durante a comunicação). O Requisito Não Funcional RNF04 exige o uso de técnicas de segurança para todos os dados sensíveis.

\subsubsection{Proteção de Dados em Repouso: Hashing de Senhas}

Para garantir a confidencialidade das credenciais de acesso, o sistema adotará a seguinte medida para dados armazenados:
\begin{itemize}
    \item \textbf{Hashing de Senhas:} Utilização do algoritmo de hash \textbf{bcrypt} para armazenar as senhas dos usuários (\textbf{Administradores} e \textbf{Operadores}) no banco de dados. Esta técnica de \textit{hashing} salgado impede a recuperação da senha original mesmo em caso de comprometimento da base de dados (RNF04).
\end{itemize}

\subsubsection{Criptografia em Trânsito: SSL/TLS (HTTPS)}

O protocolo HTTPS (\textit{HyperText Transfer Protocol Secure}) é essencial para o projeto \textit{CarneUp}, pois garante a segurança das comunicações entre o navegador do usuário e o servidor de aplicação. O HTTPS é o protocolo HTTP combinado com a camada de segurança SSL/TLS, criando um link criptografado.

\paragraph{Critérios de Segurança e Conformidade}
A implementação do SSL/TLS no \textit{CarneUp} atende a critérios cruciais para um sistema de gestão:
\begin{itemize}
    \item \textbf{Proteção de Credenciais:} As senhas são protegidas durante o trânsito da máquina do cliente até o servidor.
    \item \textbf{Integridade dos Dados de Venda:} Garante que os dados financeiros críticos (valor final da venda, custo do produto ou atualização de estoque) não possam ser modificados por terceiros maliciosos durante a transmissão.
\end{itemize}

\paragraph{Implantação na Infraestrutura AWS}
Na arquitetura do \textit{CarneUp}, a proteção HTTPS será implementada utilizando os serviços nativos da Amazon Web Services (AWS):
\begin{itemize}
    \item \textbf{Certificado SSL/TLS:} Será utilizado um certificado SSL/TLS para autenticar a identidade do domínio.
    \item \textbf{Terminação SSL (\textit{Offloading}):} A terminação SSL será configurada no Application Load Balancer (ALB), o ponto de entrada do tráfego. Isso permite que a carga de criptografia/descriptografia seja removida dos servidores de \textit{Back-end} (Spring Boot), melhorando o desempenho da aplicação (RNF01).
\end{itemize}
Em suma, a obrigatoriedade do protocolo HTTPS é uma medida fundamental que blinda a aplicação contra as vulnerabilidades mais comuns da web, protegendo informações sensíveis em conformidade com as melhores práticas de segurança da informação (RNF04).
    

\subsection{Observância à Legislação (LGPD)}

Embora o sistema \textit{CarneUp} trate primariamente de dados do negócio, ele processa dados pessoais de seus usuários internos (administradores e operadores). Dessa forma, o desenvolvimento adere integralmente aos princípios e diretrizes da \textbf{Lei Geral de Proteção de Dados (LGPD) – Lei nº 13.709/2018}.

\paragraph{Princípios de Proteção e Transparência}
O projeto materializa a conformidade com a LGPD nos seguintes pontos de design:

\begin{itemize}
    \item \textbf{Princípio da Finalidade e Necessidade:} A coleta de dados é limitada estritamente ao necessário para a autenticação e gestão de permissões do sistema (nome, e-mail e credenciais de login), aplicando o conceito de \textbf{minimização de dados}.
    
    \item \textbf{Segurança da Informação (\textit{Privacy by Design}):} Todas as medidas técnicas de segurança, como o \textbf{hashing em repouso (bcrypt)} e a \textbf{criptografia em trânsito (SSL/HTTPS)}, são o mecanismo técnico para cumprir o Artigo 46 da LGPD, garantindo que os dados pessoais estejam protegidos.
    
    \item \textbf{Direitos do Titular:} A arquitetura é projetada para permitir que o administrador do açougue possa, a qualquer momento, atender a solicitações dos titulares de dados, como a \textbf{exclusão de conta} de um ex-funcionário ou a \textbf{consulta} aos seus dados cadastrais.
\end{itemize}

\section{Modelo de Banco de Dados}

O modelo de banco de dados é a espinha dorsal do \textit{CarneUp}. A escolha por um modelo relacional e a tecnologia \textbf{PostgreSQL} é mandatória para suportar as transações de alta frequência no Ponto de Venda (PDV) e permitir cálculos financeiros complexos, como a margem de lucro por peça.

\subsection{Modelo Entidade Relacionamento (MER)}

O Modelo Entidade Relacionamento (MER) do \textit{CarneUp} foi concebido para atender diretamente às Regras de Negócio, centrando-se nos fluxos de Estoque, Vendas e Compras. O modelo é composto pelas seguintes entidades principais:

\begin{itemize}
    \item \textbf{Entidade Produto:} Armazena dados de itens (carne, carvão, bebidas), incluindo o nome e o código de barras padrão (GTIN).
    \item \textbf{Entidade Estoque (Lote/Item):} É a entidade central para a rastreabilidade. Armazena dados essenciais do lote em estoque, incluindo o \textit{peso/quantidade}, \textit{data de validade} (para perecíveis), \textit{valor de custo} e a chave estrangeira da Marca. Esta entidade é ligada ao Produto.
    \item \textbf{Entidade Marca:} Armazena apenas o nome da marca para rastreabilidade de custo (substitui Fornecedor).
    \item \textbf{Entidade Venda:} Representa a transação completa (data/hora e operador).
    \item \textbf{Entidade ItemVenda:} Conecta a Venda à Peça/Lote de Estoque, registrando o valor final de venda e realizando a baixa de estoque em tempo real.
    \item \textbf{Entidade Usuário:} Armazena os dados de login e o perfil de acesso (Administrador/Operador).
\end{itemize}

\subsection{Diagrama Entidade Relacionamento (DER)}

O Diagrama Entidade Relacionamento (DER) a seguir (Figura \ref{fig:der}) representa a estrutura lógica do banco de dados, ilustrando as entidades, atributos e os relacionamentos definidos no MER.

\begin{figure}[htb]
	\centering
	    \includegraphics[width=0.9\linewidth]{Imagens/DER.jpg}
         \caption{Diagrama Entidade Relacionamento (DER)}
	    \label{fig:der}
	\end{figure}

% -----------------------------------------------------------------------------
% FIGURA: Diagrama de Implantação
% -----------------------------------------------------------------------------
\subsection{Diagrama de Implantação}
O Diagrama de Implantação (Figura \ref{fig:implantacao}) ilustra a distribuição física e lógica dos componentes, detalhando a comunicação entre as camadas de apresentação, aplicação e dados.

O modelo de implantação utiliza um \textbf{Load Balancer (AWS ALB)} como ponto de entrada para distribuir o tráfego do Front-end (\textit{React}) para a camada de aplicação (\textit{Spring Boot}). Esta separação é crucial para:

\begin{itemize}
    \item \textbf{Segurança (RNF04):} O ALB é configurado para a \textbf{Terminação SSL/TLS (HTTPS)}, assegurando que todas as comunicações sejam criptografadas e protegendo o Back-end de exposição direta à internet.
    \item \textbf{Dados:} O banco de dados \textbf{PostgreSQL} é gerenciado pelo \textbf{AWS RDS}, que fornece robustez transacional, backups automáticos e segurança para os dados de estoque e financeiros.
    \item \textbf{Escalabilidade (RNF02):} A aplicação \textit{Back-end} é hospedada em instâncias elásticas (AWS EC2), permitindo o dimensionamento automático para lidar com picos de uso no Ponto de Venda (PDV).

\end{itemize}

\begin{figure}[htb]
    \centering
    \includegraphics[width=\linewidth]{Imagens/DiagramaImplantacao.jpg}
    \caption{Diagrama de Implantação do Sistema CarneUp}
    \label{fig:implantacao}
\end{figure}
% -----------------------------------------------------------------------------
% FIGURA: Diagrama de Componentes
% -----------------------------------------------------------------------------
\subsection{Diagrama de Componentes}

O Diagrama de Componentes (Figura \ref{fig:componentes}) oferece uma visão de alto nível da \textbf{estrutura interna do sistema}, ilustrando como os principais módulos lógicos interagem no *Back-end* (aplicação \textit{Spring Boot}). Ele demonstra a separação de responsabilidades (Front-end, Back-end e persistência) e o fluxo de comunicação via APIs REST, garantindo a manutenibilidade do código (RNF05).

Os componentes são segregados em três camadas principais:

\begin{itemize}
    \item \textbf{Camada de Apresentação (Front-end):} É a interface \textit{React} que interage com o usuário (Administrador/Operador). Sua função é enviar e receber dados do Back-end através de requisições HTTP (APIs).

    \item \textbf{Camada de Aplicação (Back-end/Spring Boot):} Contém os módulos centrais que implementam as Regras de Negócio (RN).
    \begin{itemize}
        \item \textbf{Módulo de Vendas (PDV):} Responsável por processar a venda (RF01), calcular o total e acionar a baixa de estoque.
        \item \textbf{Módulo de Estoque:} Gerencia a entrada, rastreabilidade e controle de validade dos itens (RF03, RF05).
        \item \textbf{Módulo de Autenticação:} Lida com o login/logout e as permissões de acesso (RF10, RF11).
    \end{itemize}

    \item \textbf{Camada de Dados (PostgreSQL):} Responsável pela persistência e integridade das informações. Todos os módulos de Back-end se comunicam com esta camada para realizar operações de CRUD (\textit{Create, Read, Update, Delete}).
\end{itemize}

\begin{figure}[htb]
    \centering
    \includegraphics[width=\linewidth]{Imagens/DiagramaComponentes.jpg}
    \caption{Diagrama de Componentes Lógicos do Sistema}
    \label{fig:componentes}
\end{figure}
% -----------------------------------------------------------------------------

% ---
% <<< CORREÇÃO: Seção 4.8 e 4.8.1 removidas por serem contraditórias e redundantes.
% A duração correta (2,5 meses) já está definida no Capítulo 3.
% ---

% ---
%\section{Dicionário de Dados}

%  \begin{figure}[htb]
%	\centering
%	    \includegraphics[width=\linewidth]{Imagens/Dicionario_de_Dados.png} % <<< CORREÇÃO: scale substituído por width=\linewidth para corrigir overflow
%        \caption{Dicionário de Dados} % <<< CORREÇÃO: Sintaxe
%	    \label{fig:dicionario_dados} % <<< CORREÇÃO: Sintaxe
%	\end{figure}


\chapter{Viabilidade Financeira}\label{cap:viabilidade}

O presente capítulo tem como objetivo delinear a viabilidade econômico-financeira do projeto \textsf{CarneUp}, focando no Investimento Inicial requerido para o desenvolvimento e entrega do Projeto Integrado (PI). Esta análise é baseada no modelo de \textbf{custo de projeto único} para o cliente-alvo (o açougue de pequeno porte).

% ---
\section{Custos}\label{sec:custos}

A estimativa de custos representa o Investimento Inicial (Capital de Giro) necessário para a fase de desenvolvimento e entrega do Produto Mínimo Viável (MVP) do \textsf{CarneUp}. Estes custos foram categorizados em duas vertentes principais: Mão de Obra (MO) e Infraestrutura, conforme detalhado nas subseções a seguir.

\subsection{Detalhamento dos Custos de Desenvolvimento (Mão de Obra)}\label{sub:custos_mo}

O custo de Mão de Obra (MO) é a componente de maior peso no investimento inicial. O cálculo baseou-se na estimativa de horas de trabalho necessárias para o período de 2,5 meses, aplicando-se uma taxa horária média ponderada.

\begin{itemize}
    \item Esforço Total Estimado: 1.000 horas (5 desenvolvedores $\times$ 80 horas/mês $\times$ 2,5 meses).
    \item Custo Horário Médio (\textit{Loaded Rate}): R\$ 50,00.
    \item Custo Total de MO: R\$ 50.000,00.
\end{itemize}

\subsection{Custos de Infraestrutura e Operacionais}\label{sub:custos_infra}

Os custos de infraestrutura consideram a utilização de serviços em nuvem (AWS) e domínios essenciais para o ambiente de desenvolvimento e testes.
% Ajuste do valor da AWS, conforme solicitado (reduzido para R$ 100/mês).

\begin{itemize}
    \item Custo Mensal Estimado (AWS + Domínio): R\$ 100,00.
    \item Custo Acumulado (2,5 meses de Desenvolvimento): R\$ 250,00.
\end{itemize}

\subsection{Investimento Inicial Consolidado}\label{sub:investimento_consolidado}

O investimento total necessário para a conclusão e entrega do MVP é consolidado na Tabela \ref{tab:custos_iniciais}.

% --- Tabela formatada em ABNT ---
\begin{table}[!htb]
    \centering
    \caption{Consolidação dos Custos Iniciais do Projeto}
    \label{tab:custos_iniciais}
    \begin{tabular}{|l|c|c|}
        \hline
        \textbf{Natureza do Custo} & \textbf{Valor (R\$)} & \textbf{Participação (\%)} \\
        \hline\hline
        Mão de Obra (Desenvolvimento) & 50.000,00 & 99,50\% \\
        \hline
        Infraestrutura (AWS, Domínio) & 250,00 & 0,50\% \\
        \hline\hline
        \textbf{INVESTIMENTO INICIAL TOTAL} & \textbf{50.250,00} & \textbf{100,00\%} \\
        \hline
    \end{tabular}
    \flushleft{\footnotesize Fonte: Elaborado pelo autor (2025).}
\end{table}

Conforme apresentado na Tabela \ref{tab:custos_iniciais}, o Investimento Inicial Total do projeto é de R\$ 50.250,00.

% ----------------------------------------------------------
% CAPÍTULO 5: CONSIDERAÇÕES FINAIS
% ----------------------------------------------------------
\chapter{Considerações Finais}\label{cap:conclusao}

O presente Projeto Integrado (PI) alcançou seu objetivo geral, que era desenvolver o \textsf{CarneUp}, um Sistema de Gestão focado na rastreabilidade e no controle de perecibilidade de estoque para açougues de pequeno e médio porte.

Com a entrega do Produto Mínimo Viável (MVP), o projeto demonstrou a viabilidade técnica da integração de tecnologias modernas para atender ao problema de ineficiência na gestão analógica. A solução implementada propõe-se a atender às exigências do mercado e fornece um controle de validade e lote mais eficaz, validando a premissa de que a tecnologia pode mitigar riscos operacionais no core business do varejo de carnes.

\section{Conclusão dos Resultados e Análise Crítica}\label{sec:analise_critica}

Esta seção sintetiza os principais resultados alcançados, as decisões estratégicas tomadas e o alinhamento do projeto com sua proposta de valor.

\subsection{Alinhamento com a Proposta de Valor e Viabilidade}\label{sub:alinhamento_viabilidade}

O \textsf{CarneUp} cumpriu seus requisitos técnicos, concentrando o foco em um sistema robusto e confiável. A decisão de utilizar o PostgreSQL foi fundamental para oferecer a robustez e a integridade transacional exigidas para dados financeiros e de estoque. O Investimento Inicial total estimado para o desenvolvimento do MVP foi de R\$ 50.250,00.

\subsection{Dificuldades e Lições Aprendidas}\label{sub:licoes_aprendidas}

O desenvolvimento foi marcado por desafios que exigiram adaptação metodológica e rigor na gestão de escopo.

\begin{itemize}
    \item Restrição Temporal e Escopo: A principal dificuldade foi a condensação do cronograma para apenas 2,5 meses, imposta pelos Marcos Fixos de Entrega Acadêmica. Essa restrição exigiu uma priorização agressiva e um controle rigoroso do escopo.
    \item Modelagem de Dados Complexa: A rastreabilidade por lote e validade demandou um esforço significativo na fase de modelagem, com refinamentos repetidos nos diagramas (MER/DER). Isso evidenciou a complexidade inerente à representação de processos de estoque perecível.
\end{itemize}

A adoção da Metodologia \textit{Scrum} permitiu a entrega incremental em Sprints curtas, um fator que foi determinante para a conclusão do MVP dentro do prazo.

\section{Limitações e Trabalhos Futuros}\label{sec:trabalhos_futuros}

As limitações impostas pelo prazo rigoroso resultaram em funcionalidades que foram deliberadamente postergadas, estabelecendo o escopo para a Fase 2 do projeto, que será objeto de Trabalhos Futuros.

\begin{itemize}
    \item Módulo de Contabilidade Completa: O foco foi mantido no core business (Venda, Estoque e Custo da Peça). Módulos avançados de geração de balancetes ou integração contábil foram adiados.
    \item Suporte a Múltiplas Filiais: O sistema foi dimensionado inicialmente para um único açougue. A expansão para uma arquitetura multi-filial será o foco principal para a otimização da escalabilidade do produto no futuro.
\end{itemize}

O desenvolvimento subsequente deve incluir a implementação dessas funcionalidades e a otimização contínua da interface do usuário (UI) baseada em *feedback* de clientes.

% ----------------------------------------------------------
% ELEMENTOS PÓS-TEXTUAIS
%
% *** CERTIFIQUE-SE QUE ESTA É A PRIMEIRA E ÚNICA CHAMADA DE \postextual ***
%
\clearpage
\postextual 
% ----------------------------------------------------------

% ----------------------------------------------------------
% Referências bibliográficas (Primeiro elemento pós-textual)
% ----------------------------------------------------------
\bibliography{Referencias}
\cleardoublepage

% ----------------------------------------------------------
% ANEXOS
% ----------------------------------------------------------

% Inicia o ambiente formal de Anexos (no plural)
\begin{anexos}

\chapter*{ANEXO A – PROJETO DE BANCO DE DADOS E DICIONÁRIO DE DADOS}
% Note que a numeração das tabelas e figuras aqui será reiniciada para A.1, A.2, etc.

O presente anexo detalha o projeto do banco de dados relacional (PostgreSQL) do sistema CarneUp, abrangendo os níveis de abstração, o modelo conceitual e o dicionário de dados (modelo lógico) completo.

\section{Modelos de Abstração}

A Tabela \ref{tab:modelos_abstracao} define os diferentes níveis de abstração utilizados no projeto do banco de dados.

\begin{table}[htb]
    \centering
    \caption{Níveis de Abstração do Projeto de Banco de Dados}
    \label{tab:modelos_abstracao}
    \begin{tabularx}{\linewidth}{|l|X|l|X|X|}
    \hline
    \textbf{Modelo} & \textbf{Nível de Abstração} & \textbf{Dependência do SGBD} & \textbf{Representação} & \textbf{Foco} \\
    \hline
    Conceitual & Mais Alto & Não (Puro negócio) & Diagrama Entidade-Relacionamento (MER) & O que é o negócio (Entidades e Regras) \\
    \hline
    Lógico & Intermediário & Não (Puro modelo relacional) & Esquema de Tabelas e Chaves (Lógico) & Como os dados serão estruturados (Relações, PK, FK, 3FN) \\
    \hline
    Físico & Mais Baixo & Sim (Dependente do SQL/SGBD) & Comandos SQL DDL, Índices, Triggers & Como o SGBD irá armazenar e otimizar (Tipos de dados, Triggers) \\
    \hline
    \end{tabularx}
\end{table}

\section{Modelo Conceitual (MER)}

É a representação de alto nível das entidades e seus relacionamentos, focado na regra de negócio.

\subsection{Entidades Principais}
\begin{table}[htb]
    \centering
    \caption{Entidades Principais do Modelo Conceitual}
    \label{tab:entidades_mer}
    \begin{tabularx}{\linewidth}{|l|X|X|}
    \hline
    \textbf{Entidade} & \textbf{Descrição} & \textbf{Atributos Principais (Exemplos)} \\
    \hline
    PRODUTO & Carnes/itens vendidos. & ID\_Produto (Chave), Nome, Corte, Estoque\_Mínimo. \\
    \hline
    CATEGORIA & Agrupa os produtos (ex: Bovino, Suíno). & ID\_Categoria (Chave), Nome\_Categoria. \\
    \hline
    FORNECEDOR & Empresas de quem se compra. & ID\_Fornecedor (Chave), Nome\_Fornecedor, CNPJ. \\
    \hline
    USUARIO & Pessoas que acessam o sistema (vendedores/administradores). & ID\_Usuario (Chave), Nome, Cargo. \\
    \hline
    \end{tabularx}
\end{table}

\section{Modelo Lógico - Dicionário de Dados}

O Dicionário de Dados detalha a estrutura de cada tabela, tipos de dados, chaves e restrições de integridade referencial.

\subsection{Tabela PRODUTO}
\begin{longtable}{|p{2.5cm}|p{2.8cm}|p{1.5cm}|p{6cm}|}
    \caption{Estrutura da Tabela PRODUTO}
    \label{tab:ddproduto} \\ % Rótulo corrigido
    \hline
    \textbf{Coluna} & \textbf{Tipo de Dado} & \textbf{Chave} & \textbf{Restrições} \\
    \hline
    \endfirsthead
    \multicolumn{4}{c}%
    {{\bfseries Tabela \thetable{} -- continuação}} \\
    \hline
    \textbf{Coluna} & \textbf{Tipo de Dado} & \textbf{Chave} & \textbf{Restrições} \\
    \hline
    \endhead
    \hline
    \multicolumn{4}{|r|}{{\bfseries Continua na próxima página}} \\
    \hline
    \endfoot
    \hline
    \endlastfoot
    id\_produto & INT & PK & NOT NULL, AUTO\_INCREMENT \\
    id\_categoria & INT & FK & NOT NULL (Referencia CATEGORIA) \\
    nome & VARCHAR(100) & & NOT NULL \\
    corte & VARCHAR(50) & & NULL \\
    estoque\_atual & DECIMAL(10, 3) & & NOT NULL \\
    estoque\_minimo & DECIMAL(10, 3) & & NOT NULL \\
    data\_cadastro & DATE & & NOT NULL \\
\end{longtable}

\subsection{Tabela CATEGORIA}
\begin{longtable}{|p{2.5cm}|p{2.8cm}|p{1.5cm}|p{6cm}|}
    \caption{Estrutura da Tabela CATEGORIA}
    \label{tab:ddcategoria} \\ % Rótulo corrigido
    \hline
    \textbf{Coluna} & \textbf{Tipo de Dado} & \textbf{Chave} & \textbf{Restrições} \\
    \hline
    \endfirsthead
    \multicolumn{4}{c}%
    {{\bfseries Tabela \thetable{} -- continuação}} \\
    \hline
    \textbf{Coluna} & \textbf{Tipo de Dado} & \textbf{Chave} & \textbf{Restrições} \\
    \hline
    \endhead
    \hline
    \multicolumn{4}{|r|}{{\bfseries Continua na próxima página}} \\
    \hline
    \endfoot
    \hline
    \endlastfoot
    id\_categoria & INT & PK & NOT NULL, AUTO\_INCREMENT \\
    nome\_categoria & VARCHAR(50) & & NOT NULL \\
    descricao & VARCHAR(255) & & NULL \\
\end{longtable}

\subsection{Tabela FORNECEDOR}
\begin{longtable}{|p{2.5cm}|p{2.8cm}|p{1.5cm}|p{6cm}|}
    \caption{Estrutura da Tabela FORNECEDOR}
    \label{tab:ddfornecedor} \\ % Rótulo corrigido
    \hline
    \textbf{Coluna} & \textbf{Tipo de Dado} & \textbf{Chave} & \textbf{Restrições} \\
    \hline
    \endfirsthead
    \multicolumn{4}{c}%
    {{\bfseries Tabela \thetable{} -- continuação}} \\
    \hline
    \textbf{Coluna} & \textbf{Tipo de Dado} & \textbf{Chave} & \textbf{Restrições} \\
    \hline
    \endhead
    \hline
    \multicolumn{4}{|r|}{{\bfseries Continua na próxima página}} \\
    \hline
    \endfoot
    \hline
    \endlastfoot
    id\_fornecedor & INT & PK & NOT NULL, AUTO\_INCREMENT \\
    nome\_fornecedor & VARCHAR(150) & & NOT NULL \\
    cnpj & VARCHAR(18) & & NOT NULL, UNIQUE \\
    contato & VARCHAR(100) & & NULL \\
\end{longtable}

\subsection{Tabela USUARIO}
\begin{longtable}{|p{2.5cm}|p{2.8cm}|p{1.5cm}|p{6cm}|}
    \caption{Estrutura da Tabela USUARIO}
    \label{tab:ddusuario} \\ % Rótulo corrigido
    \hline
    \textbf{Coluna} & \textbf{Tipo de Dado} & \textbf{Chave} & \textbf{Restrições} \\
    \hline
    \endfirsthead
    \multicolumn{4}{c}%
    {{\bfseries Tabela \thetable{} -- continuação}} \\
    \hline
    \textbf{Coluna} & \textbf{Tipo de Dado} & \textbf{Chave} & \textbf{Restrições} \\
    \hline
    \endhead
    \hline
    \multicolumn{4}{|r|}{{\bfseries Continua na próxima página}} \\
    \hline
    \endfoot
    \hline
    \endlastfoot
    id\_usuario & INT & PK & NOT NULL, AUTO\_INCREMENT \\
    nome & VARCHAR(100) & & NOT NULL \\
    email & VARCHAR(150) & & NOT NULL, UNIQUE \\
    senha\_hash & VARCHAR(255) & & NOT NULL \\
    cargo & ENUM ('Administrador', 'Vendedor') & & NOT NULL \\
\end{longtable}

\subsection{Tabela ESTOQUE}
\begin{longtable}{|p{2.5cm}|p{2.8cm}|p{1.5cm}|p{6cm}|}
    \caption{Estrutura da Tabela ESTOQUE}
    \label{tab:ddestoque} \\ % Rótulo corrigido
    \hline
    \textbf{Coluna} & \textbf{Tipo de Dado} & \textbf{Chave} & \textbf{Restrições} \\
    \hline
    \endfirsthead
    \multicolumn{4}{c}%
    {{\bfseries Tabela \thetable{} -- continuação}} \\
    \hline
    \textbf{Coluna} & \textbf{Tipo de Dado} & \textbf{Chave} & \textbf{Restrições} \\
    \hline
    \endhead
    \hline
    \multicolumn{4}{|r|}{{\bfseries Continua na próxima página}} \\
    \hline
    \endfoot
    \hline
    \endlastfoot
    id\_lote & INT & PK & NOT NULL, AUTO\_INCREMENT \\
    id\_produto & INT & FK & NOT NULL (Referencia PRODUTO) \\
    quantidade & DECIMAL(10, 3) & & NOT NULL \\
    custo\_unitario & DECIMAL(10, 2) & & NOT NULL \\
    data\_validade & DATE & & NOT NULL \\
    data\_entrada & DATETIME & & NOT NULL \\
\end{longtable}

\subsection{Tabela COMPRA}
\begin{longtable}{|p{2.5cm}|p{2.8cm}|p{1.5cm}|p{6cm}|}
    \caption{Estrutura da Tabela COMPRA}
    \label{tab:ddcompra} \\ % Rótulo corrigido
    \hline
    \textbf{Coluna} & \textbf{Tipo de Dado} & \textbf{Chave} & \textbf{Restrições} \\
    \hline
    \endfirsthead
    \multicolumn{4}{c}%
    {{\bfseries Tabela \thetable{} -- continuação}} \\
    \hline
    \textbf{Coluna} & \textbf{Tipo de Dado} & \textbf{Chave} & \textbf{Restrições} \\
    \hline
    \endhead
    \hline
    \multicolumn{4}{|r|}{{\bfseries Continua na próxima página}} \\
    \hline
    \endfoot
    \hline
    \endlastfoot
    id\_compra & INT & PK & NOT NULL, AUTO\_INCREMENT \\
    id\_fornecedor & INT & FK & NOT NULL (Referencia FORNECEDOR) \\
    data\_compra & DATE & & NOT NULL \\
    valor\_total\_compra & DECIMAL(10, 2) & & NOT NULL \\
    status\_compra & ENUM ('Concluída', 'Pendente', 'Cancelada') & & NOT NULL \\
\end{longtable}

\subsection{Tabela ITEM\_COMPRA}
\begin{longtable}{|p{2.5cm}|p{2.8cm}|p{1.5cm}|p{6cm}|}
    \caption{Estrutura da Tabela ITEM\_COMPRA}
    \label{tab:dditemcompra} \\ % Rótulo corrigido
    \hline
    \textbf{Coluna} & \textbf{Tipo de Dado} & \textbf{Chave} & \textbf{Restrições} \\
    \hline
    \endfirsthead
    \multicolumn{4}{c}%
    {{\bfseries Tabela \thetable{} -- continuação}} \\
    \hline
    \textbf{Coluna} & \textbf{Tipo de Dado} & \textbf{Chave} & \textbf{Restrições} \\
    \hline
    \endhead
    \hline
    \multicolumn{4}{|r|}{{\bfseries Continua na próxima página}} \\
    \hline
    \endfoot
    \hline
    \endlastfoot
    id\_compra & INT & PK, FK & NOT NULL (Referencia COMPRA) \\
    id\_produto & INT & PK, FK & NOT NULL (Referencia PRODUTO) \\
    quantidade & DECIMAL(10, 3) & & NOT NULL \\
    custo\_unitario & DECIMAL(10, 2) & & NOT NULL \\
\end{longtable}

\subsection{Tabela VENDA}
\begin{longtable}{|p{2.5cm}|p{2.8cm}|p{1.5cm}|p{6cm}|}
    \caption{Estrutura da Tabela VENDA}
    \label{tab:ddvenda} \\ % Rótulo corrigido
    \hline
    \textbf{Coluna} & \textbf{Tipo de Dado} & \textbf{Chave} & \textbf{Restrições} \\
    \hline
    \endfirsthead
    \multicolumn{4}{c}%
    {{\bfseries Tabela \thetable{} -- continuação}} \\
    \hline
    \textbf{Coluna} & \textbf{Tipo de Dado} & \textbf{Chave} & \textbf{Restrições} \\
    \hline
    \endhead
    \hline
    \multicolumn{4}{|r|}{{\bfseries Continua na próxima página}} \\
    \hline
    \endfoot
    \hline
    \endlastfoot
    id\_venda & INT & PK & NOT NULL, AUTO\_INCREMENT \\
    id\_usuario & INT & FK & NOT NULL (Referencia USUARIO) \\
    data\_venda & DATETIME & & NOT NULL \\
    valor\_total\_venda & DECIMAL(10, 2) & & NOT NULL \\
    status\_venda & ENUM ('Finalizada', 'Cancelada') & & NOT NULL \\
\end{longtable}

\subsection{Tabela ITEM\_VENDA}
\begin{longtable}{|p{2.5cm}|p{2.8cm}|p{1.5cm}|p{6cm}|}
    \caption{Estrutura da Tabela ITEM\_VENDA}
    \label{tab:dditemvenda} \\ % Rótulo corrigido
    \hline
    \textbf{Coluna} & \textbf{Tipo de Dado} & \textbf{Chave} & \textbf{Restrições} \\
    \hline
    \endfirsthead
    \multicolumn{4}{c}%
    {{\bfseries Tabela \thetable{} -- continuação}} \\
    \hline
    \textbf{Coluna} & \textbf{Tipo de Dado} & \textbf{Chave} & \textbf{Restrições} \\
    \hline
    \endhead
    \hline
    \multicolumn{4}{|r|}{{\bfseries Continua na próxima página}} \\
    \hline
    \endfoot
    \hline
    \endlastfoot
    id\_venda & INT & PK, FK & NOT NULL (Referencia VENDA) \\
    id\_produto & INT & PK, FK & NOT NULL (Referencia PRODUTO) \\
    quantidade & DECIMAL(10, 3) & & NOT NULL \\
    preco\_unitario & DECIMAL(10, 2) & & NOT NULL \\
    custo\_unitario\_registro & DECIMAL(10, 2) & & NOT NULL (Crucial para cálculo de lucro) \\
\end{longtable}

\end{anexos}

\end{document}
